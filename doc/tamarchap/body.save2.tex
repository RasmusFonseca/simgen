\section{Implementation}

\subsection{Overview}

As outlined in the introduction, the nature of the macromolecular
structure data is clearly hierarchic so it is natural to turn to a
hierarchic data structure to capture it.  With a view to generality,
it was our aim that the structure of each unit in the hierarchy at 
any level should be identical with the distinction between levels
being made only through externally specified (user) parameters.
The most fundamental aspects to be specified include the shape of
the unit, its linkage to other units (such as whether it is in a
chain or multiply bonded) and how units interact with each other.
For ease of discription, a unit contained in a higher level unit
will be called its "child" with the higher level referred to as
the "parent" of the lower level unit.   Units with the same parent
are therefore "siblings" and if they form a chain, are designated
"sister" (preceeding) and "brother" (following).   Anthropomorphic
terms based on family relationships will be used through.

Interactions can be distinguished as inter- and intra-level.
Within a level (intra) these will either be repulsive (active for
a pair of units in collision) or attractive (active between a
bonded pair).  However, there are no bonds between levels
(unless individually specified by the user) and no bumps between
levels.  Inter-level interaction consists only of coordination of
motion and containment.  Coordinate movement means that when a parent
moves, all its children move too, which by implication, continues
down through all successive generations.  In the opposite direction,
the centroid of the children determines the position of the parent. 
This relationship also implies indirect interaction between the 
levels so that when children in different families collide then
their parents will also experience a lesser repulsion.
Containment was implemented as a simple kick-back to any children
that had 'strayed' beyond the shape boundary of their parent.

The motion of each unit is purely random with a fixed stepsize
defined in the user-specified parameter file.  Every move is
accepted whether it violates bond geometry of leads to collisions.
Clearly this would lead to a degregation in the molecular structure
so independently of this imposed motion, the bond lengths and local
chain geometry (if there is a chain) are continually refined towards
their initial configuration.  Similarly, units in collision are
also corrected.   All these processes run concurrently (implemented
as separate threads) and in addition a user specified process is
also runs in parallel in which the directed elements of the model
are specified.   All of these processes operate on a single
representation of the coordinates so care has been taken to ensure
that undefined states do not arise in one process that would 
disrupt another.  In general, this can be avoided with each 
process working on a temporary copy of the coordinates it needs
then writing these in one step back into the structure.

The overall structure of the implemation is shown in Figure 1 along
with the names of the processes that will be referred to below.
The two processes that maintain the integrety of the molecular structure
are the collision detection and correction process and the process that
maintains the specified links in the chain: respectively, called the 
{\tt bumper} and the {\tt linker} which will be considered first.

\subsection{Steric exclusion}

It is intended that the implementation should be applied to very large
systems so any collision detection based on a full pairwise algorithm
would be impractical.  This is commonly avoided in molecular (and other)
dynamics programs through the use of a neighbour list in which each atom
maintains a list of its current neighbours and checks only these for
collision.  This has the unavoiadable problem that the list must be
revised periodically.   We adopted a similar approach except that we
used the hierarchic structure of the data to provide built-in neighbour
lists where any unit only checks its siblings for collisions.  As each
family is usually in the order of 10--100 units, this would not be a
large task to compute in a pairwise manner, however, we use a faster
approach based on ranked lists of units that are maintained for each
dimension (X.Y,Z).   As sorting avoids quadratic operations in the
number of objects, this is much faster for large families.

When two units are in collision, they are repeled only if they have no
children.  Again a single (user specified) kick-back step is applied
equally to each unit along the line connecting their centres.
If the units have children, then collisions are found between their
joint families within a restricted range along the line connecting the
two parent centres.  If there are no grandchildren, repulsion is again
implemented along the line connecting their two centres, otherwise the
process is repeated at each lower level until the lowest (atomic
\footnote{The term "atomic" only means the lowest level in the 
hierarchy, which in the protein applications discussed, is the
residue level (based on the \CA\ position).}

As mentioned above, parents of different families will automatically
adjust their position indirectly to the repulsion between their families
as the centroid of the children will move slightly apart.   This is 
often a small effect, and before it becomes significant, the children 
can become bunched at the collision interface which has the reverse
effect of bringing the parents (and hence their children) even closer
together.   To avoid this, on the return path from the traversal of the
family hierarchy (ie: revisiting the parents of the colliding children),
the parents themselves are given a small direct displacemnt proportional
to the number of their children that were in collision.   If only the 
positions of the atomic level units were observed, this would have the
appearance of a repulsive 'field'  as there would seem to be "action
at a distance" across a family.  Alternatively it can be imagined that
the children are embedded in a soft parental jelly-like matrix.
Computationally the approach means that in any collision at a high
level, there will be fewer low-level collisions generated which saves
on computational expense.

This approach to the treatment of collisions has an additional effect
in that it is relatively insensitive to the shape of the colliding
objects, which in our implementation can be spheres,
ellipsoids or tubes (discussed in mode detail below).  If we assume
that the atomic level consists of spheres, then different shapes at 
higher levels are primarilly defined by the distribution of their
children within them which in turn is determined by the shape within
which the family is confined.  The only discrepancy comes through
the point at which units at the parental level detect collision as,
for computational simplicity, this is kept as an isotropic test at the
radius of their (user defined) bumping sphere.   Objects that are
long/thin tubes or extremly prolate elipsoids which are fully contained
in a bumping sphere can therefore be considered in collision before
any of their family members come in contact.  At worst, however, this
is just a slight waste of computer time as the count of colliding
children will be zero and the parents will not respond.

\subsection{Polymers and Cross-linking}

\subsubsection{Specifying chain connectivity}

Unless liquids are bring modelled, the links between units (which are not
distinguished from bonds) are the components that impart greatest structure.
For biological polymers, links between just adjacent units along a chain,
combined with steric exclusion, are sufficient to define a basic model. However,
even this, apparently simple, imposition of structure leads to complications
in a hierarchic model.  If the atomic level is a linear chain, then so too
are all higher levels but this is not so if each atomic family forms a separate
or a circular chain.  Then higher levels can be unlinked (eg; a liquid of cyclic
peptides) or otherwise linked in their own way.  The linkage polymer state of
each level can be specified by the user independently but chain interdependencies
are checked internally and imposed by the program.

The structural hierarchy (family structure) is not determined by chain
connectivity but should be based on groups of units that will tend to move
together as they are all acted upon by their parent's transform operations.
At the lower level, such groupings will typically consist of consecuitive 
units along a chain (such as an \AH), however, at the domain level and
secondary structure level for RNA, families of units will also be composed of
sequentially discontinuous segments.   Computationally, this requires some
book-keeping to keep note of which members are linked across families.  To
facilitate this, each unit holds a record of its preceding unit, referred to
as its "sister", and its following unit, referred to as its "brother"; both
of which may specify any unit in any family at the same level.  Together,
the sisters and brothers constitute two linked-lists, with brothers running
from the start of the chain to the end and sister in the opposite direction.

As chain connectivity is not restricted by family groupings, its path at 
the next higher level is not necessarilly linear and can be branched.  This
means that each unit may have more than one brother or sister which is
equivalent to branches in the chain in both directions.  In general, this
specifies a network and to deal with the associated "book-keeping", each
unit holds a stack of its brothers and sisters.   Following a chain is therefore
not simple and when listing a chain in "sequential order", the lists of brothers
and sisters are followed recursively from any given starting unit.

\subsubsection{Inter-chain cross-links}

Polymer chain links are such a common feature of biological macromolecules that
the capacity to encode them was included as a general feature in the data structure
of each unit.   Inter-chain cross-links are less ubiquitous are were allocated
only as requested by the user in the data file that specifies the model.  For any
level a fixed number of links could be specified, not all of which need necessarily
be used.   If no linking capacity was specified then computer memory was not
allocated.

Although links can be individually specified, some automated features were 
incorporated to ease the burden of assigning the local cross-links associated with
secondary structure, both in proteins and RNA.  For proteins, two types of secondary
structure can be defined: the \AH\ and \BS.   The former is purely local and two
links were automatically set to the relative chain positions +4 and -3 of the 
ideal length found in proteins.  Similarly, two local links were made along a \Bs
to the +2 and -2 positions.   However, each strand in a sheet makes non-local
links which can be specified by data provided in the coordinate input specification
which is automatically generated by a separate program that calculates the
definition of secondary structures.    Similarly, for RNA, the base-pairings are
pre-calculated by a separate program.

At a higher level, links between secondary structures can be allocated within a
given range of interaction.   Unlike the secondary structures which are pre-calculated,
these links are computed as part of the data input process.

\subsection{Geometric regularisation}

Steric exclusion combined with the range of linkage described above can generate a
relatively stable structure.   However, given a background of "thermal" noise, any
less constrained parts of the structure will be free to diverge from their starting
configuration under the given distance constraints.  Typically, this involves twisting
and shearing that can generate large motions with little violation of the specified
distances, which in principle, cannot constrain chirality.  A general mechanism,
based only on local angles and distances was provided to reduce these distortions
and was applied equally to all levels that form a chain.

In a chain segment of five units (designated: b2,b1,c0,a1,a2), six distances were
recorded from the starting configuration in the upper half of the matrix of pairwise
distances excluding adjacent units.  Three angles were also recored as b1-c0-a1 and
the torsion angles around b1-c0 and c0-a1.   These local distances were continually
refined as were the angles.  Distances can be regularised with little disruption,
however, refining torsion angles can sometimes lead to an error propagation
with dramatic effects.  To limit the potential for this the torsion angles were
dialed-up exactly to generate new positions for b2 (b2') and a2 (a2').   These were
used to form a basis-set of unit length vectors along: $x = a2'-b2', y = c0-(a2'+b2')/2$,
with $z$ mutually orthogonal.   Starting from the centroid of the five points, the 
coeficients of an equivalent basis-set defined on the original positions were applied
to the new basis-set to generate the new coordinate positions.  The result is a
compromise between angle and position that remains stable over repeated application.

Although only local information is used, its application over all levels leads to a
global effect and indeed is sufficient by itself to recapitulate a large structure.
As the procedure was designed to correct defects caused by the addition of random
motion (caused by {\tt mover} in Fig.1), it was not implemented as an independent
parallel process but included in {\tt mover} and applied after the coordinates had been 
displaced, so keeping a balance between disruption and correction.  A final feature
was included to allow for the necessary requirement that in a dynamic model, the
starting configuration of the structure should not be exclusively maintained.  This
was accommodated by periodically shifting the target distances and angles towards
those found in the current configuration.   The overall effect of this procedure is
to provide a buffering effect against random motion and is similar to giving
rigidity to the structure but still allowing movement under a persistent "force".
In the current implementation, this shift is by 1\% once in roughly every 100
activations of {\tt mover}.  This may need to be adjusted depending on the appliaction.

\subsection{Shape specfication}

Three basic shapes were implemented: cylinder, ellipsoid and sphere.  Although the 
sphere is a special instance of an ellipsoid, there are implementation details,
described below, that make them distinct.   Each shape type by itself has elements
of symmetry that can make their orientation arbitrary, however, this symmetry is
broken when a unit contains children in an irregular configuration.  Thus each unit
needs to have an associated reference frame that determines its orientation and is
acted on by rotational operations.   For a unit in a chain, the current reference 
frame is based on the direction from its sister to brother (X) with the Y direction 
as the projection of the unit's position onto this line in an orthogonal direction
and Z as their mutual perpendicular.   A consequence of this is that flexing of the
chain does not preserve the end-point distances between consecuitive cylinders or
ellipsoids along the chain.

The length of cylinders and ellisoids is set by reading in two end-points from the
input data which have been precalculated from the inertial axes of
the point-set that comprises the current unit (say, a secondary structure element or
a domain).  As well as the length, the line linking these end-points specifies the
axis that corresponds with the X direction in the internal reference frame. 
The two end-points are set within the data-structure that defines each unit as
two points equidistant from the central point along the X direction.   The size of
each unit on a given level is specified by a value specified in the parameter file
that describes the model and does 
