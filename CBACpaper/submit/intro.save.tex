\subsection{Introduction}

Molecular simulation methods are increasingly being applied to ever larger systems.
However, given finite computational resources, there is enevitably a limit to the
size of the system that can be simulated in a reasonable time.   Approaches to 
circumvent this limitation gemmerally follow the original approach of Levitt and
reduce the number of simulated particles by combining groups of atoms, such as an
amino acid side-chain, into a single pseudo-atom with a spherical radius that
reflects the volume of the combined atoms.
This coarsd-grainned approach suffers from the problem that as the number of combined
atoms increases in number, so their representation becomes less realistic.  Taken
to an extreme degree, if a protein is representated by a single sphere, then all 
the details ofits structure become hidden.  

In this work, I describe a development of an earlier algorithm for collision 
detection between groups of multiple points into a general method that allows
points (atoms) to be grouped into a variety of shapes but still retaining the 
property that the interaction (or collision) of the higher-level objects is based
on the interaction of their component atoms.

\subsection{Hierarchic collision detection}

Fas collision detection in interactive computer game simulations is enabled by the
use of a bounding-box construct.  This is a box in which a group of points are contained
and the calculation of the interactions between points in different boxes is not
evaluated until their boxes overlap.    This approach is similar to, but distinct from,
the use of neighbour-lists in molecular dynamics (MD).   In general, the bounding
box can be any shape but should, ideally have a simple shape to allow for fast
overlap calculation.    The advantage of a box that is alighned with the coordinate
frame is that only X,Y,Z values need be compared, without the more costly calculation
of a 3D distance.  

Previously, this approach was used to speed collision detection
between atoms using a simple spherical "box" but when dealing with non-spherical objects,
such as alpha-helices of nucleic acid segments, a sphere is not an ideal shape and
when made large enough to enclose an elongated object many other objects can be
brought into the calculation even when they are far from interacting, especially if
they too are elongated.   In this work, the original method based on spherers is
extended to a wider variet of shapes and into a generalised hierarchy in which the
boxes themselves can be assigned different collision properties at any level in the
hierarchy. 

For example; attributing a box object with hard-shell collision behaviour is equivalent to
ignoring all their internal components when two objects collide.    If this absolute
degree of repulsion is softened, then the objects can now partly interpenetrate,
allowing their internal components to come into contact, with their own collision
properties contributing to the interaction.   If the high-level objects do not
repel at all, then all replusion will be determined by the structure and properties of
the internal components.

Through this approach, the coarse-grainned representation at the high-level does not
mask the details of the interaction at the lower level, however, it is still being
used to save computation time, especially where it retains a good shape match to the
interacting surface of its components.

\subsection{Object descriptions}
 
The choice of shapes for the higher level objects (or containers) is, in principle,
not limited but simple shapes have been chosen thatreflect those encountered in 
biological models.   These include a sphere which can generalised as an oblate or
prolate ellipsoid (but not scalene, as will be discussed below) and a tube (or more
precisely, a fixed-length straight section of pipe).   The use of the original box
registered on the coordinate frame was avoided because, unlike objects in computer
games, in the molecular world the dominant orientation dictated by gravity is absent. 
Objects at any level can adopt any mix of these basic shapes which can be different 
sizes and differing degrees of eccintrictiy for ellipsoids (from "cigar" to "flying 
saucer") and for tubes, have differing length:radius ratios (from "coin" to "pencil").

For each of these objects and their pairwise interactions, it is necessary to have 
a fast algorithm to compute their surface and the point at which their surfaces
make contact.   For spheres, both these are trivial and between fixed-length straight
tubes, both with themselves and spheres, the calculations require only slightly more
"book-keeping".    However, the interaction of ellipsoids is less simple and and 
an analytic solution for the contact-normal between two ellipsoids is far from
trivial.    Although it is slightly simpler for non-scalene ellipsoids, an analytic
solution was passed over in favour of a fast iterative heuristic.  

\subsection{Coupling the levels}

It would be of little use of objects were to wander outside their container as their
interactions would not be detected until their containers eventually made contact.
Although this might be avoided or reduced by having a large container, for computational
efficency, it is better if the container maintains a close fit to its contents.

The task of ensuring a good match between the parent container and its enclosed
children was obtained it two main ways: firstly, by maintaining the cetre of the
parent parent container at the centroid of its children and secondly, by applying any
displacement or rotation of the parent automatically to all its children and recursively
to any children that they also contain.   However, while these two conditions serve
to synchronise the motion between levels, they do not prevent the escape of children
through isotropic diffusion.   This was tackled directly by applying a corrective
push to return any wayward children back into the parental fold.
These couplings constitute the only direct connection between levels in the hierarchy
of objects as there are no collisions possible between objects at different levels.

Despite their simplicity, these couplings are sufficient to generate appropriate
behaviour during collisions.   For example; if at one extreme, the parent shapes interact
as hard surfaces then in a collision, all their contents will move with them during
recoile.   At the other extreme, if the parents have no repulsion but their children do,
then the parent containers will interpenetrate, allowing collisions to occur between
the children which will repel each other and as a result, their centroids will 
separate with the centres of their parents tracking this separation.   In the intermediate
situation, where the parents retain some repulsive behaviour, their interpenetration
will be reduced like the collision of two soft bodies but retaining the hard-shell
repulsion between children.   

An option was added to exaggerate this "current-in-jelly" collision model by making
the soft parental repulsion dependent on the number of colliding children.   This
allows two parental containers to pass through each other undeflected until their
children clash: at which point, the parental repulsion becomes active and guides the
two families of points apart before extensive interaction occurs between the 
children.   Such behaviour prevents the separation of the parents being completely
dependent on the displacement of the children as in high speed collisions, any
internal struture within the children can be disrupted before separation is attained.

A further modification was introduced to deal with the possible situation where
two families have collided and overlapped to such an extent that there is no
dominant direction of separation provided by the children. 
