\subsection{Overview}

\subsubsection{Introduction}

Molecular simulation methods are increasingly being applied to ever larger systems.
However, given finite computational resources, there is inevitably a limit to the
size of the system that can be simulated in a reasonable time.   Approaches to 
circumvent this limitation generally follow the original approach of Levitt and Warshel
\cite{LevittMet75}
and reduce the number of simulated particles by combining groups of atoms, such as an
amino acid side-chain, into a single pseudo-atom with a spherical radius that
reflects the volume of the combined atoms
\cite{BondPJet07,IzvekovSet05}.
This coarse-grained approach suffers from the problem that as the number of combined
atoms increases in number, so their representation becomes less realistic.  Taken
to an extreme degree, if a protein is represented by a single sphere, then all 
the details of its structure become hidden.  

In this work, I describe the development of an earlier algorithm for collision 
detection between groups of multiple points 
\cite{KatsimitsouliaZet10a,TaylorWRet10a}
into a general method that allows
points (atoms\footnote{
In the following description the use of the terms "atom" and "atomic" are used
only to indicate the lowest level in a hierarchy of objects (the leaf nodes).
Although they are restricted to a spherical shape they do not necessarily
represents atoms (in the chemical sense). 
}) to be contained within a variety of shapes but still retaining the 
property that the interaction (or collision) of these higher-level objects is based
on contact between their component atoms.
The approach follows a divide-and-conquer strategy in which the problem of
dealing with a quadratic computational complexity in the number of points
is reduced by partitioning the interactions into a series of grouped interactions
on a hierarchic tree.

\subsubsection{Hierarchic collision detection}

Fast collision detection in interactive computer game simulations is enabled by the
use of a bounding-box construct.  This is a box in which a group of points are contained
and the calculation of the interactions between points in different boxes is not
evaluated until their boxes overlap.    This approach is similar to, but distinct from,
the use of neighbour-lists in molecular dynamics (MD).   In general, the bounding
box can be any shape but should, ideally have a simple shape to allow for fast
overlap calculation.    The advantage of a box that is aligned with the coordinate
frame is that only X,Y,Z values need be compared, without the more costly calculation
of a 3D distance. Unfortunately, unlike objects in computer games, in the molecular
world the dominant orientation dictated by gravity is absent so a construct based
on the "world" coordinate frame is less relevant.

Previously, this approach was used to speed collision detection
between atoms using a simple spherical "box" 
\cite{TaylorWRet10a,TaylorWRet10b}
but when dealing with non-spherical objects,
such as alpha-helices of nucleic acid segments, a sphere is not an ideal shape and
when made large enough to enclose an elongated object many other objects can be
brought into the calculation even when they are far from interacting, especially if
they too are elongated.   In this work, the original method based on sphere's is
extended to a wider variety of shapes and into a generalised hierarchy in which the
boxes themselves can be assigned different collision properties at any level in the
hierarchy. 

For example; attributing a box object with hard-shell collision behaviour is equivalent to
ignoring all their internal components when two objects collide.    If this absolute
degree of repulsion is softened, then the objects can now partly interpenetrate,
allowing their internal components to come into contact, with their own collision
properties contributing to the interaction.   If the high-level objects do not
repel at all, then all repulsion will be determined by the structure and properties of
the internal components.

Through this approach, the coarse-grained representation at the high-level does not
completely mask the details of the interaction at the lower level, however, it is still being
used to save computation time, especially where it retains a good shape match to the
interacting surface of its components.


\subsection{Model specification}

\subsubsection{Object descriptions}
 
The choice of shapes for the higher level objects (or containers) is, in principle, not
limited but simple shapes have been chosen that reflect those encountered in biomolecular
models.   These include a sphere, which can be generalised as an oblate or prolate
ellipsoid\footnote{
Oblate and prolate ellipsoids, which have two equal axes (and hence a unique axis of symmetry)
are referred to jointly as spheroids, with scalene being the remaining assymetric type.  
However, to avoid confusion with spherical objects, the term will be avoided.}
(but not scalene, as will be discussed below) and a tube (or more precisely, a fixed-length
straight section of pipe with hemi-spherical end-caps).   The use of the original box
registered on the coordinate frame was avoided for reasons mentioned above and without
this registration, the overlap between arbitraily orientated rectangular boxes are are
not so simple.

Objects at any level can adopt any mix of these basic shapes which can be different 
sizes and for ellipsoids, have different degrees of eccentricity (from "cigar" to "flying 
saucer"), similarly, tubes can have differing length:radius ratios (from "coin" to "pencil").
However, as tubes have hemispherical end-caps, they approach a spherical shape as their
length decreases, as do ellipsoids as their axes become equal.
For each of these objects and their pairwise interactions, it is necessary to have 
a fast algorithm to compute their surface and the point at which their surfaces
make contact.   For spheres, both these are trivial and between fixed-length straight
tubes, both with themselves and spheres, the calculations require only slightly more
"book-keeping".    However, the interaction of ellipsoids is less simple and an analytic
solution for the contact-normal between two ellipsoids is not known and its numerical
solution is far from trivial.    Although it is slightly simpler for non-scalene ellipsoids,
an exact solution was passed over in favour of a fast iterative heuristic.  

\subsubsection{Coupling the levels}

It would be of little use if objects were to wander outside their container as their
interactions would not be detected until their containers eventually made contact.
Although this might be avoided or reduced by having a large container, for computational
efficiency, it is better if the container maintains a close fit to its contents.

The task of ensuring a good match between the parent container and its enclosed
children was obtained it two main ways: firstly, by maintaining the centre of the
parent container at the centroid of its children and secondly, by applying any
displacement or rotation of the parent automatically to all its children and recursively
to any children that they also contain.   However, while these two conditions serve
to synchronise the motion between levels, they do not prevent the escape of children
through isotropic diffusion.   This was tackled directly by applying a corrective push
to return any wayward children back into the parental fold.   With a few minor elaborations
described below, these couplings constitute the only direct connection between levels in the
hierarchy of objects as no collisions are possible between objects at different levels.

Despite their simplicity, these couplings are sufficient to generate appropriate
behaviour during collisions.   For example; if at one extreme, the parent shapes interact
as hard surfaces then in a collision, all their contents will move with them during
recoil.   At the other extreme, if the parents have no repulsion but their children do,
then the parent containers will interpenetrate, allowing collisions to occur between
the children which will repel each other and as a result, their centroids will 
separate with the centres of their parents tracking this displacement.   In the intermediate
situation, where the parents retain some repulsive behaviour, their inter-penetration
will be reduced like the collision of two soft bodies but retaining the hard-shell
repulsion between children.   

An option was added to exaggerate this "currants-in-jelly" collision model by making
the soft parental repulsion dependent on the number of colliding children.   This
allows two parental containers to pass through each other undeflected until their
children clash: at which point the parental repulsion becomes active and guides the
two families of points apart before extensive interaction occurs between the 
children.   Such behaviour prevents the separation of the parents being completely
dependent on the displacement of the children as in high speed collisions, any
internal structure within the children can be disrupted before separation is attained.

A further safeguard was introduced to deal with the possible situation where two
families of points have collided and overlapped to such an extent that there is no
dominant direction of separation provided by the children.   To avoid this family
grid-lock, children with different parents were not repelled along their contact
normal but each was given an additional vectorial component back towards their
parental centre.

\subsubsection{Parental realignment}

The way in which parents track their children was described above only in terms of
translation, but for objects other that the sphere, a rotational realignment must also
be considered.   Any parental rotation is automatically communicated to its children but
a rotation of the children, either caused randomly or by a push or pull on a group of
bonded children, does not get communicated up to the parent in the same way.

However, for both the ellipsoid and tube, there is a unique axis of symmetry and this can
be recalculated from the configuration of the children.  If the children are bonded
in a chain then the axis can be reset simply by considering a small group of positiona
around the termini (or jointly the complementary $5',3'$ pairs in double-stranded
nucleic acids).   More generally, the axis can be recalculated from the moments
of inertia of the point set, irrespective of their connectivity.
This computationally more expensive calculation was made much less frequently
compared to the positional tracking up-date. 

\subsection{Shape correspondance to molecular objects}

Although the mapping of the objects described above to molecular substructures is quite
arbitrary and could be divised "from scratch" with each application, there are some
natural associations and in the program that implements the methods (called \NAME),
the construction of these have been facilitated by specialised routines that parse
the input stream. (See \cite{TaylorWRet12c} for an outline).

\subsubsection{Proteins}

The "atomic" level for protein structure is assumed to be the chain constructed on
consecutive \CA\ positions.   By default, the atomic level is always a hard-sphere
and in the internal coordinate representation of 0.6 units to 3.8\AA\ (the \CA--\CA\
'bond' distance), the bump-radius is set to 3 which is the closest approach distance 
between positions i and i+2.   Bonded positions (i, i+1) do not bump.  

The natural object to represent secondary structure elements is a tube.   Preset
length:radius ratios are adopted for alpha, beta and coil structures with the ratio
determined by the axial displacement per residue for each secondary structure which
is 1.5 and 3.0 for alpha and beta, respectivefully.   Given the more irregular nature
of the coil regions, an arbitrary value of 0.5 was used but the coil residues were
only weakly constrained to keep within this (short) tube.

The domain level of structure is best represented by an ellipsoid \cite{TaylorWRet83b,TaylorWR83}.  
Although the average domain shape is a scalene ellipsoid (semiaxes A$\ne$B$\ne$C)
\cite{AszodiAet94a}, the program picks
the closest symmetric prolate ellipsoid (A$\ne$B=C) or oblate ellipsoid (A=B$\ne$C).

The overall protein envelope is again generally ellipsoidal, but this and higher
level (quaternary) assemblies are best determined on an individual basis.

\subsubsection{Nucleic acids}

The atomic level representation for DNA and RNA was taken as the phosphate atom
and in the internal scale, the P---P distance is 1 unit.

In double stranded RNA and DNA it is desirable to have base-paired phosphates
move as a linked entity and this was achieved by enclosing them at either end of
a tube.  As well as linking the phosphates, the tube provides a volume to mimic
the bulk of the (unrepresented) nucleotide bases.   The bulk of the bases should
lie between the midpoints of the P---P virtual bonds and in the refinement the
double helix geometry, which are distinct in RNA and DNA, this difference
was accommodated in the ideal distances and angles.

Segments of base-paired phosphate ladders are again best represented as a tube
which is a good model for the hairpins found in RNA but to represent an extended
chain of DNA, a succession of segments was used that allowed the double-helix
of the phosphate backbone to progress uninterupted from one to the next.
