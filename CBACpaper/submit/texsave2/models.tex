\section{\TT{models}}

\subsection{Overview of the object hierarchy}

The \TT{models} routine continues reading the main input file ("test.run")
after \TT{params} has completed.  It reads-in a hierarchy of groups of objects
(as outlined in the \Tt{main} section).   Besides the basic hierarchy sketched
previously, each group declaration can be preceded by a series of commands
(referred to below as a "Move set") that specify geometric operations that
will be applied to the group and all its sub-groups.  Correspondingly, at the
end of each group, another set of instructions (called the "Bond set") can
be included to specify bonds and links or modify those that have been created
automatically.    Although the end of each group is usually identified
automatically (after all its children have been read or by the appearance of
a command that implies the end), it is convenient for the moment to use the
\Tt{TER} command which explicitly marks the end of a group. 

Using \Tt{TRANS} to represent a Move-set and \Tt{REBOND} to represent a
Bond-set, then the model outlined in the opening section can be elaborated
as it might appear in the file "test.run".  (Indentation is not required but
can help clarity).
\begin{singlespace}
---------------------------------------------------------------------------------------------------
\begin{verbatim}
        PARAM some.model
    END
        TRANS set-1
        GROUP 0 1
            GROUP 0 2
                TRANS set-2
                GROUP 0 3
                    ATOM...
                    ATOM...
                    ATOM...
                TER
                GROUP 0 4
                    ATOM...
                    ATOM...
                    ATOM...
                    ATOM...
                REBOND set-A
                TER
           TER
       REBOND set-B
       TER
   END
\end{verbatim}
\ \ \ \ ---------------------------------------------------------------------------------------------------
\end{singlespace}
The transforms specified in the first Move-set (\Tt{TRANS set-1}) will apply rotations
and translations to everything that is created whereas the second set will apply
just to the first group of 3 'atoms'.   The first Bond-set (\Tt{REBOND set-A})
could specify new bonds within the second group of 4 atoms while set-B could 
make links between and within the two groups of atoms.

Representing
a Group (\Tt{GROUP}) declaration as  \Tt{G}, 
a Move-set (\Tt{TRANS}) declaration as  \Tt{M}, 
an atomic level (\Tt{ATOM}) declaration as  \Tt{A},
a Bond-set (\Tt{REBOND}) declaration as \Tt{B} and
a group termination (\Tt{TER}) as \Tt{T};
the model in the example above could be written:
\Tt{MGGMGAAATGAAAABTTBT}.   Using this representation, each group-structure has the 
form \Tt{MGABT}, or writing the optional commands in lower-case:  \Tt{mGAbt}.
If \Tt{*} now designates any number of atoms or any number of valid group-structures,
then the input stream expected by \TT{models} is: \Tt{mG*bt}. 
Expanding each '\Tt{*}' recursively within parentheses and bracketing groups on the same 
level, the above example can be seen to be a valid stream which will be parsed as:
\Tt{MG( G( [MG( AAA )T][G( AAAA )BT] )T )BT}.

In a structure that includes many repeated substructures, it would be tedious to 
specify the group structure for each occurrence.  To avoid this, a group structure can
be contained within a file and just the filename specified (using the \Tt{INPUT} command,
of which more below) instead of a full group specification.  As different instances of
each substructure may require a different Move-set, these can precede the \Tt{INPUT}
command but the end of an input file is treated as an End-of-Group signal, so group
definitions cannot span files.    \Tt{INPUT} files can, of course, contain \Tt{INPUT}
commands.

\subsection{Move-set commands}

These are described in the order in which they are applied to the completed group.
Characters within brackets indicate options and if separated by a bar '$\mid$', are optional
strings.  Strings within angle brackets represent numbers.

\subsubsection{\Tt{TRANS  <x> <y> <z>}}

A translation of the group by a displacement vector \{x,y,z\}. 

\subsubsection{\Tt{SPIN[XYZ] <theta>}}

A rotation of the group by an angle of \Tt{theta} degrees about the specified axis
(X,Y,Z).  For example: \Tt{SPINY 90} rotates the group by 90\dgr\ around the Y-axis. 

\subsubsection{\Tt{SPINS <tx> <ty> <tz>}}

Is a more general \Tt{SPIN} command, rotating about the X, then Y, then Z axes by the three
specified angles: \Tt{tx}, \Tt{ty}, \Tt{tz}.

\subsubsection{\Tt{TWIST <x><y><z> <p><q><r> <twist>}}

Rotate about the line defined from \{x,y,z\} to \{p,q,r\} by and angle \Tt{twist}.

\subsubsection{\Tt{HELIX <x><y><z> <turns>}}

Move the group by \Tt{x,y}, rotate by \Tt{<turns>} degrees about the Z axis then move \Tt{z}
along Z.    (N.B., the commands \Tt{TWIST} and \Tt{TRANS} can be combined to generate a helix
along any axis.)

\subsection{\Tt{GROUP <sort> <kids> [<x><y><z> <p><q><r>]}}

The full syntax of the \Tt{GROUP} command allows not only the the specification of the number
of expected children (\Tt{kids}) but also the \Tt{sort} of group that is wanted.   This is an
integer number that acts on a non-spherical object type (tube or ellipsoid) to modify its length.  
This is not something that can  be specified in the parameter file as each object will typically
have a different length depending on the number of atoms it contains.

The axis of these objects can also be specified as two additional end-point coordinates
(\{x,y,z\} and \{p,q,r\}).

\subsubsection{Tubes}

If the molecule type is protein and the group is at the secondary structure level (one up 
from atomic) then the \Tt{sort} value is used to specify secondary structure sort as:
0 = loop, 1 = an \AH\ and 2 = a \Bs.   Each have a particular tube length/thickness ratio
determined by the helical rise/residue along the secondary structure and the expected bulk
for a given loop length.  For each sort (0,1,2) the values used are: 0.5, 1.5, 3.0.
If the end-points are unspecified, then the axis is estimated from the terminal residues.
Even when the axis is specified, it provides only the direction and is set to pass through
the centre of the object.

If the molecule type is nucleic acid, the length is set in a similar manner using the
known rise/basepair.

Tubes that are not associated with a known secondary structure can have their length specified
by the value of \Tt{sort}, either as a multiple of their thickness (up to 10 times) or if
the value is negative, by the distance between the end-points (which must then be provided). 

\subsubsection{Ellipsoids}

As there are no secondary structures automatically associated with the ellipsoid shape, the
length of the ellipsoid axis is set only by the value of \Tt{sort} but whereas the value for a
tube was a simple ratio (1..10), the value for an ellipsoid runs from 1..21 with values over
10 creating oblate shapes and under 10, prolate shape (10=spherical).  The scaling is not 
linear but follows the progression: $n/10$ up to 10 and $10/n$ over 10 where $n = abs(10-sort)$.

Any negative value of \Tt{sort} again causes the length of the specified axis to be used. 

\subsubsection{Nucleic acids}

For both RNA and DNA, the \Tt{GROUP} command can adopt a two aliases, \Tt{DOUBL} and \Tt{SINGL},
to deal with differences between double and single stranded segments.  The latter is in fact
identical to  \Tt{GROUP} and was included just for neatness but the \Tt{DOUBL} command uses
the value of  \Tt{sort} to label a strand (Watson) that requires a complementary strand (Crick)
to occur later with the negated value of \Tt{sort}.

The default construction for nucleic acids treats a base-pair as a secondary structure so when
a \Tt{DOUBL} command is encountered, rather than allocating the specified number of \Tt{kids}
as individual atoms, the equivalent number of two-atom groups (basepairs) are created and the
following \Tt{ATOM} records are used to fill the first child in each basepair.   When the 
complementary strand is encountered, the the second child is then filled (running backwards).
Clearly, complementary strands must be equal in length and the first atom in Watson will be
basepaired with the last atom in Crick. 


\subsection{Bond-set commands}

\subsubsection{{\tt REBOND} and {\tt REJOIN} commands}

The commands that create and modify a chain take the form: {\tt RE[BOND|JOIN|TERM] mode <from> <to>}.
For example {\tt REBOND pdbid 22 66} sets a bond between the atoms with residue numbers 22 and 66.
But what if the residue numbers in the PDB file (which became \Tt{ATOM} records) are inconsistent or
missing or, as is likely, residues 22 and 66 are already bonded?

So as not to rely on PDB specified residue numbers \NAME\ maintains three internal atom counts as
well as a separate count of all objects (called their Unique IDentifier, or UID).  The {\tt allatom}
count starts with the first and ends with the last atom.  The {\tt teratom} count is set to zero with the
start of each group and between these is the {\tt endatom} count that is re-zeroed automatically by
any Bond-set command or explicitly by an {\tt INPUT} command with a zero appended ({\tt INPUT0}).

These numbering schemes are selected by the {\tt mode} string which can be set as: {\tt group}, {\tt local},
{\tt atomid}, {\tt unique} and {\tt pdbid}, referring respectively to the  {\tt teratom},  {\tt endatom}
and  {\tt allatom} counts, the UID and the PDB number.   The first two can obviously only be used
withing a level of nesting (scope) where their values have not been re-zeroed but the {\tt atomid} and
{\tt unique} counts are universal.   The uniqueness of the PDB numbering depends both on the
original source and how often it is reread using the {\tt INPUT} command, so must be used cautiously.

Only the {\tt unique} values can be used to rewire objects higher than the atomic level but this
must be done carefully as internal consistency checks refer only to the atomic level chain (see below)
so any higher level rewiring may well get wiped-out.

The {\tt REJOIN} command has an identical syntax and operation but creates a virtual bond that is
not refined or rendered. (In other words it makes a gap without terminating the chain).

Both commands overwrite whatever existing bond may have been there so after a series of reconnections,
there may even be no free termini.  To avoid this, the {\tt RETERM} command can be used to specify new
start and final positions of the chain using the {\tt from} and {\tt to} fields as start and end.
Again, care must be taken as the inclusion of a {\tt RETERM}
command triggers the execution of the internal consistency check described below.  Otherwise it
is assumed that the rewiring operations are harmless. 

\subsubsection{Rewiring consistency}

Even when applied properly, the above commands have the potential to alter the chain connectivity
of the polymer, leaving the higher levels out-of-step with the order at the atomic level.
To restore consistency, once all the groups have been completed, a subroutine traces the new
chain path at the atomic level and renumbers the higher levels in a sequential order.
This process is only executed if there is one or more {\tt RETERM} commands with each new
terminus taken as a start point.  So if the chain topology has been altered but the termini
remain the same, a {\tt RETERM} command re-specifying the original termini must be included.

At the domain level (atomic-2) and above, the atomic level chain can make multiple
entries and exits from the same object leading to branched topologies.
To accommodate branched topologies, the original bond allocation must be equal to the maximum
valance encountered on each level.   The recalculated numbering of the higher level objects then 
follows the order in which they are encountered by a path (a double linked-list) that traces a 
path following the outside of the topology tree.  (See the actin example below for clarification). 

\subsubsection{{\tt BONDS} and {\tt LINKS} commands}

The {\tt BOND} and {\tt LINK} commands operate irrespective of whether there is any polymer chain
and each follow the same syntax to read a list of simple bond/link assignments, as:
{\tt [LINKS|BONDS] mode filename}, where {\tt mode} is the numbering scheme to be used (as described
above) and the {\tt filename} contains a list of lines of four numbers specifying:
{\tt <donate> <accept> <type> <link>}.   The value of {\tt type} sets the bond thickness on
rendering or how far a link can extend without breaking and the {\tt link} value is used to determine
which ends to join between elongated objects.  (More below).

Unlike the {\tt RE[BOND|JOIN]} commands, {\tt BONDS} and {\tt LINKS} will not
overwrite an existing bond/link but will fill the first free slot in the bond/link list of the
donor object.   In the context of a chain, they behave as cross-links and if the {\tt BONDS} are
specified after any {\tt RETERM} command, they will not alter the chain topology.

\subsubsection{{\tt SHEET} and {\tt BETA} commands}

This pair of commands are a specialised version of the {\tt LINKS} command, described above,
and are used to create cross-links between strands in a \BS.   As would be expected they can
be used only when the molecule type is a protein.

The command {\tt SHEET [mode]}, alerts \TT{models} to expect a series of links, but rather than
open a new filename, these are just read from the current stream with each link specified by the
command {\tt BETA <donate> <accept> <strength>}.   Generally, these commands are kept within
the group that contains the sheet although links between groups can be defined (I think).

The string {\tt mode} specifies the numbering scheme as described above and if omitted the
default numbering is the {\tt group} atom number.
%
%#include "util.hpp"
%#include "geom.hpp"
%#include "cell.hpp"
%#include "data.hpp"
%
%//#define MAXIN 100000 //(needed for 3chy/unit3+.run)
%#define MAXIN 100000
%#define DEEP  10
%
%typedef struct {
%	int	grows, helix, twist, shift, spins, trans;
%	Vec	grow,  move,  spin,  tran;
%	float	turns, theta;
%	Seg	axis;
%	int	set;
%} Moves;
%
%void setCell ( Cell*, int, int, int, int, int );
%void setAtom ( Cell*, char* );
%void getEnds ( Cell*, char* );
%void endCell ( Cell*, Moves* );
%int  paired ( Cell*, int, int );
%int  moveSet ( char*, Moves* );
%void newPath ( char*, char );
%int  setModel ( char* );
%FILE* getInput ( char*, Moves* );
%void pass1set ( FILE* );
%void pass2set ( Cell*, int );
%void pass3set ( Cell*, int );
%void zipDNA ( Cell* );
%void setSheet ( FILE* );
%void setAngle ( Cell*, int );
%void readlinks( char* );
%int  makeLink ( Cell*, Cell*, int, int );
%int  makeBond ( Cell*, Cell*, int, int );
%void rebond ();
%void rewire ( Cell* );
%
%Cell    **atom1cell, **atom2cell, **secs2cell;
%int      *atom2atom,  *atom2resn,  *resn2atom;
%
%Vec	cent;
%int	secs, atoms, teratoms, endatoms, allatoms;
%
%typedef struct {
%        Cell   *at, *to, *scope; // cell source, target, scope of relink
%        int     ends;     //  0 = relink, -1 = new start, 1 = new finish
%} Relinks;
%Relinks *relink;
%int     nrelinks;
%
%void models ( FILE *run )
%{
%int	i,j,k,n, wait;
%Cell	*world = Cell::world;
%	world->model = world->level = world->uid = world->id = 0;
%	secs = teratoms = endatoms = allatoms = 0;
%	model = nrelinks = 0;
%        // allocate pairs of cell to collect chain relinks in pass1set
%        relink = new Relinks[MAXIN];
%	Cell::uid2cell = new Cell*[MAXIN*8]; TEST(Cell::uid2cell) // global back-ref.lookup (total)
%	atom2cell = new Cell*[MAXIN*6];	// lookup array for all atoms (allatoms)
%	secs2cell = new Cell*[MAXIN*1];	// WAS 10000);	// lookup for all SSEs
%	atom1cell = new Cell*[MAXIN*1];	// WAS 1000);	// lookup for domain (teratoms)
%	atom2resn = new int[MAXIN*6];	// lookup from global atom id to pdb resid (allatoms)
%	atom2atom = new int[MAXIN*4];	// lookup from local to global atom id (endatoms)
%	resn2atom = new int[MAXIN*1];	// lookup from global pdb resid to atom id
%	Cell::uid2cell[0] = world;
%	Cell::total = 0;
%	pass1set(run);
%	pass2set(world,0);
%	pass3set(world,0);
%	DO(i,Cell::total) Cell::uid2cell[i]->done = 0;
%	if (nrelinks) rebond(); // remake connections at the atomic level and propagate upwards
%	DO(i,Cell::total) Cell::uid2cell[i]->done = 0;
%	// set-up the scene data
%	world->solid = -1;
%	world->live = 999;
%/*
%	for (i=0; i<SPARE; i++) { // contents allocated in copyCell()
%		n = world->kids + i;
%		world->child[n]->kids = -1; // flag that no childern exist
%		world->child[n]->nbonds = -1; // or bonds to them
%		world->child[n]->nlinks = -1; // or links to them
%	}
%	world->link = uid2cell; // world is never linked so using link to pass uid2cell address
%*/
%	Pi(Cell::total) NL
%	DO(i,world->kids) setAngle(world->child[i],0);
%	printf("Data structures all setup\n");
%}
%
%void rezero ( Moves *holds ) {
%	holds->set = 0;
%	holds->grows=holds->helix=holds->twist=holds->shift=holds->spins=holds->trans=0;
%	holds->grow.zero(); holds->move.zero();holds->spin.zero();holds->tran.zero();
%	holds->turns = holds->theta = 0.0;
%}
%
%void pass1set ( FILE *runfile ) {
%// read the command <file> to allocate memory, set values and coordinates
%/* command syntax
%	M = MOVEset = {TRANS, SHIFT, SPIN, HELIX,...} applied to the scope of the GROUP they precede
%	G = GROUP   = a group of GROUPs or ATOMs
%	A = ATOM    = atomic level coordinate (PDB format)
%	T = TER     = end the GROUP (but ends automatically when full (all kids are read)
%	B = BONDset = {BOND, LINK, REBOND, RELINK}  (see below for scope)
%  with, lower-case = optional and X* = repeat X, a block has the structure:
%	MGATB or mGAtb or mGA*tb or mG*A*t*b or (mG*A*t*b)*
%  each time a GROUP is opened, the value of <level> increases and decreases on close
%eg: 
%-------------------------------------------
%level.id (0.0 = world)
%1.0	GROUP 0 3
%2.0	GROUP 0 6
%3.0	ATOM      1  CA  GLY A   1     -23.868 -17.022  -6.339  1.33  1.00
%:	:
%3.5	ATOM      6  CA  GLY A   6     -17.825 -16.538  -2.703  1.33  1.00
%2.1	GROUP 2 5   -14.79 -14.10 -3.15   -4.36 -8.33 -0.83
%3.0	ATOM      7  CA  GLY A   7     -14.841 -14.717  -4.363  1.33  2.00
%:	:
%3.4	ATOM     11  CA  GLY A  11      -4.370  -7.485  -0.937  1.33  2.00
%2.2	GROUP 0 6
%3.0	ATOM     12  CA  GLY A  12      -1.017  -8.244   0.718  1.33  1.00
%:	:
%3.5	ATOM     17  CA  GLY A  17      -3.385  -6.721   4.645  1.33  1.00
%1.1	GROUP 0 1
%2.3	GROUP 0 6
%3.0	ATOM      1  CA  GLY A   1     -23.868 -17.022  -6.339  1.33  1.00
%:	:
%3.5	ATOM      6  CA  GLY A   6     -17.825 -16.538  -2.703  1.33  1.00
%-------------------------------------------
%DO(i,10) group[i] = group[0];
%Pi(level) Pt(at) DO(i,4) printf("%3d", at[i]); Pt(uid) DO(i,4) printf("%3d", group[i]->uid); NL
%*/
%char	full[222], *line, have, last = 'G'; // the start state
%int	level = 0, at[DEEP], setMove, deep, sheet, model = 0;
%Cell	*cell, *group[DEEP]; group[0] = Cell::world;
%FILE	*files[DEEP], *file;
%Moves	moves[DEEP], holds[1];
%int	place, baseN, baseC, basepair = 0, print = 0;
%	deep = 0;
%	files[0] = file = runfile; // current file
%	DO(i,DEEP) { at[i] = moves[i].set = 0; }
%	holds->set = 0;
%	depth = Data::depth;
%	LOOP // keep reading until all nested GROUPs are complete
%	{ int	in = read_line(file,full);
%		line = full;
%		while (*line == '\t' || *line == ' ') line++;
%		have = line[0];
%		if (in<0) {	// EoF
%			Pt(EoF\n)
%			teratoms = 0;
%			if (deep) { // return to upper file stream
%				Pt(Moved back to previous file\n)
%				fclose(file);
%				file = files[--deep];
%				last = 'T';
%				continue;
%			} else {	// end of runfile (add END)
%				strcpy(line,"END"); in = 3;
%			}
%		}
%		if (in<3) {	// junk
%			continue;
%		}
%		if (have=='#') {	// echo comments
%			Ps(line) NL
%			continue;
%		}
%		if (strstr(line,"PRINT")) {
%			Pt(Print PDB on exit\n)
%			print = 1;
%			continue;
%		}
%		if (strstr(line,"STOP")) {
%			Ps(line) NL
%			if (print) putpdb(Cell::world);
%			exit(1);
%		}
%		if (line[0]=='R' && line[1]=='E') {	// get a RE[BOND|LINK|TERM] command
%			newPath(line,toupper(line[2]));	// flag =   B    L    T 
%			continue;
%		}
%		if (strstr(line,"TER")) {	// TERminate a group
%			teratoms = 0;
%			if (last=='T') {
%				Pt(teratoms zeroed by TER command\n)
%			}
%			continue;
%		}
%		if (have=='Z') { // zip-up the last (two) DNA chains
%			zipDNA(cell->parent);
%			continue;
%		}
%		//
%		cell = group[level];	// set current cell for this level
%		//
%		if (have=='M') {	// switch MODEL type (should occur only between groups)
%			model = setModel(line);
%			continue;
%		}
%		if (moveSet(line,holds)) { // get transforms for the next group (store in holds/moves)
%			Pt(Moves) Pi(holds->set) NL
%			continue;
%		}
%		if (have=='I') {     // INPUT[0|*] <filename> [<x> <y> <z>] (NB just one space before file)
%			// the sub file must contain a complete group set as EoF = end-of-GROUP. (min = GA)
%			files[++deep] = file = getInput(line,holds);
%			Pi(deep) NL
%			continue;
%		}
%		if (strstr(line,"SHEET")) {
%			Pt(Beta sheet\n)
%			sheet = 1;
%			setSheet(file);		// read (to EOF or END) and set BETA pairs
%			last = 'S';
%			continue;
%		}
%		if (strstr(line,"BOND") || strstr(line,"LINK")) {
%			Ps(line) NL
%			readlinks(line);
%			continue;
%		}
%		if (last=='T') { // close groups up to lowest incomplete level
%			LOOP {
%				level--;
%				if (level==0) break;
%				cell = group[level];
%				Pt(EoG) Pi(level) Pi(cell->uid) Pi(moves[level].set) NL
%				endCell(cell,moves+level); // finalises cell settings
%				if (strstr(line,"DOUBL") && level==depth-1 ) {
%					if (line[6] == '-') break; // NB just one space
%				}
%				at[level]++;
%				if (at[level] < cell->parent->kids) break; // more to fill
%			}
%			if (level==0) {
%				Pt(all groups complete\n)
%			} else { // group[level] = the current cell at each level
%				group[level] = cell->parent->child[at[level]];
%				cell = group[level];
%				Pt(Next) Pi(cell->uid) Pc(last) Pc(have) NL
%			}
%		}
%		if (strstr(line,"END") || level < 0) {	// END of everything
%			Pt(End\n)
%			endCell(group[0],moves);
%			Pt(End of World\n)
%			if (print) {
%				putpdb(Cell::world);
%				exit(1);
%			}
%			return;
%		}
%		if (strstr(line,"DOUBL"))   // switch to nucleic acid base-pairing mode
%		{ int	hit, id, n;
%			sscanf(line+6,"%d %d", &id, &n);
%			hit = paired(cell->parent,id,n);    // scan parent for complementary strand
%			if (hit < 0) { // no match (so make new segment)
%				basepair = n;
%				have = 'G'; 	// follow on as if a new GROUP
%			} else { // jump back to the end of the matching complementary strand
%				basepair = -n;
%				group[level] = cell = cell->parent->child[hit];
%				level = cell->level;
%				place = at[level];	// remember the current position
%				at[level] = hit;	// move to end of matching strand
%				last = 'A';
%				continue;
%			}
%		}
%		if (strstr(line,"SINGL")) have = 'G';
%		if (have=='G')	// GROUP <sort> <members> [<x> <y> <z>  <x> <y> <z> ]
%		{ int	n, sort;
%			total = Cell::total; // ++ in setCell but Cell::total ++ in spawn()
%			sscanf(line+6,"%d %d", &sort, &n);
%			if (basepair) // make more slots in parent for basepair SSEs
%			{ Cell	*pa = cell->parent; int m; // the parent needs n-1 more slots
%				Pt(set BPAIR) Pi(sort) Pi(n) Pi(level) Pi(at[level]) Pi(cell->uid) NL
%				m = pa->kids + n-1;	// new family size
%				baseN = at[level];	// first strand (current) position in family
%				baseC = baseN+n-1;	// final strand position in extended family
%				pa->extend(m,sort,0);	// enlarge family and fill from current up
%				for (int i=baseN; i<=baseC; i++) { Cell *kidi = pa->child[i];
%					setCell(kidi,model,level,sort,i,0);
%					kidi->spawn(2);
%					DO(j,2) setCell(kidi->child[j],model,level+1,sort,j,0);
%				}
%				group[level] = pa->child[baseN];
%				pa->child[baseC]->nbro = sort;	// use last SSE to hold id in nbro
%				pa->kids = m;		// set new family size
%			} else {
%				Pt(set GROUP) Pi(sort) Pi(n) Pi(level) Pi(at[level]) Pi(cell->uid) NL
%				setCell(cell,model,level,sort,at[level],n); // fill the current cell
%				if (holds->set) moves[level] = *holds; // copy trans on hold to moves
%				rezero(holds);
%				cell->spawn(n); // make kids and set: id, level, parent and ++Cell::total
%				level++;
%				if (level==depth) { // set-up cells for all the ATOMs in the GROUP
%					DO(i,n) setCell(cell->child[i],model,level,sort,i,0);
%				}
%				group[level] = cell->child[0];
%				at[level] = 0;
%			}
%			getEnds(cell,line);	// if more numbers, treat as axis end-points
%			sheet = 0;
%			last = 'G';
%			continue;
%		}
%		if (have=='A') { int done = 0; // ATOM (PDB format)
%	 		Pt(set ATOM) Pi(level) Pi(at[level]) Pi(cell->uid) NL
%			if (basepair) { int s;  // W/C strand = 0/1
%				if (basepair > 0) { // on the up strand
%					s = 0;
%				} else {	  // on the down strand
%					s = 1;
%				}
%				setAtom(cell->child[s],line); // read the atom data
%				if (basepair > 0) { // on the up strand
%					if (at[level] == baseC) { // end of the strand
%						group[depth-1] = cell;
%						done = 1;
%					} else {
%						at[level]++;
%					}
%				} else {	  // on the down strand
%					basepair++;	// increase (-ve) basepair until zero
%					if (basepair==0) {	// done (back at first basepair SSE)
%						at[depth-1] = place; // reset to current SSE position+1
%						group[depth-1] = cell->parent->child[place];
%						done = 1;
%					} else { 
%						at[level]--;
%					}
%				}
%			} else { // not in basepair mode
%				if (at[level]==cell->parent->kids-1) done = 1;
%				setAtom(cell,line); // read the atom data
%				at[level]++;
%			}
%			if (done) {	// end of family
%				level = depth;	// set level to atomic (-1 in basepair mode)
%				basepair = 0;
%				last = 'T';
%			} else {	// move to next
%				group[level] = cell->parent->child[at[level]];
%				last = 'A';
%			}
%			continue;
%		}
%	}
%}
%
%int paired ( Cell *cell, int id, int n ) {
%// check for an existing complementary strand in the current group
%	DO(i,cell->kids) {
%		if (cell->child[i]->nbro == -id) { // found
%			Pt(Complementary strand found) NL
%			return cell->child[i]->id;
%		}
%	}
%	return -1;
%}
%
%int setModel ( char* line ) {
%// switch MODEL type (should occur only between groups)
%int	model;
%	sscanf(line+6,"%d", &model);
%	printf("\nSwitching to model %d\n", model);
%	moltype = Data::model[model].moltype; // 0=PROT, 1=Nucleic, 2=CHEM, 3=CELL
%	subtype = Data::model[model].subtype; // 0=RNA, 1=DNA
%	if (moltype==0) printf("New molecule type = PROTein\n");
%	if (moltype==1) {
%		printf("New molecule type = Nucleic ");
%		if (subtype==1) printf("DNA\n"); else printf("RNA\n");
%	}
%	if (moltype==2) printf("New molecule type = CHEMical\n");
%	if (moltype==3) printf("New molecule type = CELLs\n");
%	return model;
%}
%
%int moveSet ( char *line, Moves *moves ) {
%Seg	axis;
%Vec	grow, move, tran;
%float	x,y,z, p,q,r, trans, turns, theta, spinx, spiny, spinz;
%float	s = Data::scalein;
%	if (line[0]=='S' && line[1]=='C') {	// SCALE <x> <y> <z> (applied after all offspring read in)
%		Ps(line) NL
%		sscanf(line+6,"%f %f %f", &x, &y, &z);
%		grow.x = x; grow.y = y; grow.z = z;
%		moves->grows = 1; moves->grow = grow;
%		moves->set = 1;
%		return 1;
%	}
%/* need to fix
%	if (line[0]=='H' && line[1]=='E') {	// HELIX <x> <y> <z> <turns> (move x,y, spin helix on Z, move z)
%		Ps(line) NL
%		sscanf(line+6,"%f %f %f %f", &x, &y, &z, &turns);
%		heli.x += x; heli.y += y; heli.z += z;
%		turns *= PI/180.0;
%		moves->helix = 1; moves->heli = heli; moves->turns = turns;
%		moves->set = 2;
%		return 1;
%	}
%*/
%	if (line[0]=='T' && line[1]=='W') {	// TWIST <x><y><z> <p><q><r> <theta> (rotate cell around axis)
%		Ps(line) NL
%		sscanf(line+6,"%f%f%f %f%f%f %f", &x,&y,&z, &p,&q,&r, &turns);
%		axis.A.x = x*s; axis.A.y = y*s; axis.A.z = z*s;
%		axis.B.x = p*s; axis.B.y = q*s; axis.B.z = r*s;
%		turns *= PI/180.0;
%		moves->twist = 1; moves->axis = axis; moves->turns = turns;
%		moves->set = 3;
%		return 1;
%	}
%	if (line[0]=='S' && line[4]=='X') {	// SPINX <theta> (applied after all offspring read in)
%		Ps(line) NL
%		sscanf(line+6,"%f", &spinx);
%		spinx *= PI/180.0;
%		moves->spins = 1; moves->spin.x = spinx;
%		moves->set = 4;
%		return 1;
%	}
%	if (line[0]=='S' && line[4]=='Y') {	// SPINY <theta> (applied after all offspring read in)
%		Ps(line) NL
%		sscanf(line+6,"%f", &spiny);
%		spiny *= PI/180.0;
%		moves->spins = 2; moves->spin.y = spiny;
%		moves->set = 4;
%		return 1;
%	}
%	if (line[0]=='S' && line[4]=='Z') {	// SPINZ <theta> (applied after all offspring read in)
%		Ps(line) NL
%		sscanf(line+6,"%f", &spinz);
%		spinz *= PI/180.0;
%		moves->spins = 3; moves->spin.z = spinz;
%		moves->set = 4;
%		return 1;
%	}
%	if (line[0]=='S' && line[4]=='S') {	// SPINS <theta> (applied after all offspring read in)
%		Ps(line) NL
%		sscanf(line+6,"%f %f %f", &spinx, &spiny, &spinz);
%		spinx *= PI/180.0; spiny *= PI/180.0; spinz *= PI/180.0;
%		moves->spins = 4; moves->spin = Vec(spinx,spiny,spinz);
%		moves->set = 4;
%		return 1;
%	}
%	if (line[0]=='S' && line[2]=='I') {	// SHIFT <x> <y> <z> (must be before TWIST in input)
%		Ps(line) NL
%		sscanf(line+6,"%f %f %f", &x, &y, &z);
%		move.x += x*s; move.y += y*s; move.z += z*s;
%		moves->shift = 1; moves->move = move;
%		moves->set = 5;
%		return 1;
%	}
%	if (line[0]=='T' && line[1]=='R') {	// TRANS <x> <y> <z> (applied after offspring read in)
%		Ps(line) NL
%		sscanf(line+6,"%f %f %f", &x, &y, &z);
%		tran.x = x*s; tran.y = y*s; tran.z = z*s;
%		moves->trans = 1; moves->tran = tran;
%		moves->set = 6;
%		return 1;
%	}
%	return 0; // none found
%}
%
%FILE* getInput ( char *line, Moves *moves ) {
%// sets up for new call of getGroup() from: INPUT[0|*] <filename> (NB just one space)
%char	nextfile[111];
%FILE	*file;
%char	*more;
%float	x, y, z;
%int	trans = 0, fat = 6;
%	Ps(line) NL
%	x = y = z = 1.0;
%	more = strchr(line+7,' '); // starts at any text after the filename
%	if (more) {	// read the position (saves having separate TRANS line)
%		sscanf(more+1,"%f %f %f", &x, &y, &z);
%		*more = (char)0;	// stop x,y,z being part of filename
%		trans = 1;
%	}
%	if (line[5]=='0') { // "0" resets local atom counts
%		teratoms = endatoms = 0;
%		Pt(Local ATOM counter set to zero\n)
%	}
%	if (line[5]=='*') { // "*" adds a random scatter (default = isotropic)
%		x = x*(randf()-0.5);
%		y = y*(randf()-0.5);
%		z = z*(randf()-0.5);
%		trans = 1;
%	}
%	if (trans) {
%		moves->tran.x = x; moves->tran.y = y; moves->tran.z = z;
%		moves->set = moves->trans = 1;
%	}
%	if (line[5] != ' ') fat++;
%	strcpy(nextfile,line+fat);
%	printf("Reading file %s\n", nextfile);
%	file = fopen(nextfile,"r");
%	return file;
%}
%
%void setCell ( Cell *cell, int model, int level, int sort, int id, int n ) {
%// fill the <cell> with data (except don't know number of children until GROUP command)
%int	nbonds =  Data::model[model].chain[level],
%	nlinks =  Data::model[model].links[level],
%	type   =  Data::model[model].shape[level];
%float	size   =  Data::model[model].sizes[level];
%	total++;
%	if (total > MAXIN*8) { Pt(Increase MAXIN) Pi(total) NL exit(1); }
%	if (n==1 && Data::model[model].chain[level+1]) { // an only child is not a chain
%		printf("*NB* chain of one at level %d\n", level+1);
%	}
%	if (level == depth-1) { // SSE level
%		secs2cell[secs] = cell;
%		secs++;
%	}
%	cell->ends = 0;
%	cell->endN.zero(); cell->endC.zero();
%	cell->endN.z = cell->endC.z = 1234.5;	// marker for unset ends
%	cell->cent.x = cell->cent.y = cell->cent.z = -1.0; // -ve dist = unset
%	cell->kids = cell->cots = n;
%	cell->model = model;
%	cell->solid = 0;
%	cell->far = 9.9;
%	cell->len = size;	// default length is the diameter
%	DO(i,NDATA) { cell->idata[i] = 0; cell->fdata[i] = 0.0; cell->vdata[i].zero(); }
%	cell->live = 9;
%	cell->atom = cell->btom = cell->ctom = 0;
%	cell->done = cell->lost = cell->bump = cell->busy = cell->empty = 0;
%	cell->hit = 0;
%	cell->type = type;	// pick-up the type assigned to the current level
%	cell->sort = sort;
%	if (cell->type==3 && sort==0) sort = E/2;  // default = 1:1 (sphere)
%	// set the bonds
%	cell->nbonds = nbonds;
%	if (nbonds < 0) nbonds = -nbonds;	// -ve flags circular topology
%	// sis and bro are shorthand for bond[0].to and bond[1].to (and exist even if no bonds)
%	cell->sis = cell->bro = cell->parent;
%	if (nbonds > 0) { int i;
%		if (nbonds > 9) { Pt(*NB*) Pi(nbonds) NL exit(1); }
%		cell->bond = new Bonds[nbonds+1]; TEST(cell->bond) // +1 to cover sis=0 + bro=1
%		for (i=0; i<=nbonds; i++) {	// NB: <=
%			cell->bond[i].to = 0;	// unset
%			cell->bond[i].type = 0; // default is thickness 0 (line) 
%			cell->bond[i].link = 0; // default connection is N--C (-1 = don't link)
%			cell->bond[i].next = i; // default is go to same
%		}
%		cell->bond[0].to = cell->bond[1].to = cell->parent; 
%	} else { cell->bond = 0; }
%	// set the links
%	cell->nlinks = nlinks;
%	if (nlinks > 0) { int i;
%		cell->link =  new Bonds[nlinks]; TEST(cell->link)
%		for (i=0; i<nlinks; i++) {
%			cell->link[i].to = 0;	// unset
%			cell->link[i].type = 0; // default is thickness 0 (line) 
%			cell->link[i].next = 5; // default is break when over 5% over extended
%		}
%	} else { cell->link = 0; }
%	DO(i,3) cell->ranks[i] = id;
%	cell->junior = cell->senior = 0;
%	cell->starts = cell->finish = 0;
%	cell->nsis = cell->nbro = 0;
%}
%
%void setAtom ( Cell *cell, char *line ) {
%// read the coordinates and set atom counters
%int	resn; float x,y,z, a,b, s = Data::scalein;
%	sscanf(line+30,"%f %f %f %f %f", &x, &y, &z, &a, &b);
%	if (a < NOISE) { // assume unset
%		x = drand48()-0.5; y = drand48()-0.5; z = drand48()-0.5;
%		if (b > NOISE) { x*=b; y*=b; z*=b; }
%	}
%	cell->xyz.x = x*s; cell->xyz.y = y*s; cell->xyz.z = z*s;
%	teratoms++; // rezeroed by EoF or TER 
%	endatoms++; // rezeroed by RELINK/REBOND
%	allatoms++; // never rezeroed
%	if (allatoms > MAXIN*6) { Pt(Increase MAXIN) Pi(allatoms) NL exit(1); }
%	if (endatoms > MAXIN*4) { Pt(Increase MAXIN) Pi(endatoms) Pt(or use INPUT0) NL exit(1); }
%	if (teratoms > MAXIN*1) { Pt(Increase MAXIN) Pi(teratoms) NL exit(1); }
%	cell->atom = teratoms;		// local number for beta
%	cell->btom = endatoms;		// group number for relinks()
%	cell->ctom = allatoms;		// number for all
%	atom1cell[teratoms] = cell;	// local to current file
%	atom2cell[allatoms] = cell;	// global over all atoms
%	atom2atom[endatoms] = allatoms; // relink scope to global
%	sscanf(line+22,"%d", &resn);
%	cell->resn = resn;
%	if (resn<0) { Pt(*NB* negative residue numbers are not good: default to 0) NL resn = 0; }
%	resn2atom[resn] = allatoms;
%	atom2resn[allatoms] = resn;
%}
%
%void endCell ( Cell *cell, Moves *moves ) {
%int	n = cell->kids, sort = cell->sort, m = cell->model;
%Data	*param = Data::model+m;
%int	moltype = param->moltype,
%	subtype = param->subtype;
%	Pt(End of Cell) Pi(cell->level) Pi(cell->id) Pi(cell->uid) Pi(cell->kids) Pi(cell->ends) NL
%	// set cell position to average of children
%	cent.zero();
%	DO(i,n) cent += cell->child[i]->xyz;
%	cell->xyz = cent/(float)n;
%	if (cell->ends) { // shift axis poles (vdata set in getEnds())
%		cell->endN = cell->xyz - cell->vdata[0];
%		cell->endC = cell->xyz + cell->vdata[0];
%	} else {
%		if (moltype==0 && cell->level == depth-1) { // Protein SSE with no ends
%			if (n==2) { // use termini
%				cell->endN = cell->child[0]->xyz; 
%				cell->endC = cell->child[1]->xyz; 
%			}
%			if (n>2) { // use average of first/last pair (good for beta and loops)
%				cell->endN = cell->child[0]->xyz & cell->child[1]->xyz; 
%				cell->endC = cell->child[n-1]->xyz & cell->child[n-2]->xyz; 
%			}
%			if (n>3 && cell->sort==1) { // alpha
%				cell->endN = cell->endN & cell->child[3]->xyz; 
%				cell->endC = cell->endC & cell->child[n-3]->xyz; 
%			}
%			cell->ends = 1;
%		}
%	}
%	// apply transformations to current cell and contents
%	if (moves->set) { float x,y,z;
%/* need to fix
%		if (moves->helix) { // HELIX rotate about <heli> and move by <heli>
%			moves->helix = 0;
%			cell->spin(moves->heli,moves->turns);
%			cell->move(moves->heli);
%			Pt(Helical shift along) Pv(moves->heli) Pr(moves->turns) NL
%		}
%*/
%		if (moves->twist) { // TWIST rotate about <axis> by <twist>
%			moves->twist = 0;
%			cell->spin(moves->axis,moves->turns);
%			Pt(Twist about) Pl(moves->axis) Pr(moves->turns) NL
%		}
%		if (moves->spins) { // SPINS about XYZ axes
%			moves->spins = 0;
%			x = moves->spin.x;
%			if (x*x > NOISE) cell->spin(Vec(1,0,0),x);
%			y = moves->spin.y;
%			if (y*y > NOISE) cell->spin(Vec(0,1,0),y);
%			z = moves->spin.z;
%			if (z*z > NOISE) cell->spin(Vec(0,0,1),z);
%			Pt(Rotated about XYZ by) Pv(moves->spin) NL
%			
%		}
%		if (moves->trans) { // TRANSlate
%			moves->trans = 0;
%			Pv(moves->tran) NL
%			cell->move(moves->tran);
%			Pt(Moved to) Pv(cell->xyz) NL
%		}
%		moves->set = 0;
%	}
%}
%
%void getEnds ( Cell *cell, char *line ) {
%// read axis (if given)
%float	x1,y1,z1, x2,y2,z2; Vec v;
%	if (strlen(line) < 20) return;
%Ps(line) NL
%	sscanf(line+12,"%f %f %f  %f %f %f", &x1,&y1,&z1, &x2,&y2,&z2);
%	v = Vec(x2-x1, y2-y1, z2-z1);
%	v *= 0.5*Data::scalein;
%	cell->vdata[0] = v; // temp store then N-C split across centre in endCell()
%	cell->ends = 1;
%	if (cell->sort < 0) {// -ve = flag to use the given axis length
%		cell->sort = -cell->sort; cell->ends = -1;
%	}
%	Pt(Axis ends read) Pv(v) Pi(cell->ends) NL
%}
%
%void newPath ( char *line, char mode ) {
%// get the rewiring moves for a new bond path (mode: B=bond, L=link, T=term)
%int	at, to; Cell *cat, *cto;
%	Ps(line) NL
%	sscanf(line+13,"%d %d", &at, &to);
%	if (toupper(line[7])=='P') { // use pdbid
%		cat = atom2cell[resn2atom[at]];
%		cto = atom2cell[resn2atom[to]];
%	}
%	if (toupper(line[7])=='L') { // local numbering (seq. from: start, INPUT0 or REBOND)
%		cat = atom2cell[atom2atom[at]]; // atom2atom[] gives the global (allatom) id 
%		cto = atom2cell[atom2atom[to]];
%	}
%	if (toupper(line[7])=='A') { // atom level numbering
%		cat = atom2cell[at];
%		cto = atom2cell[to];
%	}
%	if (toupper(line[7])=='G') { // global (uid) numbering
%		cat = Cell::uid2cell[at];
%		cto = Cell::uid2cell[to];
%	}
%	// in T mode, cat = new N-terminus, cto = new C-terminus
%	relink[nrelinks].at = cat;
%	relink[nrelinks].to = cto;
%	//Pt(New path) Pc(mode) Pi(cat->uid) Pi(cto->uid) Pi(nrelinks) NL
%	if (mode=='T') relink[nrelinks].ends = 0;
%	if (mode=='B') relink[nrelinks].ends = 1;
%	if (mode=='L') relink[nrelinks].ends = 2;
%	nrelinks++;
%	teratoms = endatoms = 0;
%}
%
%void pass2set ( Cell *cell, int level )
%// second pass sets chains, links and relationships
%{
%int	kidbonds;
%int	i,j,k, m,n, id, nlinks;
%	id = cell->id;
%	n = cell->kids;
%	m = cell->model;
%	moltype = Data::model[m].moltype;
%	subtype = Data::model[m].subtype;
%	chain = Data::model[m].chain;
%	links = Data::model[m].links;
%	nlinks = links[level+1];          // for lower level
%	if (nlinks && moltype==0 && cell->type==2) { // SSE (link->next is used as %x10 break point)
%		if (nlinks < 2) { Pi(nlinks) Pt(SSE children need at least 2 links\n)  exit(1); }
%		for (i=0; i<n; i++) if (cell->child[i]->nlinks < 2) cell->child[i]->nlinks = 2; // may be set
%	      	if (cell->sort==1) { // alpha helix
%			for (i=0; i<n; i++) { // foreach atom (i) in the helix set i+3, i+4 links
%				if (i+3<n)  { Bonds *link = cell->child[i]->link+0;
%					link->to = cell->child[i+3];
%					link->type = 10;
%					link->next = 1;
%				}
%				if (i-4>=0) { Bonds *link = cell->child[i]->link+1;
%					link->to = cell->child[i-4];
%					link->type = 11;
%					link->next = 1;
%				}
%			}
%		}
%	      	if (cell->sort==2) { // beta sheet
%			for (i=0; i<n; i++) { // foreach atom (i) in the strand set i+2, i-2 links
%				if (i+2<n)  { Bonds *link = cell->child[i]->link+0;
%					link->to = cell->child[i+2];
%					link->type = 12;
%					link->next = 1;
%				}
%				if (i-2>=0) { Bonds *link = cell->child[i]->link+1;
%					link->to = cell->child[i-2];
%					link->type = 13;
%					link->next = 1;
%				}
%			}
%		}
%	}
%	if (n==0) return;
%	// set the last and next children for each child (may be changed by RELINK later)
%	if (cell->child[0]->bond==0) kidbonds = 0; else kidbonds = 1;
%	m = n-1;		// m = index of last child[0...m]
%	if (chain[level+1] && n > 1) { // in a chain (NB assumes no mixed models)
%		for (i=0; i<n; i++)
%		{ Cell	*kidi = cell->child[i];
%		  int	modi = kidi->model;
%			if (i<m) { int modj = cell->child[i+1]->model;
%				if (modi==modj) kidi->bro = cell->child[i+1];
%			}
%			if (i>0) { int modh = cell->child[i-1]->model;
%				if (modh==modi) kidi->sis = cell->child[i-1];
%			}
%			if (!kidbonds) continue;
%			kidi->bond[0].to = kidi->sis;
%			kidi->bond[1].to = kidi->bro;
%			kidi->nbro = 1; // only count brothers (always just one sister)
%		}
%		if (chain[level+1] < 0) { // chain is circular so link ends
%			cell->child[m]->bro = cell->child[0];
%			cell->child[0]->sis = cell->child[m];
%			if (kidbonds) {
%				cell->child[m]->bond[1].to = cell->child[0];
%				cell->child[m]->bond[1].next = 1;
%				cell->child[0]->bond[0].to = cell->child[m];
%				cell->child[0]->bond[0].next = 0;
%			}
%		}
%		if (chain[level+1] > 0) { // chain is linear so use parent as dummy siblings
%			cell->child[m]->bro = cell;
%			cell->child[0]->sis = cell;
%			if (kidbonds) {
%				cell->child[m-1]->bond[1].next = 0; // cyclise the bond path (only at
%				cell->child[ 1 ]->bond[0].next = 1; // the ends of an unbranched chain)
%			}
%		}
%	}
%	if (chain[level+1] && n==1) { // in a 'chain' of one (just dummy siblings)
%		cell->child[0]->sis = cell;
%		cell->child[m]->bro = cell;
%	}
%	// set the terminal children (for current cell)
%	cell->starts = cell->junior = cell->child[0];
%	cell->finish = cell->senior = cell->child[n-1];
%	//
%	if (moltype==1 && level==depth-2 && chain[level+1] && chain[level+2]) // in a NA seg.
%	{ int	seg[4]; // for termini of double-stranded (ds) segments
%		DO(i,n) // loop over SSE segments to set internal ss and ds bonds (sort: ss=0, ds=1)
%		{ Cell	*kidi = cell->child[i]; // kidi = a base-paired rung or loop (SSE level)
%			if (kidi->sort==0) { // single-stranded loop segment
%				DO1(j,kidi->kids-1) { // set internal loop bonds
%					kidi->child[j]->sis = kidi->child[j-1];
%				}
%				kidi->junior = kidi->starts = kidi->child[0];
%				kidi->senior = kidi->finish = kidi->child[kidi->kids-1];
%				kidi->endN = kidi->junior->xyz; kidi->endC = kidi->senior->xyz;
%				kidi->ends = 1;
%			} else		// double-stranded base-paired segment
%			{ Cell	*W = kidi->child[0], *C = kidi->child[1]; // Watson/Crick strands
%				if (i && cell->child[i-1]->sort!=0) {
%					W->sis = W->bond[0].to = cell->child[i-1]->child[0];
%					C->bro = C->bond[1].to = cell->child[i-1]->child[1];
%				}
%				if (i<n-1 && cell->child[i+1]->sort!=0) {
%					C->sis = C->bond[0].to = cell->child[i+1]->child[1];
%					W->bro = W->bond[1].to = cell->child[i+1]->child[0];
%				}
%				kidi->endN = W->xyz; kidi->endC = C->xyz;
%				kidi->junior = kidi->starts = W;
%				kidi->senior = kidi->finish = C;
%				kidi->ends = 1;
%			}
%		}
%		DO(i,4) seg[i] = -1;
%		DO1(i,n-1) // loop over the SSE termini and set bonds between ds and ss segments
%		{ Cell *kidi = cell->child[i-1], *kidj = cell->child[i];
%		  int	si = abs(kidi->sort), sj = abs(kidj->sort);
%			if (si>0 && sj>0) { // pair..pair (in a ds seg)
%				if (seg[0]<0) { // ds seg start
%					seg[0] = kidi->child[0]->ctom;
%					seg[1] = kidi->child[1]->ctom;
%				}
%				seg[2] = kidj->child[0]->ctom;
%				seg[3] = kidj->child[1]->ctom;
%				continue;
%			}
%			if (seg[0]<0) continue; // no ds segs yet
%			DO(k,n) // check all seg ends against current ds ends (0=Wn, 1=Cc, 2=Wc, 3=Cn)
%			{ Cell	*kid = cell->child[k],
%				*jun = kid->junior, *sen = kid->senior, *sej;
%			  int	nk = jun->ctom, ck = sen->ctom;
%				if (kid->sort != 0) continue;
%				DO(j,4) if (nk==seg[j]+1) {
%					sej = atom2cell[seg[j]];
%					Pt(N) Pi(k) Pi(j) Pi(seg[j]) Pi(nk) Pi(jun->resn) Pi(sej->resn) NL
%					jun->sis = jun->bond[0].to = sej;
%					sej->bro = sej->bond[1].to = jun;
%				}
%				DO(j,4) if (nk==seg[j]-1) {
%					sej = atom2cell[seg[j]];
%					Pt(N) Pi(k) Pi(j) Pi(seg[j]) Pi(nk) Pi(sej->resn) Pi(jun->resn) NL
%					sej->sis = sej->bond[0].to = jun;
%					jun->bro = jun->bond[1].to = sej;
%				}
%				DO(j,4) if (ck==seg[j]+1) {
%					sej = atom2cell[seg[j]];
%					Pt(C) Pi(k) Pi(j) Pi(seg[j]) Pi(ck) Pi(sen->resn) Pi(sej->resn) NL
%					sen->sis = sen->bond[0].to = sej;
%					sej->bro = sej->bond[1].to = sen;
%				}
%				DO(j,4) if (ck==seg[j]-1) {
%					sej = atom2cell[seg[j]];
%					Pt(C) Pi(k) Pi(j) Pi(seg[j]) Pi(ck) Pi(sej->resn) Pi(sen->resn) NL
%					sej->sis = sej->bond[0].to = sen;
%					sen->bro = sen->bond[1].to = sej;
%				}
%			}
%			DO(j,4) seg[j] = -1;
%		}
%	} else { // continue down
%		DO(i,n) pass2set(cell->child[i], level+1);
%	}
%}
%
%void pass3set ( Cell *cell, int level )
%// third pass adds links across chain of chains (can only be done after pass2set() )
%// sums atoms into higher level centres and sets ends for ellipsoids and tubes
%{
%float	bondCA = Data::bondCA;
%int	i,j,k, dna,
%	m = cell->model,
%	n = cell->kids,
%	id = cell->id,
%	type = abs(cell->type),
%	sort = cell->sort;
%Data	*param = Data::model+m;
%	moltype = param->moltype;
%	subtype = param->subtype;
%	chain = param->chain;
%	sizes = param->sizes;
%	split = param->split;
%	dna = moltype;//*subtype; // 1 = true
%	if (n==0) {
%		cell->xyz -= cent;
%		return;
%	}
%	if (n==1) { // one child cannot form a chain (yet)
%		if (chain[level+1] != 0) {
%			printf("*NB* chain of one at level %d\n", level+1);
%			// chain[level+1] = 0;
%		}
%	}
%	if (chain[level] && chain[level+1]>0 && split[level]==0 && dna==0) { // join cousins across parents
%		// cell can be linear or circular but child must be linear and not split (ie forced break)
%		if (cell->level == cell->sis->level) {	// set sister of first child as last child of sister
%			cell->starts->sis = cell->sis->finish;
%			if (cell->starts->bond) cell->starts->bond[0].to = cell->sis->finish;
%		}
%		if (cell->level == cell->bro->level) {	// set brother of last child as first child of brother
%			cell->finish->bro = cell->bro->starts;
%			if (cell->finish->bond) cell->finish->bond[1].to = cell->bro->starts;
%		}
%	}
%	//
%	for (i=0; i<n; i++) pass3set(cell->child[i], level+1);
%	//
%	if (level < Data::depth)
%	{ float	size = sizes[level],	// for tube and ellipsoids, size = section diameter
%		fn = (float)cell->kids,
%		axis = 1.0;
%	  Vec	x;
%		cell->setWcent();	// sum (weighted) children into upper levels
%		if (cell->ends) {	// ends have been read in
%			axis = cell->endC | cell->endN;	// set length for ellipsoid (unless sort>0 or ends<0)
%			x = cell->endC - cell->endN;		// set axis direction
%		} else {
%			if (cell->kids > 1) {	// take the direction between termini
%				x = cell->senior->xyz - cell->junior->xyz;
%			} else {
%				x.set_rand(); 	// pick a random axis direction
%			}
%		}
%		x.setVec();		// set to unit length
%		if (type==1) {		// for sphere, lenght = diameter
%			axis = cell->len;
%		}
%		if (type==2)		// scale SSE tube lengths
%		{ float	sf = fn-1.0;
%			if (cell->ends > 0) { // use ideal axis length (otherwise keep the given length)
%				if (moltype==0) { // protein
%					sf *= 0.6; // ?
%					if (sort==0) axis = loopAXIS*size*sqrt(fn); // loops
%					if (sort==1) axis = alphAXIS*sf*bondCA/3.8; // alpha
%					if (sort==2) axis = betaAXIS*sf*bondCA/3.8; // beta
%				}
%				if (moltype==1) {
%					if (subtype==0 && level==depth-2) { // RNA tube
%						if (sort==0) axis = 0.2*sqrt(fn);
%						// RNA = 2.3 rise/bp --> 0.115 (1/2 for duplex, 1/10 for scale)
%						if (sort==1) axis = 0.12*fn;
%					}
%					if (level==depth-1) { // DNA/RNA (rungs of basepaired ladder)
%						axis *= 0.7; // keep ends inside Ps
%					} 
%				}
%			}
%		}
%		if (type==3)	// assign ellipsoid sort
%		{ float last = 0.0, ratio = axis/size;
%		  int	best = -1;
%			if (cell->sort==0) {// find preset ellipsoid with best fit to axis/size ratio
%				for (i=1; i<E; i++) { float y, fi = (float)i;
%					y = Data::Eratio[i];
%					if (ratio>last && ratio<y) {
%						Pi(i) Pr(last) Pr(y) Pr(ratio) NL
%						if (ratio-last < y-ratio) best = i-1; else best = i;
%					}
%					last = y;
%				}
%				if (best < 0) { // no fit found
%					if (size > axis) {
%						best = 1;
%						printf("Most oblate ellipsoid used:");
%					} else {
%						best = E-1; // for E sorts
%						printf("Most prolate ellipsoid used:");
%					}
%				} 	// set axis length to selected ellipsoid
%				axis = size*Data::Eratio[best];
%				Pi(best) Pr(axis) Pr(size) NL
%				cell->sort = best;
%			} else { // sort given. If ends<0 keep the given axis length otherwise use the ideal length
%				if (cell->ends >= 0) axis = size*Data::Eratio[cell->sort];
%			}
%		}
%		// make symmetric about centre
%		x *= axis*0.5;
%		cell->endN = cell->xyz - x;
%		cell->endC = cell->xyz + x;
%		if (moltype==1 && subtype==0 && type==2 && sort==1) // for RNA SSE put endC near termini
%		{ float dnn, dcn, dnc, dcc;
%			dnn = cell->endN|cell->junior->xyz;
%			dnc = cell->endC|cell->junior->xyz;
%			dcn = cell->endN|cell->senior->xyz;
%			dcc = cell->endC|cell->senior->xyz;
%			if (dnn+dcn < dnc+dcc) { // swap ends
%				cell->endN = cell->xyz + x;
%				cell->endC = cell->xyz - x;
%			}
%		}
%		DO(i,n) { Cell *kidi = cell->child[i]; // set internal reference distances
%			kidi->cent.x = kidi->xyz|cell->endN;
%			kidi->cent.y = kidi->xyz|cell->xyz;
%			kidi->cent.z = kidi->xyz|cell->endC;
%		}	
%		cell->len = axis;
%	}
%}
%
%void zipDNA( Cell *cell ) {
%int	n = cell->kids;
%	DO(i,n/2)
%	{ Cell	*w = cell->child[i],
%		*c = cell->child[n-i-1];
%	  float d = w->xyz|c->xyz;
%		w->link[0].to = c;
%		c->link[0].to = w;
%	}
%}
%
%void setSheet ( FILE *beta )
%{ // set the beta sheet links (link[].next = c has weight c/10)
%char	line[222];
%	while (1) { int i, io, a,b,c; Cell *from, *to;
%		io = read_line(beta,line);
%		if (io < 0 ) break;
%		if (io < 10) continue;
%		if (line[0] == '#') continue;
%		sscanf(line+5,"%d %d %d", &a, &b, &c);
%		if (c>5) { Ps(line) NL }
%		from = atom1cell[a]; to = atom1cell[b];	// set links at CA level (uses domain seq. ids)
%		//if (from->parent->sort<2 || to->parent->sort<2) continue; // SHEET atoms not in beta
%		for (i=2; from->link && i<from->nlinks; i++) { // 0,1 used for +/-2 in strand
%			if (from->link[i].to == to) break; // already set
%			if (from->link[i].to == 0) {
%				from->link[i].to = to;
%				from->link[i].type = 15;
%				from->link[i].next = c;
%				from->nlinks = i+1;
%				break;
%			}
%		}
%		from = from->parent; to = to->parent;	// set links at strand (midpoint) level
%		if (from->sort<2 || to->sort<2) continue; // some SHEET2 atoms may be in loops
%		for (i=0; from->link && i<from->nlinks; i++) {
%			if (from->link[i].to == to ) break;  // already set
%			if (from->link[i].to == 0) {
%				from->link[i].to = to;
%				from->link[i].type = 9;
%				from->link[i].next = 1;
%				from->nlinks = i+1;
%				break;
%			}
%		}
%	}
%}
%
%void setAngle ( Cell *cell, int update )
%{
%// alpha theta, tau = 90.5, 50.4
%// alpha theta, tau =  1.6, -0.9
%// betas theta, tau =  2.2,  2.7
%//
%// alpha d 0-2,3 = 5.4, 5.1, 6.2
%// betas d 0-2,3 = 6.7, 9.9,13.5
%//
%float	theta, tau1, tau2;
%float	dis12, dis1, dis2;
%int	i, n, id, level = cell->level;
%float	f, g, fix = 0.1;
%Vec	alphA, alphD, betaA, betaD;
%float	s = Data::bondCA/3.8; // scale internal/real
%	alphA.x = alphA.z = -0.9; alphA.y =  1.6;
%	betaA.x = betaA.z =  2.7; betaA.y =  2.2;
%	alphD.x = alphD.z = 5.1*s; alphD.y =  6.2*s;
%	betaD.x = betaD.z = 9.9*s; betaD.y = 13.5*s;
%	id = cell->id;
%	n = cell->kids;
%	theta = tau1 = tau2 = 999.9;
%	dis12 = dis1 = dis2 = 999.9;
%	if (cell->sis && cell->bro && cell->parent->kids>3) { // connected at both ends in a chain
%		theta = angle(cell->sis->xyz, cell->xyz, cell->bro->xyz);
%		if (theta > NOISE) {
%			if (cell->sis->sis) {
%				if (angle(cell->sis->sis->xyz, cell->sis->xyz, cell->xyz) > NOISE) {
%					tau1 = torsion(cell->sis->sis->xyz,cell->sis->xyz,cell->xyz,cell->bro->xyz);
%				}
%				dis1 = vdif(cell->sis->sis->xyz,cell->bro->xyz); // i-2..i+1
%			}
%			if (cell->bro->bro) {
%				if (angle(cell->xyz, cell->bro->xyz, cell->bro->bro->xyz) > NOISE) {
%					tau2 = torsion(cell->bro->bro->xyz,cell->bro->xyz,cell->xyz,cell->sis->xyz);
%				}
%				dis2 = vdif(cell->bro->bro->xyz,cell->sis->xyz); // i+2..i-1
%			}
%		}
%		if (dis1<999.0 && dis2<999.0 && cell->parent->kids>4) {
%			dis12 = vdif(cell->bro->bro->xyz,cell->sis->sis->xyz); // i+2..i-2
%		}
%		if (update > 0) {
%			f = fix; g = 1.0-f;
%			cell->geom.x = g*cell->geom.x + f*tau1;
%			cell->geom.y = g*cell->geom.y + f*theta;
%			cell->geom.z = g*cell->geom.z + f*tau2;
%			//f = fix*fix; g = 1.0-f;
%			cell->dist.x = g*cell->dist.x + f*dis1;
%			cell->dist.y = g*cell->dist.y + f*dis12;
%			cell->dist.z = g*cell->dist.z + f*dis2;
%		} else {
%			cell->geom.x = tau1; cell->geom.y = theta; cell->geom.z = tau2;
%			cell->dist.x = dis1; cell->dist.y = dis12; cell->dist.z = dis2;
%		}
%		if (cell->sis->type==2 && cell->type==2 && cell->bro->type==2) { // in SSE (not end)
%			if (cell->sis->sort==1 && cell->sort==1 && cell->bro->sort==1) { // in alpha
%				vave(cell->geom, alphA, &(cell->geom));
%				vave(cell->dist, alphD, &(cell->dist));
%			}
%			if (cell->sis->sort==2 && cell->sort==2 && cell->bro->sort==2) { // in beta
%				vave(cell->geom, betaA, &(cell->geom));
%				vave(cell->dist, betaD, &(cell->dist));
%			}
%		}
%	}
%	if (level && cell->bond && update<1) // local bond lengths not updated
%	{ int	m = cell->model;
%	  Data	*param = Data::model+m;
%	  float bond = param->sizes[level]+param->bonds[level];
%	  Cell	*sis = cell->sis, *bro = cell->bro;
%	  int	freesis, freebro; // -ve dist = free bond (set by RELINK in rebond())
%	  	freesis = freebro = 0;
%	  	if (cell->prox.x < -NOISE) freesis = 1;
%	  	if (cell->prox.z < -NOISE) freebro = 1;
%		cell->prox.x = cell->prox.y = cell->prox.z = 999.9;
%		if (sis && cell->level == sis->level)       cell->prox.x = vdif(cell->xyz,cell->sis->xyz);
%		if (sis && bro && sis->level == bro->level) cell->prox.y = vdif(cell->sis->xyz,cell->bro->xyz);
%		if (bro && cell->level == bro->level)       cell->prox.z = vdif(cell->xyz,cell->bro->xyz);
%		if (level == Data::depth) { // force ideal bond length for atom level
%			cell->prox.x = bond;
%			if (cell->prox.y>999.0) cell->prox.y = bond*1.85; // for 135 deg. angle
%			cell->prox.z = bond;
%		}
%		if (freesis) { // mark sister bond as free
%			cell->prox.x *= -1.0;
%		}
%		if (freebro) { // mark brother bond as free
%			cell->prox.z *= -1.0;
%		}
%	}
%//Pi(level) Pi(cell->uid) Pi(cell->type) Pi(cell->sort) Pv(cell->geom) Pv(cell->dist) Pv(cell->prox) NL
%	if (n == 0) return;
%	for (i=0; i<n; i++) setAngle(cell->child[i], update);
%}
%
%void readlinks ( char *line ) 
%{
%Cell	*ci, *cj;
%FILE	*lin;
%char	mode[5];
%int	i,j, in, nlinks, resids = 0, at = 12;
%//	read line: [LINKS|BONDS] [local|global] <filename> (local = internal numbering, global = PDB resid numbering)
%	if (line[0]=='L') strcpy(mode,"Link"); else strcpy(mode,"Bond");
%	resids = 0;
%	if (toupper(line[6])=='P') {	// pdbid
%		resids = 1;
%		printf("%sing by PDB resid numbers\n",mode);
%	}
%	if (toupper(line[6])=='L') {	// local
%		resids = 2;
%		printf("%sing by local sequential numbering\n",mode);
%	}
%	if (toupper(line[6])=='G') {	// global (uid)
%		resids = 3; at++;
%		printf("%sing by global sequential numbering\n",mode);
%	}
%	if (resids==0) { Pt(No valid numbering scheme specified\n)  return; }
%	printf("reading: %s\n", line+at);
%	lin = fopen(line+at,"r");
%	while (1) { int type, link, err = 0;
%		in = read_line(lin,line);
%		if (line[0]=='#') { Ps(line+1) NL continue; }
%		if (in < 3) { fclose(lin); return; }
%		sscanf(line,"%d %d %d %d", &i, &j, &type, &link); // type = thickness, link = end-end
%		if (resids==1) { // PDB id (apply to all sequential pairs with those id.s)
%			ci = cj = 0;
%			DO1(n,Cell::total) { Cell *cn = Cell::uid2cell[n];
%				if (cn->resn == i) ci = cn;
%				if (ci && cn->resn==j) cj = cn;
%				if (ci && cj) {
%					if (mode[0]=='L') err = makeLink(ci,cj,type,link);
%					if (mode[0]=='B') err = makeBond(ci,cj,type,link);
%					if (err) { Ps(line) Pi(resids) NL exit(1); }
%					ci = cj = 0;
%				}
%			}
%		} else {
%			if (resids==2) { // local
%				i = atom2atom[i]; j = atom2atom[j];
%				ci = atom2cell[i];
%				cj = atom2cell[j];
%			}
%			if (resids==3) { // global
%				ci = Cell::uid2cell[i];
%				cj = Cell::uid2cell[j];
%			}
%			if (mode[0]=='L') err = makeLink(ci,cj,type,link);
%			if (mode[0]=='B') err = makeBond(ci,cj,type,link);
%			if (err) { Ps(line) Pi(resids) NL exit(1); }
%		}
%	}
%}
%
%int makeLink ( Cell *ci, Cell *cj, int type, int link ) {
%int	nlinks, m = ci->model;
%	links = Data::model[m].links;
%	nlinks = links[ci->level];
%	if (nlinks>0) { int made = 0;
%		for (int k=0; k<nlinks; k++) {
%			if (ci->link[k].to == 0) { // free
%				ci->link[k].to = cj;
%				ci->link[k].type = type;
%				ci->link[k].link = link;
%				made = 1;
%				break;
%			}
%		}
%		if (made) { int i = ci->uid, j = cj->uid;
%			printf("Cell %d linked to cell %d   PDBids = %d to %d   dist = %f\n",
%				i, j, atom2resn[i], atom2resn[j], vdif(ci->xyz,cj->xyz));
%		} else {
%			printf("NB: cell has no free links\n");
%			return 1;
%		}
%	} else {
%		printf("NB: tried to link a cell with no links\n");
%		return 1;
%	}
%	return 0;
%}
%
%int makeBond ( Cell *ci, Cell *cj, int type, int link ) {
%int	nbonds, bi,bj, m;
%	m = ci->model;
%	bonds = Data::model[m].bonds;
%	chain = Data::model[m].chain;
%	nbonds = abs(chain[ci->level]);
%	bi = -1;
%	if (nbonds>0) {
%		for (int k=0; k<=nbonds; k++) { // NB <=
%			if (ci->bond[k].to == 0) { // free
%				bi = k; break;
%			}
%		}
%		if (bi < 0) {
%			printf("NB: cell has no free bonds (uid=%d)\n",ci->uid); Pi(nbonds) NL
%			return 1;
%		}
%	} else {
%		printf("NB: tried to bond a cell with no bonds (uid=%d)\n",ci->uid);
%		return 1;
%	}
%	m = cj->model;
%	bonds = Data::model[m].bonds;
%	nbonds = abs(chain[cj->level]);
%	bj = -1;
%	if (nbonds>0) {
%		for (int k=0; k<=nbonds; k++) { // NB <=
%			if (cj->bond[k].to == 0) { // free
%				bj = k; break;
%			}
%		}
%		if (bj < 0) {
%			printf("NB: cell has no free bonds (uid=%d)\n",cj->uid); Pi(nbonds) NL
%			return 1;
%		}
%	} else {
%		printf("NB: tried to bond a cell with no bonds (uid=%d)\n",cj->uid);
%		return 1;
%	}
%	ci->bond[bi].to = cj;	cj->bond[bj].to = ci;
%	ci->bond[bi].next = bj;	cj->bond[bj].next = bi;
%	ci->bond[bi].type =     cj->bond[bj].type = type;
%	ci->bond[bi].link =     cj->bond[bj].link = link; // -ve flags a cross-link (not a chain-link)
%	bi = ci->uid, bj = cj->uid;
%	printf("Cell %d bonded to cell %d (link-type = %d)  PDBids = %d to %d   dist = %f\n",
%		bi, bj, link, atom2resn[bi], atom2resn[bj], vdif(ci->xyz,cj->xyz));
%	return 0;
%}
%
%void rebond () {
%	Pi(nrelinks) NL
%	DO(i,nrelinks) // rewire the atomic level
%	{ Cell *cell = relink[i].at, *link = relink[i].to, *scope = relink[i].scope;
%		if (relink[i].ends == 0) {	// terminus (cell = new N end, link = new C end)
%			link->bond[1].to = link->bro = link->parent; link->bond[1].next = -1;
%			cell->bond[0].to = cell->sis = cell->parent; cell->bond[0].next = -1;
%			rewire(cell);
%		}
%		if (relink[i].ends > 0) {	// reconnect
%			cell->bro = link;
%			if (cell->bond) cell->bond[1].to = link;
%			link->sis = cell;
%			if (link->bond) link->bond[0].to = cell;
%		}
%		if (relink[i].ends == 2) {	// flag for unrefined bond length
%			cell->prox.z = -999.9;	// -ve z = link to bro not refined
%			link->prox.x = -999.9;	// -ve x = link to sis not refined
%			cell->bond[1].link = link->bond[0].link = -999; // dito
%		}
%	}
%}
%
%void rewire ( Cell *start ) {
%int	i, j, k, n;
%int	path[MAXIN];
%float	d;
%int	m = start->model,
%	moltype = Data::model[m].moltype,
%	subtype = Data::model[m].subtype,
%	dna = moltype*subtype;
%	// propagate rewiring at the atomic level up through the higher levels 
%	if (Data::depth < 2) return; // nothing to check (the world cannot be in a chain)
%	for (i=Data::depth; i>1; i--)
%	{ Cell	*lo = start, *hi = start->parent, *newlo, *newhi, *oldlo, *oldhi;
%	  int	lolev = i, hilev = i-1,
%		locha = Data::model[lo->model].chain[lolev],
%		hicha = Data::model[hi->model].chain[hilev];
%		if (locha < 1) continue;	// no chain at this level
%		if (hicha < 1) break;	// no chain at this level
%		for (j=0; j<hi->parent->kids; j++) { Cell *c = hi->parent->child[j];
%			m = c->model;
%			if (Data::model[m].moltype != moltype) continue; // don't mix models
%			for (k=0; k<hicha; k++) { c->bond[k].to = 0; c->bond[k].next = -1; }
%		}
%		n = 0; path[0] = hi->uid;
%		LOOP {
%			if (lo->done) {
%				Pt(Loop in new bond path:) NL
%				Pi(lo->uid) Pi(lo->atom) Pi(lo->btom) Pi(lo->ctom) Pi(lo->level) NL
%				exit(1);
%			}
%			lo->done = 1;
%			if (path[n] != hi->uid) path[++n] = hi->uid;
%			oldlo = lo->sis; oldhi = hi->sis;	// short for previous chain positions
%			newlo = lo->bro; newhi = newlo->parent;	// next cell in low-level chain
%			if (newlo->level != lolev) {	// end of the low-level chain
%				if (hicha == 0) {
%					Pt(end of low chain in detached parent) NL
%					continue; // no chain at high level
%				} else { // end of lo chain so newlo is at hi level
%					Pt(end of low chain in chained parent) NL
%					if (moltype==1 && lolev==depth) {
%						// a nucleic acid chain ends at the start so don't mark as an end
%						newlo->bro = newlo->bond[1].to = newlo; // reset to old brother
%						n++;
%					} else {
%						newlo->bro = newlo->bond[1].to = newhi; // set end of parent chain
%					}
%					//newlo->bond[1].to = start->parent; // cyclic bond path
%					//newlo->bond[1].next = 1;
%					break;	  // must be end of level
%				}
%			}
%			if (newhi != hi) {	// change of parent
%				if (newhi == hi->bro) { // normal move into next high-level cell
%					//Pt(normal move in parental chain) NL
%				} else {	// the new parent is unexpected 
%					//Pt(unexpected jump in parental chain) NL
%					if (newhi == hi->sis) { // just been there
%						//Pt(back to last) NL
%					} else {
%						//Pt(move to new parent) NL
%					}
%					hi->bro = newhi;
%					newhi->sis = hi;
%				}
%			}
%			lo = newlo; hi = newhi;	// next cell and parent
%		}
%		// path[] has collected the sequence of hi-level cells visited, now set bonds
%		k = 1;	// 1=out (bro path), -1=back (sis path)
%		Cell::uid2cell[path[0]]->sis = Cell::uid2cell[path[0]]->parent;
%		for (j=0; j<n; j++) { int p = path[j], q = path[j+1], end = 0;
%			hi = Cell::uid2cell[p], newhi = Cell::uid2cell[q];
%			hi->nsis++;	// use nsis to count visits (not used elsewhere)
%			if (j>0 && newhi->nsis && q==path[j-1]) {
%				k = -1; end = 1;	// on the return path
%			}
%			if (k<0 && newhi->nsis==0) { Cell *last = Cell::uid2cell[path[j-1]];
%				k = 1;			// back to an outward path
%				last->bond[0].next = hi->nsis; // sis path switch
%			}
%			if (k>0) { // out (bro path)
%				hi->bro = hi->bond[hi->nsis].to = newhi;
%				hi->bond[hi->nsis].next = newhi->nsis+1;
%			} else {  // back (sis path)
%				hi->sis = hi->bond[0].to = newhi;
%				hi->bond[0].next = 0;
%				if (hi->type > 1) { // find closest ellipsoid/tube ends
%					hi->bond[0].link = sort4min( // 1=NN, 2=NC, 3=CN, 4=CC
%						hi->endN|newhi->endN, hi->endN|newhi->endC,
%						hi->endC|newhi->endN, hi->endC|newhi->endC
%					);
%				}
%			}
%			if (end) { // redirect bro pointer to exit (sis, 0)
%				newhi->bond[newhi->nsis].next = 0; // NB if end, last==next
%			}
%			hi->nsis++;	// use nsis to count visits (not used elsewhere)
%		}
%		if (dna && i==depth ) // use half double stranded nucleic chain
%		{ Cell	*last = Cell::uid2cell[path[n-1]],
%			*past = Cell::uid2cell[path[n-2]],
%			*half = Cell::uid2cell[path[n/2]];
%				half->bro = half->parent;
%				last->bro = past;
%		} else {
%			Cell::uid2cell[path[n-1]]->bro = Cell::uid2cell[path[n-1]]->parent;
%		}
%		start = start->parent;
%		/* print the path 
%		Pt(Rewired path at level) Pi(i) NL
%		{ Bonds *b = start->bond+1; int first = b->next;
%		  Cell	*last = start;
%			Pi(start->uid) Pi(first) NL
%			LOOP{ // if b==last->bond the link is on the return path (sis) used for drawing bonds
%				if (b==last->bond) { Pi(last->uid) Pt(-->) Pi(b->to->uid) Pi(last->bond[0].link) }
%				if (b!=last->bond) { Pi(b->to->uid) Pi(b->to->sis->uid) Pi(b->to->bro->uid) NL }
%				if (b->to->sis == b->to->bro) break; // dead-end
%				last = b->to;
%				b = b->to->bond+b->next;
%				if (b->to==start && b->next==first) {
%					Pi(last->uid) Pt(-->) Pi(b->to->uid) Pi(last->bond[0].link) NL
%					break;
%				}
%			}
%		}
%		*/
%	}
%}
%
%/*
%		while (line[0]=='H' && line[1]=='I')	// HINGE <level> <id1> <id2> <type>
%		{ int	n, a, b, lev, len, lin;		// <id[12]> = sequential number in <level>
%		  Cells *ca, *cb, *cc;			// <len> = length*10, <lin> = 1..4 NN,NC,CN,CC
%		  int	seta, setb, make = 5;
%			Ps(line) NL
%			sscanf(line+6,"%d %d %d %d %d", &lev, &a, &b, &len, &lin);
%			ca = cb = 0;
%			n = 0;
%			for (i=1; i<total; i++) {
%				cc = uid2cell[i];
%				if (cc->level != lev) continue;
%				if (n == a) ca = cc;
%				if (n == b) cb = cc;
%				n++;
%			}
%			if (ca==0) { Pi(a) Pt(not found for HINGE) NL exit(1); }
%			if (cb==0) { Pi(b) Pt(not found for HINGE) NL exit(1); }
%//Pi(ca->level) Pi(ca->id) Pi(ca->type) Pi(cb->level) Pi(cb->id) Pi(cb->type) NL
%			// store hinge as CHEM bond (next = link, type = length)
%			if (ca->bond==0) {
%				ca->bond = (Bonds*)malloc(sizeof(Bonds)*make); TEST(ca->bond)
%				for (i=0; i<make; i++) ca->bond[i].to = 0;
%			}
%			if (cb->bond==0) {
%				cb->bond = (Bonds*)malloc(sizeof(Bonds)*make); TEST(cb->bond)
%				for (i=0; i<make; i++) cb->bond[i].to = 0;
%			}
%			for (i=0; i<make; i++) { if (ca->bond[i].to==0) { seta = i; break; }}
%			for (i=0; i<make; i++) { if (cb->bond[i].to==0) { setb = i; break; }}
%			ca->bond[seta].to = cb; cb->bond[setb].to = ca; // two (a-b, b-a) links set
%			if (lin==1) { ca->bond[seta].next = 1; cb->bond[setb].next = 1; } // aN->bN, bN->aN
%			if (lin==2) { ca->bond[seta].next = 2; cb->bond[setb].next = 3; } // aN->bC, bC->aN
%			if (lin==3) { ca->bond[seta].next = 3; cb->bond[setb].next = 2; } // aC->bN, bN->aC
%			if (lin==4) { ca->bond[seta].next = 4; cb->bond[setb].next = 4; } // aC->bC, bC->aC
%			ca->bond[seta].type = cb->bond[setb].type = len;
%			in = read_line(pdb,line);
%			if (in<6) break;
%		}
%		if (line[0]=='L' && line[4]=='S') {	// LINKS <mode> <filename>
%			Ps(line) NL
%			readlinks(line,0);
%		}
%*/
