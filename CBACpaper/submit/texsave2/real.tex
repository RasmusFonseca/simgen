\subsection{Real data}

In this section two examples are provided using known macromolecules.
As the previous section used a globular protein structure, an example is
taken firstly of an integral membrane protein to illustrate a different 
combination of objects and secondly of an RNA structure to show how the
different objects can be combined to represent nucleic acid structures.
As the basic method implements only geometric constraints, both examples
were extended to incorporate external distance constraints derived from
the analysis of correlated mutations.  (See \cite{} for a review).

\subsubsection{Rhodopsin}

The first structure of an integral trans-membrane (TM) protein to be determined was
that of bacteriorhodopsin and this protein and its much larger sister family,
the opsins (including GPCR receptors), remains a favourite for testing
modelling and prediction methods.

These structures consist of 7-TM helices arranged in a simple bundle.  Each
was modelled as an \AH\ confined in an tube, as described above for the
small globular protein.   The seven tubes were then contained in a larger
tube which had a diameter narrow enough to confine the helices in a compact
packing arrangement in the plane of the membrane and long enought to allow 
the helices to shift relative to each other to a reasonable extent up and
down relative to the membrane.   Because the ends of the helical tubes are
not constrained to lie wihtin their containing tube, they are still free to 
twist relative to each other, as is commonly observerd in such structures.

As an exercise in structure prediction, the helices were allowed to move
under the influence of the pairwise residue constraints derived from the
correlated mutation analysis, starting from a number of configurations
obtained from combinatorial enumeration over a hexagonal lattice \cite{}.
The resulting models were then ranked on how well they had satisied the
given constraints.

\begin{figure}
\centering
\subfigure[Score vs RMSD]{
\label{Fig:secstr2DFS}
\rotatebox{270}{
\epsfxsize=160pt \epsfbox{figs/rhod/rms04.ps}
}}
\caption{
\label{Fig:myo2DFS}
{\bf Myosin-V structure.}
The structure of dimeric myosin-V ({\tt 2DFS}), determined by single-particle cryo-electron microscopy
is shown ($a$) with secondary structures represented in cartoon style (using {\small RASMOL}) with
\AH\ coloured pink and \Bs s yellow.  ($b$) The same structure is shown as a virtual \CA\
backbone with the two heavy chains coloured cyan and green and their associated light-chains
in alternating yellow/orange and red/magenta, respectively.
}
\end{figure}

The method was also applied to a protein of unknown structure, flhA, which is
a component in the type-III secretion system.

\subsubsection{SAM riboswitch}

The structure of the S-adenylate-methionine (SAM) riboswitch is a small RNA involved
in the control of something.   RNA secondary structure prediction methods produce
a "clover-leaf" structure reminiscent of tRNA, and like that molecule, its structure
can be viewed as two basepaired hairpins with each being an insertion into the other.
Unlike the secondary structure prediction, the tertiary structure reveals an additional
short region of base-pairing between the two hairpins (a pseudo-knot) which serves
to lock the 3D structure.   Interestingly, these interactions are clearly predicted
by the correlation analysis and, together with the more 'trivial' base-pairing
correlations, were used as constraints for modelling.

The predicted base-paired regions were set-up as tubes with the phosphates of the paired bases
at either end of a smaller tube forming rungs of a ladder (as described above) and these stem-loops
were then specified to be confined inside a larger sphere.    Random starting positions were taken
for each of the stem-loops and under the influence of the confining pull (to move inside the
sphere), their bonded phosphates and the imposed distance constraints, the stem-loop moved
inwards and packed to best accommodate the constraints.    As not all the constraints can be
simultaneously satisfied (due to possible prediction error), once inside the sphere,
the longest constraints were gradually culled (with a stochastic bias to retain the
strongest and those not within a stem-loop).   By the end of a short run, a compact
structure remained and, as with the TM-protein predictions, after many runs, the structures
were ranked on how well they satisfied the constraints.
