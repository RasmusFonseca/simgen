\section{\TT{params}}

\subsection{Model parameter specification}

The \TT{params} routine reads the file specified as the first argument on the \NAME\ command
line (usually "something.run"), or by default, "test.run" if no name is given.  The main
purpose of the \TT{params} routine is to read a file of parameters (specified by the
\Tt{PARAMS} command described below) and set-up the generic behaviours for each level in
the model hierarchy.

Each property is specified
on one line as a series of integers corresponding to successively deeper levels.  There
can be up to 10 levels and unused fields are left as trailing zeros which can be followed
by an optional comment.   Presently, there are also ten different properties specified
by the parameter file which will be described below.  (With their parameter name italicized
in the sub-section title).

Firstly however, the routine expects a
declaration of the molecule type, which can be any of: \Tt{PROT}, \Tt{RNA}, \Tt{DNA},
\Tt{CHEM} or \Tt{CELL}.  Any other text will probably behave in a protein-like way.
This can be followed by a secondary command that is used by the \TT{viewer} to specify
a colour scheme for rendering which will be described under the section on that routine.

Except for models specified as \Tt{CHEM}, 
it is assumed that the lowest (most detailed) level of the molecular representation will
be a backbone 'virtual' chain, without any atomic detail.  

\subsubsection{\Tt{Line 1}: {\em type} of object}

As described above, there can be three types of object, a: sphere, tube and ellipsoid that
are specified by the numbers 1,2,3, respectively.   There can also be a virtual object that
can have all the properties of a sphere but just is not rendered except as a tiny sphere.
Ellipsoids (with axis lengths, $A,B,C$) can only be oblate ($A<B=C$) or prolate ($A>B=C$).
These shapes are generally, referred to as spheroids but the term ellipsoid will be retained
below to maintain a distinction from the sphere-object (which persists even when $A=B=C$).

If a negative number is specified, then the object is rendered as a wire mesh but otherwise
behaves identically.

\subsubsection{\Tt{Line 2}: {\em size} of object}

The second line specifies the size of the object.  For spheres and tubes, this is the diameter
while for ellipsoids it is the diameter across the axis of symmetry ($B$ or $C$, in the above 
specification).   The values specified here are used at a tenth their size inside \NAME.

With a specified known type of molecule (protein of nucleic acid), standard values will be
adopted and maintained for bond lengths and internal secondary structure links (more below).
This means that higher levels (eg: protein domains) are free to move within their specified
constraints.  While this is usually the normal desired behaviour, it is sometimes required
that the local starting geometry of the configuration should be preserved.  This is implemented
by the \TT{fixer} and a negative prefix on the \Tt{size} value will signal that it should be
applied during the simulation to that level of structure. 

\subsubsection{\Tt{Line 3}: {\em bump} size of object}

Although it might be assumed that objects will bump when their surfaces make contact, however,
there are situations where it is useful to have a different bump size, which can be specified
on this line.

\subsubsection{\Tt{Line 4}: number of {\em links}}

Any object can form links to any other object (on the same level).  To avoid allocating unused
space to objects that never link, the maximum number of links can be specified on this line. 

\subsubsection{\Tt{Line 5}: number of {\em bonds}}

The number of bonds an object can make is specified in an identical manner but has some
implications for polymers (protein, nucleic acid) that are expected to form chains.

Specifying a non-zero number of bonds for a polymer automatically results in the creation
of bonds between adjacent monomers within their group. 
So in the \Tt{GROUP} structure declared in the previous section, if bonds
were specified at the atomic level, then there would be two chains of 3 and 4 monomers.
If bonds were allocated to the next level up (normally the secondary structure level), then
a chain of two secondary structures would be created.  However, for consistency, the
program will also create the bond joining their two atomic level chains (giving a single
chain of 7 monomers at the 'atomic' level).  If the atomic level has no bonds, then each
secondary structure in the chain will contain a group of free-floating atoms.

If the number of bonds specified is negative, this is taken as a flag to create a circular
chain.  So in the current example, a negative number on the atomic level will create two
circular chains of 3 and 4 monomers.   If the secondary structure level is also negative,
and the atomic level positive, there will be a 'ring' of 2 secondary structures containing 
a ring of 7 monomers and if both values are negative, the 'ring' of 2 will contain a
triangle and square 'rings' with no bond between them.

In some higher level polymer domain configurations and chemical structures, each object
may form multiple bonds and the value of the bond number is equivalent to the maximum
valance of the object.  

\subsubsection{\Tt{Line 6}: {\em move} step size}

The step size applied by the \TT{shaker} is specified on this line.   As with all motion,
it is applied from the top level downwards with every movement at the higher level also
propagated to all offspring below.

If the specified value of $size$ ($S$) is positive, the motion is a translation in a random direction
with a magnitude of $S$/100, followed by a rotation in a random direction with a angular
displacement of $S$/100 radians.

If the value is negative, then just the translation is applied.

\subsubsection{\Tt{Line 7}: {\em keep} within parent boundaries}

If a child strays outside its parent's boundary (as defined by its $size$), then a shift
(or kick) is applied to redirect it inwards with a magnitude specified on this line.
Note that the value appears in the position of the level of the parent but is applied
to the level below).

If the parent is a sphere, then 'inwards' is towards its centre  and for an ellipsoid,
'inwards' is towards the nearest focus (simpler than calculating surface normals).  For a tube,
the default action is to shift the child towards the surface whether inside or outside
the tube, in a direction perpendicular to the tube axis.

If the value of the parameter is negative, then these default actions are reversed and
children within spheres and ellipsoids will be directed towards the parental surface
while children will be free to wander inside the tube.

Some of these behaviours can be modified for individual objects by values specified in
the model description read in the \TT{models} routine described below.

\subsubsection{\Tt{Line 8}: {\em bond} length}

The bond lengths specified on this line are the default ideal values which every bond
associated with that level will be refined towards.  As not all objects are spheres, the
bond length is not a simple centre---centre distance.   For spheres, it is the distance
between the surfaces (so the centre---centre distance = $bond+size$).  For tubes, it is
the desired separation between the ends of two tubes (axis end---end distance) and for
ellipsoids it is the equivalent pole---pole distance between the ends of their axes of
symmetry.  (The ends of the $A$ axis, in the example above.).

A negative value of the bond length evokes a behaviour that prevents the default joining
of chains of children when their parents are in a chain.  This is 
useful when constructing a fiber formed from unbonded subunits (such as actin).

\subsubsection{\Tt{Line 9}: {\em hard} bump repulsion}

The way in which objects bump is simple only at the atomic level where a kick is applied
to both atoms in a direction away from their common centre with a magnitude of $M$/100,
where $M$ is the specified value.  This repulsion is applied when the distance between
the atom centres is less than the value of $bump$ and is independent of the degree of
penetration or any other constraint.

\subsubsection{\Tt{Line 10}: {\em soft} bump repulsion}

Objects at a higher level do not behave in this way but can 'happily' inter-penetrate,
providing none of their children are clashing.   The degree to which they are separated
is a function of the number of inter-family bumps on a non-linear scale between the value
of the $hard$ repulsion specified on the previous line and the $soft$ value specified
on this line.  Specifically, the kick size, $k$, is: $k = g*S + (1-g)*H$, where $S$ 
and $H$ are the $hard$ and $soft$ values and $g = \exp (-m^2)$, for $m$ child collisions.

As the value of $g$ ranges from 1 (no bumping children) to 0 (many bumping children) the
degree of parental repulsion will scale from $soft$ to $hard$.   This means that if $soft$
has a non-zero value, then there will be a 'soft' repulsion before any of the children in the
two families clash.  This creates a 'jelly' like behaviour to the collision of high-level
objects.

If the value of $soft$ is negative, then behaviour of the objects reverts to hard repulsion.

\subsection{An example input file}

\begin{singlespace}
------------------------------------------------------------------------------------------------------
\begin{footnotesize}
\begin{verbatim}
PROT sec
    0,    0,    3,    2,    1,    0,    0,    0,    0,    0    	// type
 9999, -120,  -50,   -8,    2,    0,    0,    0,    0,    0    	// size
    0,  120,   50,    8,    8,    0,    0,    0,    0,    0    	// bump
    0,    0,    0,    6,    4,    0,    0,    0,    0,    0    	// nlinks
    0,    0,    0,    1,    1,    0,    0,    0,    0,    0    	// nbonds
    0,    0,    1,    1,    1,    0,    0,    0,    0,    0    	// move
    0,    2,    5,   10,    0,    0,    0,    0,    0,    0    	// keep
    0,    0,    0,    0,    4,    0,    0,    0,    0,    0    	// bond
    0,   10,   10,   10,   10,    0,    0,    0,    0,    0    	// hard bump
    0,    1,    2,    5,    0,    0,    0,    0,    0,    0    	// soft bump
\end{verbatim}
\end{footnotesize}
------------------------------------------------------------------------------------------------------
\end{singlespace}

In this example, a protein model is set-up with 4 levels (not including the world).
From bottom up, the levels corresponds to: \CA-chain, secondary structure elements,
domains and individual chains (subunits). 

The first column describes the world simply as big (9999).  Other values can be assigned to
the world and, except that it will not move or be rendered, it is a normal object.
A useful behaviour is to give it a smaller size and assign a value to $keep$ to stop
objects wandering too far away. 

The atomic level (fifth column) specifies each atom is a small sphere ($size = 2$) and is in
a linear chain ($nbonds > 0$).  So with a relatively long bond between atoms ($bond = 4$),
the chain will will appear as a "ball-and-stick" model (with an atom---atom distance of 6). 
It has a bump size that is larger than this ($bump = 11$) but as bonded atoms do not bump (more
detail in the \Tt{bumper} section), this will first apply to neighbours-but-one along the chain,
keeping the virtual bond angle along the chain almost obtuse. (Since $6*6+6*6 \approx 8*8$).
The value of 8 was chosen as it is the closest approach made between monomers in an \AH\ and
as linked atoms do bump, this will avoid disruption. (In \AA ngstroms, this is 8*3.8/6 = 5.1
which is close to the i---i+2 distance in the \AH)). 
Each atom has the capacity to make 4 links, two of which will be automatically used in local
secondary structure formation).  The atoms will have a slight motion and a reasonably strong
($hard$) bump repulsion.

The secondary structure level is a chain ($nbonds = 1$) of tubes ($type = 2$) with equal
$size$ and $bump$ values that just enclose their atomic configurations. Note that the $size$
values at this level and above are negative, indicating that there is no generic bond length 
and local geometry will be preserved by the \Tt{fixer}.   If the $size$ value were positive,
the \TT{linker} would enforce the generic bond length, which is set at zero, and would link
the tubes without any spacer.  So not a disaster, but causing a marked shift away from
the original structure once the simulation is started.  Quite a few links are allocated (6)
and as none are automatically assigned at this level (?), there is scope to tie-down the 
structure further.  A value of $keep$ is specified to hold the atoms to the tube surface
(except in loop regions where this is not enforced).
Both the $hard$ and $soft$ bump settings are now used so the secondary
structures have a bit of a jelly-like constitution.

At the next higher (domain) level, the objects are ellipsoids (of softer jelly) and are still
in a chain.  Parents are also a bit more relaxed about keeping their children close.  These
trends continue to the next (subunit) level which is no longer chained and is rendered as a
virtual sphere ($type = 0$).
 

\subsection{Global value specification}

Although the main purpose of \TT{params} is to parameterise the model behaviour, it also
sets-up some general behaviours most of which are used in 'debugging' the model specification
script described in the next section.   These commands should generally be specified in the
order in which they are described below.

\subsubsection{\Tt{NORUN}}

Compile and run the model but do not execute the \TT{driver} routine.

\subsubsection{\Tt{NOMOVE}}

Compile and run the model but do not execute the \TT{shaker}, \TT{keeper} or \TT{bumper} routines.
This command is very useful for checking (and as the \TT{viewer} is active, seeing) if the model 
is OK before it has the chance to do anything (like explode).

\subsubsection{\Tt{NOVIEW}}

The \TT{viewer} is not executed.   Essential for use in batch mode, say on a computer cluster.

\subsubsection{\Tt{HIDDEN}}

Freezes execution of objects that are out of view, either outside the field-of-view or
beyond the back-plan or behind the point-of-view.   Distant objects also become solid and
their contents is also frozen.

\subsubsection{\Tt{SCALE <in> <out>}}

Sets two scale factors, the first is applied to all coordinates read in by \TT{models} and the
second to coordinates written out (usually from \TT{looker}).  Default values are used for
protein and nucleic acid for input but when writing very large structures in PDB format, a
small scale factor is often needed to get the values to fit the format.

The line: {\tt SCALE 0.158 0.633  \# scalein = .6/3.8, scaleout = 3.8/6}, scales a PDB input
to have a CA---CA bond length of 0.6 and an output scale of 0.38 (tenth size).

\subsubsection{\Tt{SHRINK <value>}}

This scale factor is applied on every simulation cycle to the highest level (1, not counting
the world). A factor of $1-<value>$ is applied to the positions of all level-1 objects causing
them to fall towards the centre of the world.   Its main use has been in testing collisions
as it saves time waiting for wandering objects to collide.

\subsubsection{\Tt{PARAM <filename>}}

This is the key command that specifies the name of the file containing all the model parameters
described in the first section.  It is usually called "something.model".

Up to five models can be introduced by separate \Tt{PARAM} lines and are assigned numbers
starting with zero upwards, in order of their occurrence.

\subsubsection{\Tt{END}}

Marks the end of the parameter input stage and the start of the model description.




%#include "util.hpp"
%#include "geom.hpp"
%#include "cell.hpp"
%#include "data.hpp"
%
%void paramin ( char*, int );
%
%int params ( FILE *run ) {
%// read the global behaviour values and model data
%// NB commands must come in the order specified below as there is no loop 
%int	n, io, nmodels;
%float	s, sget, sput;
%char	line[222];
%	Data::shrink = 0.0;	// used as 1-shrink (-ve = mass factor)
%	Data::scalein = 0.1;	// default unless specified by SCALE <in> <out>
%	Data::scaleout = 1.0;	// unless specified by SCALE
%	LOOP {
%		io = read_line(run,line);
%		if (io < 0) { Pt(Unexpected end of file\n) exit(1); }
%		if (io==0) continue;
%		if (line[0]=='#') continue; // NB comments (#) only at start
%		break;
%	}
%	if (line[0]=='N' && line[2]=='R') {
%		Data::norun = 1;		// NORUN flag (skip driver)
%		read_line(run,line);
%	}
%	if (line[0]=='N' && line[2]=='M') {
%		Data::norun = 2;		// NOMOVE flag (no linker,shaker,bumper) 
%		read_line(run,line);
%	}
%	if (line[0]=='N' && line[2]=='V') {
%		Data::noview = 1;		// NOVIEW flag (skip viewer)
%		read_line(run,line);
%	}
%	if (line[0]=='H' && line[2]=='D') {
%		Data::hidden = 1;		// HIDDEN flag (skip viewer)
%		if (io > 7) {
%			sscanf(line+6,"%d", &n);
%			Data::hidden = n;
%		}
%		read_line(run,line);
%	}
%	if (line[0]=='S' && line[1]=='C') { // output SCALE factor (saved in world->far)
%		Ps(line) NL
%		sscanf(line+6,"%f %f", &sget, &sput);
%		Data::scalein = sget;
%		Data::scaleout = sput;
%		read_line(run,line);
%	}
%	if (line[0]=='S' && line[2]=='R') { // SHRINK factor (saved later in scene->far)
%		Ps(line) NL
%		sscanf(line+7,"%f", &s);
%		Data::shrink = s;
%		read_line(run,line);
%	}
%	if (line[0] != 'P') {
%		printf("Need a PARAM filename\n");
%		return 2;
%	}
%	nmodels = 0;
%	while (1) { char atomtype; // read in param file(s)
%		printf("Reading parameters from %s\n", line+6);
%		paramin(line+6,nmodels);
%		nmodels++;
%		read_line(run,line);
%		if (line[0] == 'P') continue;
%		if (line[0] == 'M' || line[0] == 'E') { // MODEL or END to finish
%			printf("End of parameters\n\n");
%			Pt(Read) Pi(nmodels) NL
%			DO(i,nmodels) // set the atomic bond lengths
%			{ float len = Data::model[i].sizes[Data::depth]
%				    + Data::model[i].bonds[Data::depth];
%				printf("Model %d is", i);
%				if (Data::model[i].moltype==0) { // protein
%					Data::bondCA = len;
%					Pt(protein with) Pr(Data::bondCA) NL
%				}
%				if (Data::model[i].moltype==1) { // nucleic
%					Data::bondPP = len;
%					Pt(nucleic with) Pr(Data::bondPP) NL
%				}
%				if (Data::model[i].moltype==2) { // chemistry
%					Pt(chemical) NL
%				}
%				if (Data::model[i].moltype==3) { // cells
%					Pt(cells) NL
%				}
%			}
%			break;
%		}
%		Pt(Bad) Ps(line) NL exit(1);
%	}
%	Data::nmodels = nmodels;
%	if (line[0] == 'M') { int model;
%		sscanf(line+6,"%d", &model);
%		printf("Using model %d\n", model);
%	}
%}
%
%void paramin ( char *param, int n ) {
%int	i, j, io;
%FILE	*dat;
%char	*at;
%char	line[222];
%char	name;
%Data	*model = Data::model+n;
%	dat = fopen(param,"r");
%	if (!dat) { printf("%s parameter file not found\n", param); exit(1); }
%	io = read_line(dat,line);
%	model->moltype = model->subtype = 0; // default (any protein-like chain)
%        if (line[0]=='P') model->subtype = 1; // PROTein (CA-CA = 3.8)
%        if (line[0]=='R') { model->moltype = 1; model->subtype = 0; } // RNA
%        if (line[0]=='D') { model->moltype = 1; model->subtype = 1; } // DNA
%        if (line[1]=='H') model->moltype = 2; // CHEM (needs fixing)
%        if (line[1]=='E') model->moltype = 3; // CELLs
%	model->colours = 0;		// default = colour by level
%	if (strlen(line) > 4) {
%        	if (toupper(line[5])=='S') model->colours = 1; // PROT with SSE coloured red/green
%        	if (toupper(line[5])=='B') model->colours = 2; // flash Bumps in a different colour
%	}
%	name = line[io-1];
%	i = 0;
%	while (1) {
%		Pi(i)
%		if (read_line(dat,line) < 0) break;
%		if (line[0] == '/') continue;
%		*(strstr(line,"/")) = '\0';
%		at = line;
%		for (j=0; j<N; j++) { int in;
%			sscanf(at,"%d", &in);
%			if (in != 0) Data::depth = max(Data::depth,j);
%			printf("%5d ", in);
%			at = strstr(at,","); at++;
%			switch (i) {
%				case 0 :
%					model->shape[j] = in;
%					//	-ve = render as wireframe
%					break;
%				case 1 :
%					model->sizes[j] = 0.1*(float)abs(in);
%					//	-ve = use values in prox and dist 
%					if (in<0) model->local[j] = 1; else  model->local[j] = 0;
%					break;
%				case 2 :
%					model->bumps[j] = 0.1*(float)in;
%					//	-ve
%					break;
%				case 3 :
%					model->links[j] = in;
%					//	-ve
%					break;
%				case 4 :
%					model->chain[j] = in;
%					//	-ve = circular
%					break;
%				case 5 :
%					model->kicks[j] = 0.01*(float)in;
%					//	-ve
%					break;
%				case 6 :
%					model->keeps[j] = 0.01*(float)in;
%					//	-ve = shell for types 1,3 or tube = open
%					break;
%				case 7 :
%					model->bonds[j] = 0.1*(float)in;
%					//	-ve = no bond between cousins
%					if (in<0) model->split[j] = 1; else  model->split[j] = 0;
%					break;
%				case 8 :
%					model->repel[j] = 0.01*(float)in; // hard bump
%					//	-ve don't repel children within parent
%					break;
%				case 9 :
%					model->rejel[j] = 0.001*(float)in; // soft bump
%					//	-ve don't repel children within parent
%					break;
%			}
%		} NL
%		i++;
%		if (i==N) break;
%	}
%	fclose(dat);
%}
