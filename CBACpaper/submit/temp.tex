\subsection{Test data}

\subsubsection{Colliding Hilbert chains}

To retain a connection to macromolecules, idealised test data was based on a linear chain of atoms.
This chain was then 'packaged' into a hierarchy of spherical objects, each of which contained
eight children arranged as a cube.   To maintain equal bond-lengths between the atoms, the path of the
chain followed a recursive Hilbert curve, in which each level of the hierachry is identical\footnote{
A Hilbert curve ({\tt https://en.wikipedia.org/wiki/Hilbert\_curve}) is the spatial equivalent of a 
Gray code ({\tt https://en.wikipedia.org/wiki/Gray\_code}) in which successive elements are
only one step away from their neighbours.
}.   This arrangement generates a homogeneously packed chain of atoms which allows the effects of
collisions to be monitored without the added complication of variable density.

Collisions between these objects were then engineered by applying an external
motion to propel two identical objects into each other.   To avoid a direct "head-on"
collision, the objects were displaced by half their radius from their line of approach.
As the chain forms a cube, this means that the collision surface encompases half a face of each cube.
During the collision, the number of bumping atoms was monitored between the two objects
and also within each object.

\paragraph{8+8 crash:\\}

To investigate the the contribution of the soft repulsion component of higher level objects,
the first construct of interest involves the collision of two cubes of eight atoms.  The
repulsion strength at both levels was set to a value of 1 for both hard and soft modes
(although the atomic level only has hard repulsion).   With these values, the containing
spherical shells repelled each other before the atoms could make contact.

The $soft$ parameter value was then decreased, allowing interpenetration of the high
level spheres,  until the internal atoms made contact.   This occured when $soft = hard/5$
but most of the displacement still derived from the high-level soft repulsion component
and the internal arrangement of the atoms was almost unchanged.   With no soft repulsion,
the atomic configuration was markedly displaced and as a compromise, $soft = hard/10$
was taken as a combination that allowed a contribution from each level.

\paragraph{64+64 crash:\\}

The next level of model considered was the 64 atom chain and keeping the values established
above for the first level in the hierarchy, an equivalment evaluation was made for the second level.
As with the smaller test object, the first level spheres initally made contact when $soft = hard/5$,
however, because of the added buffering effect of the first-level spheres, the atoms remained
well separated between the two colliding objects even when $soft = hard/10$.  To compensate for
this additive contribution, the $hard$ parameter value on both levels was halved and the 10\%
ratio to the $soft$ parameter retained giving $hard = 0.5, soft = 0.05$. 

\begin{figure}
\centering
\subfigure[64+64]{
\epsfxsize=190pt \epsfbox{figs/cube/crash2.eps}
}
\subfigure[512+512]{
\epsfxsize=192pt \epsfbox{figs/cube/crash3.eps}
}
\subfigure[4096+4096]{
\epsfxsize=390pt \epsfbox{figs/cube/crash4.eps}
}
\caption[]{
\label{Fig:cubes}
\begin{footnotesize}
{\bf Idealised chain collisions} of increasing size. $a$) Two 64 atom chains are directed into each
other from left and right.   Groups of 8 atoms are enclosed in transparent green virtual spheres
which in turn are enclosed in a larger red spheres.  $b$) A third level is added to the hierarchy
with the virtual spheres and atoms now coloured by their collision state: dark-blue designates no
clash while cyan to green to yellow colours are associated with collisions of increasingly distant
relatives.  Cyan = same parent (cousins), green = second cousins, yellow = third. $c$) The number
of levels is increased to four but with smaller atoms and the virtual spheres not rendered to allow
the distribution of clashes to be visualised at the atomic level.  The colliding bodies remain distinct
with only occasional flashes of red (fouth cousins) indicating clashes between atoms in the collision
interface (which runs bottom left to top right).   The distortion of the structures has been 
distributed evenly through many localised (green) interactions.
\end{footnotesize}
}
\end{figure}


\paragraph{512+512 crash:\\}

The evaluation protocol was extended to the 3-level hierarch of 512 atoms per chain.
Transfering the values from the previous test again led to a lack of direct contact at the
atomic level and these were reduced to $hard = 0.2$, keeping the $soft = hard/10$ ratio.
This produced a result at the mid-point of the collision (when the centroids of the bodies
draw level on their collision course) that was comparable to the smaller systems.

\paragraph{4096+4096 crash:\\}

For the largest model tested, with the chain packaged into 4 levels of containers,  the
progression of reducing the values of the $hard$ and $soft$ parameters was continued.
However, this led to a marked number (100s) of steric violations at the atomic level both
between and within the colliding bodies.   Keeping the two parameters at their previous
levels ($hard = 0.2$, $soft = 0.02$) the number of clashes between the bodies decreased
(10s) but the number of internal violations remained high.

As this model extends beyond the normal size range of compact biological molecules,
no further experimentaion was made.   Of greater interest is the degree of distortion 
observed in the "crumple-zone" and to investigate this a more realistic protein model
was used.


\subsubsection{Colliding multiple protein domains}

To progress to a more biologically realistic system and also to introduce a variety
of container shapes, the small chemotaxis-Y protein (PDB code: {\tt 3chy}), was used
and modelled with tubes to contain its secondary structure elements and an ellipsoid
to contain the protein.    The structure was also stabilised with 'hydrogen-bond'-like
links between the $i..i+3$ and $i..i+4$ positions in the \AHs\ and links between
hydrogen-bonded positions in the \BS.

A series of multi-domain models were then constructed with the protein chain as a
node on a Hilbert curve giving models of 1, 8 and 64 domains.   The link between 
domains was also optionally broken, making the domains equivalent to subunits.
Collisions between these constructs was engineered as above with the coordinates
being saved at the start of the run and at the end, after the structures were well
past each other.    To monitor the distortion experienced by the domains/subunits
during the collision, each domain in the starting structure was comapred to each
domain in the final structure and the smallest, mean and largest root-mean-square
deviation (RMSD) recorded. 

\paragraph{Single domain collision:\\}

Two single domain structures were collided using the same parameters as determined
for the equivalent sized "cubic" chains of the previous subsection ($hard = 1.0$, $soft = 0.1$).
Comparing combinations of the two starting and two final strutctures, the mean RMS
deviation was 3.6\AA , which is not far in excess of the 2.4\AA\ mean deviation seen
when the two structures travel the same distance but do not collide.   For a protein
of this size, with any RMSD under 5\AA\ the stuctures retain a clear correspondance. 

Increasing the $soft$ parameter to 0.2 led to less distortion (mean RMS = 3.1) which
is closer to the un-collided value.   While, retaining the same $hard:soft$ ratio
with $hard = 0.5$ gave a deviation of 4.0\AA, which is still acceptable and only
when the parameter values were dropped as low as $hard = 0.5$, $soft = 0.01$ was the
5\AA\ 'threshhold' exceded.  

\begin{figure}
\centering
\subfigure[]{
\epsfxsize=190pt \epsfbox{figs/prot/net3chy.eps}
}
\subfigure[]{
\epsfxsize=190pt \epsfbox{figs/prot/bad3chy.eps}
}
\caption[]{
\label{Fig:bumps}
\begin{footnotesize}
{\bf Idealised chain collisions} of increasing size. $a$) Two 64 atom chains are directed into each
with only occasional flashes of red (fouth cousins) indicating clashes between atoms in the collision
interface (which runs bottom left to top right).   The distortion of the structures has been 
distributed evenly through many localised (green) interactions.
\end{footnotesize}
}
\end{figure}

\paragraph{8+8 domain collision:\\}

The same approach was applied to the larger 8-domain construct over a series of collisions 
ranging from a glancing blow to almost head-on collision.  The tests were repeated
both with the domains in a continuous chain and as separate (unbonded) subunits.
The repulsion of the highest level sphere was set initially low with $hard=0.2, soft=0.1$
and the two $hard$/$soft$ parameter combinations applied to the secondary structure and domain 
levels were examined above, were tested.

The mean RMSD over the domains before and after the collision seldom exceded the (self-imposed)
threshold if 5\AA\ RMSD for both parameter combinations and, as would be expected, the deviations
were slightly reduced when the domains were treated as subunits.   However, some of the worst
distortions seen in the full (almost head-on) collisions had markedly elevated RMSD values, up to 8\AA.
Despite excluding the 5 residue amino and 5 residue carboxy terminal linking segments from the comparison
(which often must diverge in different directions), examination of these worst cases revealed that a
large component of the error often came from displacement of the un-tethered C-terminal \AH.

To reduce this source of error, a link as added betweent the mid-points of the 5-residue N- and C-terminal
segments that link domains.  However, this had little effect, and even led to a slight overall increase in RMS
deviations.   On visual examination the distortions appeared to remain associated with the terminal helix
which was still able to be markedly displaced despite the C-terminal tether but now this occurred more
at the expense of distupting other neighbouring secondary structure elements.

To allow contact at the atomic level between the colliding objects,
it is desirable to limit the repulsion from the higher levels of the hierarchy.
Such contact was seen visually and measured as transient steric clashes with the $hard$:$soft$
lower parameter combinations used above which were (expressed as a percentage of the atomic level): 
20:10, 50:20, 50:20 for the protein, domain and secondary structure levels respectively.
These were adopted throughout all further protein simulations.

%
%hard:soft = 50:20
%    subunit
%	no link  miss     frame = 990 aa = 25 bb = 25 ab = 0	RMS:  min = 2.296 ave = 2.82268 max = 3.324
%	no link  clip     frame = 860 aa = 38 bb = 31 ab = 2	RMS:  min = 2.315 ave = 3.02307 max = 3.688
%	no link  half     frame = 830 aa = 53 bb = 82 ab = 7	RMS:  min = 2.608 ave = 3.82729 max = 5.013
%	no link  full     frame = 350 aa = 10 bb = 5 ab = 5	RMS:  min = 4.195 ave = 5.13120 max = 6.289
%
%	link on  miss     frame = 990 aa = 18 bb = 28 ab = 0	RMS:  min = 2.271 ave = 2.81993 max = 3.483
%	link on  clip     frame = 1010 aa = 37 bb = 35 ab = 2	RMS:  min = 2.616 ave = 3.04413 max = 3.581
%	link on  half     frame = 770 aa = 67 bb = 66 ab = 8	RMS:  min = 2.488 ave = 3.84289 max = 5.183
%	link on  full     frame = 1120 aa = 125 bb = 132 ab = 5	RMS:  min = 3.481 ave = 5.26102 max = 7.018
%
%    domains
%	no link  miss     frame = 990 aa = 37 bb = 22 ab = 0	RMS:  min = 2.400 ave = 3.06962 max = 4.162
%	no link  clip     frame = 1050 aa = 53 bb = 85 ab = 2	RMS:  min = 2.344 ave = 3.72846 max = 5.174
%	no link  half     frame = 1030 aa = 104 bb = 105 ab = 5	RMS:  min = 2.815 ave = 4.45303 max = 6.800
%	no link  full     frame = 370 aa = 13 bb = 18 ab = 3	RMS:  min = 3.749 ave = 5.25678 max = 7.099
%
%	link on  miss     frame = 990 aa = 29 bb = 24 ab = 0	RMS:  min = 2.101 ave = 3.12647 max = 4.451
%	link on  clip     frame = 1190 aa = 64 bb = 77 ab = 3	RMS:  min = 2.870 ave = 3.93120 max = 5.458
%	link on  half     frame = 1130 aa = 91 bb = 94 ab = 3	RMS:  min = 3.027 ave = 4.71534 max = 8.282
%	link on  full     frame = 1400 aa = 135 bb = 120 ab = 7	RMS:  min = 3.717 ave = 5.72397 max = 8.475
%
%
%hard:soft = 100:20
%    subunit
%	no link  miss     frame = 990 aa = 19 bb = 32 ab = 0	RMS:  min = 2.369 ave = 2.91038 max = 3.504
%	no link  clip     frame = 390 aa = 12 bb = 11 ab = 3	RMS:  min = 2.501 ave = 3.04747 max = 3.628
%	no link  half     frame = 790 aa = 63 bb = 45 ab = 5	RMS:  min = 2.634 ave = 3.60256 max = 4.835
%	no link  full     frame = 1080 aa = 97 bb = 106 ab = 7	RMS:  min = 3.799 ave = 4.66473 max = 5.485
%
%	link on  miss     frame = 990 aa = 25 bb = 12 ab = 0	RMS:  min = 2.066 ave = 2.75384 max = 3.274
%	link on  clip     frame = 810 aa = 20 bb = 37 ab = 3	RMS:  min = 2.540 ave = 3.08316 max = 3.814
%	link on  half     frame = 780 aa = 49 bb = 55 ab = 8	RMS:  min = 2.358 ave = 3.68389 max = 4.761
%	link on  full     frame = 1060 aa = 107 bb = 98 ab = 8	RMS:  min = 3.525 ave = 4.56773 max = 5.496
%
%    domains
%	no link  miss     frame = 990 aa = 36 bb = 23 ab = 0	RMS:  min = 2.088 ave = 2.93714 max = 4.186
%	no link  clip     frame = 700 aa = 39 bb = 67 ab = 2	RMS:  min = 2.352 ave = 3.64122 max = 5.143
%	no link  half     frame = 1320 aa = 75 bb = 98 ab = 3	RMS:  min = 2.739 ave = 4.40182 max = 7.633
%	no link  full     frame = 1610 aa = 120 bb = 104 ab = 6	RMS:  min = 3.309 ave = 4.83634 max = 8.263
%
%	link on  miss     frame = 990 aa = 25 bb = 34 ab = 0	RMS:  min = 2.269 ave = 3.11580 max = 4.255
%	link on  clip     frame = 1020 aa = 61 bb = 64 ab = 3	RMS:  min = 2.617 ave = 3.79832 max = 5.482
%	link on  half     frame = 1350 aa = 80 bb = 121 ab = 9	RMS:  min = 2.786 ave = 4.37453 max = 6.655
%	link on  full     frame = 990 aa = 85 bb = 108 ab = 3	RMS:  min = 3.041 ave = 5.09632 max = 8.449

\begin{singlespace}
\begin{verbatim}
------------------------------------------------------------------------------
    0,    50,   50,   50,  100,  // hard  :kick size = []/100
    0,    20,   20,   20,  100,  // soft  :kick size = []/100

subunit

miss	min = 1.954 ave = 2.50335 max = 3.183
clip	min = 2.082 ave = 2.72707 max = 3.441
half	min = 2.338 ave = 3.24126 max = 4.075
full	min = 2.959 ave = 3.94160 max = 5.076

domain

miss	min = 2.025 ave = 3.15143 max = 4.868
clip	min = 2.298 ave = 3.44083 max = 5.519
half	min = 2.243 ave = 4.08091 max = 6.746
full	min = 2.947 ave = 4.82165 max = 8.467


------------------------------------------------------------------------------
    0,   100,  100,  100,  100,  // hard  :kick size = []/100
    0,    20,   20,   20,  100,  // soft  :kick size = []/100

subunit

miss	min = 1.848 ave = 2.57102 max = 3.307
clip	min = 2.207 ave = 2.70812 max = 3.240
half	min = 2.261 ave = 3.09951 max = 4.096
full	min = 2.840 ave = 3.66765 max = 4.670

domain

miss	min = 1.850 ave = 3.14264 max = 4.838
clip	min = 2.375 ave = 3.50274 max = 5.259
half	min = 2.335 ave = 3.69418 max = 6.195
full	min = 2.572 ave = 4.16718 max = 6.271

------------------------------------------------------------------------------
\end{verbatim}
\end{singlespace}

\paragraph{64+64 domain collision:\\}

The next larger complete Hilbert curve of protein domains comprises 64 domains (8256 residues)
and the interaction of two such objects approaches the computational limits of what can be run
on a laptop computer in real time.   Nevertheless, a small number of test were conducted
with the highest level in the hierarchy cosisting of a sphere of unlinked proteins,
each composed of 8 linked domains as employed above.    The $soft:hard$ prameter combination
for this level was again reduced to 10:5 (percent of atomic).

The mean domain start/final RMSD value after a half-face collision was 6.5\AA.  No inter-object
clashes were seen at the atomic level but a marked number of intra-object clashes built-up
during the collision.   It seemed likely that this higher than expected RMSD value 
was therefore a consequence of the speed of the collision giving insufficient time for the
'shock-wave' compression of the material to disipate throught the domains.

The collision was then re-run in 'slow-motion' with a time-step reduced by a factor of 10.
(Collisions that normally take a few minutes now took 30).  The mean RMSD then dropped to
the acceptable value of 5.3\AA\ but intra-body clashes were still prominant during the collision.
% RMS:  min = 3.422 ave = 5.36615 max = 8.861

\begin{figure}
\centering
\subfigure[]{
\epsfxsize=190pt \epsfbox{figs/prot/bump1.eps}
}
\subfigure[]{
\epsfxsize=190pt \epsfbox{figs/prot/bump2.eps}
}
\subfigure[]{
\epsfxsize=190pt \epsfbox{figs/prot/bump3.eps}
}
\subfigure[]{
\epsfxsize=190pt \epsfbox{figs/prot/bump4.eps}
}
\caption[]{
\label{Fig:bumps}
\begin{footnotesize}
{\bf Idealised chain collisions} of increasing size. $a$) Two 64 atom chains are directed into each
with only occasional flashes of red (fouth cousins) indicating clashes between atoms in the collision
interface (which runs bottom left to top right).   The distortion of the structures has been 
distributed evenly through many localised (green) interactions.
\end{footnotesize}
}
\end{figure}
