\section{Conclusions}

In this chapter we have described a general multi-scale modelling
and simulation system that can be easily customised to any large
system both in terms of setting-up the model geometry and in
directing the behaviour of the simulation.

The use of a multi-level approach to the treatment of collisions
means that, at higher levels, the system does not depend strongly
on the choice of size or shape assigned to each level.   Thus,
crude shape approximations do not determine the direct behaviour
of the system but just alert the lower level to check for
collisions.   The propagation of this cascade reaches down to the 
atomic level where eventually the "hard-sphere" repulsion is
implemented giving a more realistic representation of the true
excluded volume. 

Unlike a conventional MD or MC simulation, the approach we have
described cannot be expected to generate predictive behaviour 
from a summation of atomic interactions.  It relies instead on
a hierarchy of user or data defined constraints, such as the 
definition of secondary structures and their packing interactions.
However, these constraints can be incomplete and the missing 
constraints will allow freedom for change in the system that
will be stochastically explored through the underlying random
motion that can be applied at every level.

The example we provided of the actin/myosin motor is typical of
the scale of application that the system can accommodate, both 
in terms of number of atoms and size of displacement.  While
there are many aspects of the behaviour of the system that can
be adjusted to reproduce realistic behaviour, we see its use
mainly as a way of setting up the physical constraints for a 
complex system that can then be used to test if a hypothesis 
about the dynamics of the system is consistent with its physical
structure.
