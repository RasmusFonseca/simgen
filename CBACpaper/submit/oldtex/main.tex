\section{\TT{main}}

The program \NAME\ simulates the motion and interaction of a collection of objects.
The objects can be a sphere, tube or ellipsoid and can be bonded or linked together.
Each object can contain other objects (called children) and the motion of a parent
object applies to all its offspring.   The objects have no mass or momentum and any
movement is applied as a fixed-size step which is made irrespective of whether the
move creates a violation of any constraint, such as bond-length or leads to an 
undesired steric overlap between objects.   

In the same way that objects move irrespective of any constraint, so too, bond
lengths and steric constraints are independently enforced by applying a fixed length
movement to restore the bond length or separate an overlap.  Precedence in the application
of these, often conflicting corrections, is random as each algorithm executes as an
asynchronous process (a thread), all operating on the same set of positional coordinates.

The process generating motion is the \TT{shaker}, bonds and link lengths are fixed
by the \TT{linker} and steric clashes by the \TT{bumper}.   In addition there is a 
routine that keeps children within the boundaries of their parent (the 
\TT{keeper}) and one that helps maintain local geometry (\TT{tinker}) and the
configuration is visualised (using openGL) by the \TT{viewer}.

The structure of the objects is set-up through a command script that is interpreted
by the \TT{models} routine and the parameters that control their behaviour is read
by the \TT{params} routine.   Both these routines have many built-in settings that
allow the constraints commonly found in protein and nucleic acids (such as their
secondary structures) to be automatically generated without having to specify every
detail.

\NAME\ is currently configured to accept a hierarchy of objects in up to 10 levels
(not all of which need be used).   The top level is referred to as the "world" and
the lowest as "atomic" (not because the objects necessarily represent atoms but only
because they have no further divisions).  Objects interact directly (through bumps,
bonds and links) only with other objects on the same level.   Besides following the
geometric transformations of their parents, interaction between levels is confined 
to keeping families (parent and children) together.   As mentioned above, this can
be enforced using the \TT{keeper} but the default action is for the parent to track
the centroid position of its children and for the children to shift together so their
centroid tracks the position of their parent.   This mutual co-location is implemented
by the \TT{center} ("er" not "re") routine.

In the input stream (read by \TT{models})
each group of objects is introduced with the line \Tt{GROUP 0 N}, where \Tt{N} is
the number of children in the group.   For example:
\begin{singlespace}
---------------------------------------------------------------------------------------------------
\begin{verbatim}
        GROUP 0 1
            GROUP 0 2
                GROUP 0 3
                    ATOM...
                    ATOM...
                    ATOM...
                GROUP 0 4
                    ATOM...
                    ATOM...
                    ATOM...
                    ATOM...
\end{verbatim}
\ \ \ \ ---------------------------------------------------------------------------------------------------
\end{singlespace}
specifies a world consisting of a single object that contains one child that itself
has two children with three and four children each at the atomic level.  The atoms at
the lowest level are specified by the \Tt{ATOM} command, which conveniently, has a
format that is identical to that used by the Protein structure DataBank (PDB), which
is the source of most of the structures being simulated.

The above example does not specify the nature of the objects in each group (although
those specified by \Tt{ATOM} commands are, by default, spheres).  Objects on the same
level within a group (such as the \Tt{ATOM}s or their parents in the above example)
are assumed to have the same properties, and to prevent repeated specification of
these, they are defined in a parameter file (read by \TT{params}).  This file consists
of a set of rows and columns with each column specifying a set of parameters for
each level.

After a model has been set-up and executed, after an initial period of equilibration
(during which conflicting constraints will attempt to resolve themselves), the structure
will remain still or jiggle around, depending on the degree of motion assigned to the
objects on each level.   Behaviour of a more purposeful nature can be introduced through
the \TT{driver} routine which implements a set of user-specified behaviours, such as 
moving components of the model to execute some action.   Irrespective of the instructions
issued from this routine, all the remaining routines will continue to act independently

Each routine will be elaborated below, beginning with the set-up routines (\TT{models}
and \TT{params}) followed by the main routines: \TT{shaker}, \TT{linker}, \TT{bumper}
and \TT{keeper} that implement their specifications.  Finally, the minor routines:
\TT{tinker}, \TT{sorter}, \TT{center} and \TT{looker} will be described, along with
some specialised routines under the title \TT{fixers}.

%#include "util.hpp"
%#include "geom.hpp"
%#include "cell.hpp"
%#include "data.hpp"
%
%int argcin; char** argvin;
%
%void params ( FILE* );
%void models ( FILE* );
%void driver ();
%void bumper ();
%void shaker ();
%void potter ( int, int );
%void tinker ( float );
%void keeper ( Cell* );
%void bonder ( Cell* );
%void linker ( Cell* );
%void center ();
%void sorter ();
%void viewer ( int, char** );
%void presort ();
%void driver (); // in application directory
%void looker (); // in application directory
%Cell* setScene ( Cell* );
%void fixSSEaxis( Cell*, float );
%
%//           100000000 = 1/10 sec.
%long nanos = 100000000;	// good speed for watching
%int  delay = 1;//10;
%int  lastlook = 0;
%
%float	atom2pdb;	// scale factor
%
%void *acts ( void *ptr )
%{
%	sleep(1); // delay start to allow stuff to be set
%	if (Data::norun) printf("Starting driver (but not used)\n");
%		    else printf("Starting driver\n");
%	while (1) {
%		if (Data::norun==0) driver(); // application control (user provided)
%		timeout(nanos);
%	}
%}
%
%void *look ( void *ptr )
%{
%	sleep(60);
%	printf("Starting looker \n");
%	while (1) {
%		sleep(2);
%		if (Data::frame > 99) {
%			looker(); // watches and measures things (user provided)
%			exit(1);
%		}
%		lastlook++;
%	}
%}
%
%void *fixs ( void *ptr )
%{
%float	weight = (float)nanos/100000000.0; // = 1 when speed = 1, .1 for 10, .01 for 100
%	sleep(1); // delay start to allow ends to be set
%	printf("Starting fixers \n");
%	weight *= 0.1;
%	while (1) {
%//		fixSSEaxis(Cell::world,weight);
%		timeout(nanos);
%/*
%		fixRNAaxis(world,0,weight);
%		nanosleep(&wait,NULL);
%		fixRNAstem(world,0,weight);
%		nanosleep(&wait,NULL);
%		fixDOMaxis(world,0,1.0);
%		fixANYaxis(world,0,weight);
%		nanosleep(&wait,NULL);
%		reset2axis(world,0,weight);
%*/
%	}
%}
%
%void *move ( void *ptr )
%{
%	printf("Starting mover\n");
%	while (1) {
%		Data::frame++;
%		shaker();		// makes random displacements at all levels
%		keeper(Cell::world);	// keeps children inside their parent
%		center();		// keeps children centred on their parent
%//		tinker(0.1);		// geometry patch-up using stored local data
%		timeout(nanos);
%	}
%}
%
%void *link ( void *ptr )
%{
%	printf("Starting linker \n");
%	while (1) {
%		bonder(Cell::world);
%		linker(Cell::world);
%		timeout(nanos);
%	}
%}
%
%void *rank ( void *ptr )
%{
%	printf("Starting sorter \n");
%	while (1) {
%		sorter(); // sorts the ranked lists
%		timeout(1,nanos); // <------------NB + 1 sec.
%	}
%}
%
%void *bump ( void *ptr )
%{
%	printf("Starting bumper \n");
%	while (1) {
%		bumper();
%		timeout(nanos/100);
%	}
%}
%
%void *pots ( void *ptr )
%{
%struct  timespec start, finish;
%long	runtime, delay;
%int	m = Data::depth;
%char	*flag = (char*)ptr;
%int	cycles = 1;
%	printf("Starting molecular dynamics for %s\n", flag);
%	while(1) { timespec start,finish;
%   		clock_gettime(CLOCK_REALTIME, &start);
%		if (flag[0]=='a') {
%			potter(m,cycles);
%		} else {
%			DO1(i,m-1) potter(i,(cycles*m)/i);
%		}
%		clock_gettime(CLOCK_REALTIME, &finish);
%		runtime = timedifL(start,finish);
%		if (runtime < nanos) {
%			delay = nanos - runtime;
%			//Ps(flag) Pt(waiting for) Pi(delay) Pi(cycles) NL
%			timeout(delay);
%			cycles++;
%		}
%	}
%}
%
%void *view ( void *ptr )
%{
%	printf("Starting viewer \n");
%	viewer(argcin,argvin); // sets-up graphical objects (types) thrn runs by callback
%}
%
%main ( int argc, char** argv ) {
%pthread_t thread[10];
%FILE    *run;
%int	i, j, m, n;
%long    rseed = (long)time(0);
%        srand48(rseed);
%	if (argc>2) { int speed;
%		sscanf(argv[2],"%d", &speed);
%		if (speed > 0) nanos /= speed;
%		if (speed < 0) nanos *= -speed;
%		Pt(Running at) Pi(speed) NL
%	}
%	run = 0;
%        if (argc > 1) run = fopen(argv[1],"r");
%        if (!run) {
%                printf("Assume test.run\n");
%                run = fopen("test.run","r");
%                if (!run) { printf("No test.run file\n"); return 1; }
%        }
%	Cell::world = new Cell;		// create the top-level cell structure (set in models())
%	params(run);
%	models(run);
%	fclose(run);
%Pi(Cell::total) NL
%Pi(Data::depth) NL
%Pr(Data::shrink) NL
%Cell::world->print();
%	Cell::world->move(-Cell::world->xyz);
%	Pt(Reset) Pv(Cell::world->xyz) NL
%	Cell::scene = setScene(Cell::world);
%	Cell::scene->sort = 0;
%	presort();
%	Data::frame = -10;
%	argcin=argc; argvin=argv;
%	if (Data::noview) {
%		Pt(No viewing) NL
%	} else {
%		pthread_create( thread+0, 0, view, 0 );
%	}
%	pthread_create( thread+1, 0, rank, 0 );
%	if (Data::norun < 1) {
%		pthread_create( thread+2, 0, move, 0 );
%		pthread_create( thread+3, 0, bump, 0 );
%		pthread_create( thread+4, 0, link, 0 );
%	}
%//	pthread_create( thread+5, 0, fixs, 0 );
%//	pthread_create( thread+6, 0, acts, 0 );
%//	pthread_create( thread+7, 0, look, 0 );
%/* toy MD calls
%	pthread_create( thread+8, 0, pots, (void*)"atoms" );
%	pthread_create( thread+9, 0, pots, (void*)"other" );
%*/
%	sleep(99999);
%}
%
%/* code to stretch chain
%if (Data::frame<200) {
%DO(i,Cell::total) { Cell *c=Cell::uid2cell[i];
%if (c->level==3) c->xyz.x -= 0.005*(i-Cell::total/2);
%}
%DO(i,Cell::total) { Cell *c=Cell::uid2cell[i];
%if (c->level==2) { Vec x = (c->endC - c->endN)*0.5;
%c->xyz = c->starts->xyz & c->finish->xyz;
%c->endN = c->xyz-x; c->endC = c->xyz+x; }
%} 
%nanosleep(&wait,NULL);
%continue;
%}
%*/
%
%main  : main.cpp potter.o driver.o looker.o params.o models.o viewer.o shaker.o keeper.o linker.o bumper.o sorter.o fixers.o util.o cell.o data.o geom.o util.hpp cell.hpp geom.hpp Vec.hpp Mat.hpp
%	c++ -ggdb main.cpp -o main potter.o driver.o looker.o params.o models.o viewer.o shaker.o keeper.o linker.o bumper.o sorter.o fixers.o util.o cell.o data.o geom.o -lGL -lGLU -lglut -lpthread -lm
%
%
%potter.o : potter.cpp util.hpp data.hpp geom.hpp Vec.hpp Mat.hpp
%	c++ -c -ggdb potter.cpp
%
%driver.o : driver.cpp util.hpp data.hpp geom.hpp Vec.hpp Mat.hpp
%	c++ -c -ggdb driver.cpp
%
%looker.o : looker.cpp util.hpp data.hpp geom.hpp Vec.hpp Mat.hpp
%	c++ -c -ggdb looker.cpp
%
%params.o : params.cpp util.hpp data.hpp geom.hpp Vec.hpp Mat.hpp
%	c++ -c -ggdb params.cpp
%
%models.o : models.cpp util.hpp data.hpp cell.hpp geom.hpp Vec.hpp Mat.hpp
%	c++ -c -ggdb models.cpp
%
%viewer.o : viewer.cpp util.hpp data.hpp cell.hpp geom.hpp Vec.hpp Mat.hpp
%	c++ -c -ggdb viewer.cpp
%
%shaker.o : shaker.cpp util.hpp data.hpp cell.hpp geom.hpp Vec.hpp Mat.hpp
%	c++ -c -ggdb shaker.cpp
%
%keeper.o : keeper.cpp util.hpp data.hpp cell.hpp geom.hpp Vec.hpp Mat.hpp
%	c++ -c -ggdb keeper.cpp
%
%linker.o : linker.cpp util.hpp data.hpp cell.hpp geom.hpp Vec.hpp Mat.hpp
%	c++ -c -ggdb linker.cpp
%
%bumper.o : bumper.cpp util.hpp data.hpp cell.hpp geom.hpp Vec.hpp Mat.hpp
%	c++ -c -ggdb bumper.cpp
%
%sorter.o : sorter.cpp util.hpp data.hpp cell.hpp geom.hpp Vec.hpp Mat.hpp
%	c++ -c -ggdb sorter.cpp
%
%fixers.o : fixers.cpp util.hpp data.hpp cell.hpp geom.hpp Vec.hpp Mat.hpp
%	c++ -c -ggdb fixers.cpp
%
%cell.o : cell.cpp util.hpp data.hpp cell.hpp geom.hpp Vec.hpp Mat.hpp
%	c++ -c -ggdb cell.cpp
%
%data.o : data.cpp util.hpp geom.hpp Vec.hpp Mat.hpp
%	c++ -c -ggdb data.cpp
%
%geom.o : geom.cpp util.hpp geom.hpp Vec.hpp Mat.hpp
%	c++ -c -ggdb geom.cpp
%
%util.o : util.cpp util.hpp
%	c++ -c -ggdb util.cpp
%
