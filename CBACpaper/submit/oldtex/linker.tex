\section{Bonds and links}

The maintenance of bond and link lengths is very similar and the two routines,
\TT{bonder} and \TT{linker}, that implement this task will be considered together.
Both recursively traverse the object tree looking for things to fix.

\subsection{\TT{bonder}}

All of the decisions about what should be bonded are determined at the model
construction stage and except for the {\tt CHEM} mode (small molecules, which will
be considered later), the bond assignments remain unaltered throughout a simulation.

\subsubsection{Bond lengths}

The \TT{bonder} simply checks if an object has any assigned bonds and if so, uses
the utility {\tt part2cells()} to push them towards their assigned bond length.
The value for the ideal bond length depends on whether the $size$ parameter was 
given as a positive or negative value.  If positive, the value of the $bond$ 
parameter is used (at 1/100 the specified value) and if negative, the starting
value of the bond length is used.   More specifically, this is the value of the
{\tt x} and {\tt z} components of the vector {\tt prox} which is part of the object
class ({\tt Cell}) data and can be varied during the simulation.   The {\tt prox.x} is
the length to the previous object in the chain and  {\tt prox.z} the length to the next.

For both generic and specific bond lengths, a random choice is made over which side
to refine as, doing both in one pass may mean that they simply cancel each other.

\subsubsection{Nucleic acid exceptions}

Exceptions need to be made when bonding tubes in nucleic acids, which occur both at
the secondary structure level (depth-1) as basepairs and the domain level (depth-2)
as segments of double helix. 

For basepairs, if these are part of a double helix, their mid-point (object centre)
is refined against their generic bond-length.   Outside a base-pair, say in a loop region,
the 'secondary structure' can contain multiple nucleotides and no bond length is used.

At the domain level, double-stranded DNA segments will always be bonded end-to-end
at a specific distance that allows the helix to run continuously from one segment to
the next.   On the other hand, when the segment is an RNA stem-loop, the chain can enter
and exit the same end of the tube or, with an insertion, even through the side.
No bonds appear to be enforced (need to fix or check this!).

\subsubsection{Cross-links}

If the number of bond slots allocated by $nbonds$ in the parameter file is more than one,
then a check is made for cross-link bonds that will have been recorded in {\tt bond[2]}
and above.  


\subsection{\TT{linker}}

\subsubsection{Breaking links}

The \TT{linker} follows the same basic outline as the \TT{bonder} but with the main
difference that links can be made and broken during the simulation.   The dynamic creation
of links is not a built-in feature of \NAME\ and must be provided through the user-supplied
\TT{driver} routine.   However, if a link becomes over stretched, it is destroyed in
\TT{linker}.    The default length of a link is the $bump$ diameter (specified in the
parameter file) and the degree to which this can be exceeded is set by the fourth
number read on the {\tt LINK} command (I think).  The value is held in the {\tt next}
value of the {\tt Bond} class which is unused otherwise for links.   The default extension
is 5\%, beyond which the link breaks.  (All this hasn't been checked for ages).

\subsubsection{Preset link lengths}

When links are automatically created for standard secondary structures, they get their
lengths from the routine {\tt setLinks()} that converts the code number held in the {\tt type}
field to the required value.  Links are set not only between the H-bonded connections in the
\AH, i---i+3 and i---i+4, but also between the i-1---i+1 separation along a \Bs.

Links are also set between adjacent \Bs s and between packed \AH--\AH\ pairs and packed
\AH--\Bs\ pairs.   These use one of six discrete values between $1.7\ldots3.2$\AA.

\subsubsection{Link refinement}

Between spheres, links are refined using the {\tt part2cells()} utility but for extended
secondary structures this is undesirable as it will tend to align their midpoints.
Instead, the ideal distance is refined along the line of their closest approach.

Links involving loop 'secondary structures' are refined only when their separation
strays beyond $\pm 50$\% of their bond length. 


%#include "util.hpp"
%#include "geom.hpp"
%#include "cell.hpp"
%#include "data.hpp"
%
%#define BOND 0.5 // 0.3
%#define LINK 0.5 // 0.6
%
%void fixSheet ( Cell* );
%
%void bonder ( Cell *cell )
%{
%float	d;
%Vec	shift;
%Cell	*a, *b;
%int	i, j, n, m = cell->model;
%int	moltype, subtype, level = cell->level;
%float	sislen, brolen, size, basic, kick = BOND;
%float	*sizes = Data::model[m].sizes;
%float	*bonds = Data::model[m].bonds;
%int	*chain = Data::model[m].chain;
%int	*local = Data::model[m].local;
%float	*bondlen;
%char	**bondstr;
%int	nbonds;
%int	end = 0;
%	if (cell==0) return;
%	if (cell->empty) return;
%	n = cell->kids;
%	if (level && cell->parent->solid > 0) return; 
%	moltype = Data::model[m].moltype;
%	subtype = Data::model[m].subtype;
%	size = sizes[level]+bonds[level];
%	if (level && cell->sis && (cell->sis->level != cell->level)) end = -1; // N end of a chain
%	if (level && cell->bro && (cell->bro->level != cell->level)) end =  1; // C end of a chain
%	if (level && chain[level]) { // in a chain
%		sislen = brolen = size;
%		if (moltype==0 && end==0) {  // PROTein chain (not on an end)
%			if (local[level]) {  // using preset lengths (set by -ve size in param.model
%				sislen = cell->prox.x;	// use specific sis length (x)
%				brolen = cell->prox.z;	// use specific bro length (z)
%			}
%			if (sislen >  999.0) sislen = size;	// use generic bond length
%			if (brolen >  999.0) brolen = size;	// dito for bro
%		}
%		if (moltype==1) { // Nucleic acid chain
%			if (level==depth-1) {	// don't refine basepaired stack bond distance to a loop
%				if (cell->sort==0) sislen = brolen = -1.0;
%				if (cell->sis->sort==0) sislen = -1.0;
%				if (cell->bro->sort==0) brolen = -1.0;
%			}
%			if (level==depth-2 && cell->type==2) {	// bond domain tube ends with 0 length bond
%				sislen = brolen = -1.0;		// skips call of part2cells() below
%				if (subtype==1) { Vec v; float w = 0.1;	// link DNA (but don't join RNA ends)
%					if (cell->sis->level==level) { Vec Nter, Cter; // fix sis side
%						v = (cell->sis->endC - cell->sis->endN).norm();
%						Cter = cell->sis->xyz + v*(cell->sis->len*0.6);
%						v = (cell->endN - cell->endC).norm();
%						Nter = cell->xyz + v*(cell->len*0.6);
%						v = (Nter-Cter)*w;
%						cell->move(-v); cell->sis->move(v);
%					}
%				}
%			}
%		}
%		if (cell->bond[0].link < -990) sislen = -1.0;
%		if (cell->bond[1].link < -990) brolen = -1.0;
%		// sis/brolen<0 = flagged by RELINK as not for refinement
%		if (randf()<0.5) {	// fix sis side
%			if (end>-1 && sislen>0.0) part2cells(cell,cell->sis,sislen,kick);
%		} else {		// fix bro side
%			if (end< 1 && brolen>0.0) part2cells(cell,cell->bro,brolen,kick);
%		}
%	}
%	nbonds = abs(chain[level]);
%	if (nbonds > 1) { // may be cross-linked
%		for (int i=2; i<=nbonds; i++) {
%			if (cell->bond==0 || cell->bond[i].to==0) continue;
%			part2cells(cell,cell->bond[i].to,sislen,kick);
%		}
%	}
%	DO(i,cell->kids) bonder(cell->child[i]);
%}
%
%void setLinks ( float bondlen[][20] )
%{ // preset link lengths for protein (bondlen[0]) and nucleic (bondlen[1])
%float	bondCA = Data::bondCA,
%	bondPP = Data::bondPP,
%	distBP = 3.0;
%	// Protein
%	// SSE link lengths
%	bondlen[0][0] = 1.7; bondlen[0][1] = 2.0;		// alpha SSE level
%	bondlen[0][2] = 2.3; bondlen[0][3] = 2.6;		// alpha SSE level
%	bondlen[0][4] = 2.9; bondlen[0][5] = 3.2;		// alpha SSE level
%	bondlen[0][6] = bondlen[0][7] = bondCA*4.7/3.8;		// beta  SSE level
%	bondlen[0][8] = bondlen[0][9] = bondCA*4.7/3.8;		// beta  SSE level
%	// residue link length
%	bondlen[0][10] = bondCA*5.1/3.8; 			// alpha CA (i->i+3)
%	bondlen[0][11] = bondCA*6.2/3.8;			// alpha CA (i->i+4)
%	bondlen[0][12] = bondlen[0][13] = bondCA*6.9/3.8;	// intra beta CA (i->i+/-2)
%	bondlen[0][14] = bondlen[0][15] = bondCA*4.7/3.8;	// inter beta CA level
%	// RNA
%	// base-pair link length
%	distBP *= bondPP;
%	bondlen[1][10] = distBP;			// RNA ladder
%	bondlen[1][11] = distBP;			// 
%}
%
%void linker ( Cell *cell )
%{
%float	bond, kick;
%float	bondlen[4][20];
%int	ctype,csort, ptype,psort;
%int	nlinks = cell->nlinks,
%	level = cell->level,
%	model = cell->model;
%Data	*param = Data::model+model;
%float	basic = param->bumps[level];	// default link length is bump limit
%	moltype = param->moltype;
%	kick = LINK;
%	if (cell==0) return;
%	if (cell->empty) return;
%//	if (level && cell->parent->solid > 0) return; 
%	if (cell->link) { float d;
%		ctype = cell->type; csort = cell->sort;
%		ptype = cell->parent->type; psort = cell->parent->sort;
%		setLinks(bondlen);
%		DO(i,nlinks) { Cell *link = cell->link[i].to; int type; float snap;
%			if (link==0) continue;
%			if (link->empty) continue;
%		  	type = cell->link[i].type;
%		  	snap = 0.01*(float)cell->link[i].next; // %stretch allowed before breaking
%			if (type && moltype<2) bond = bondlen[moltype][type]; else bond = basic;
%			snap = bond*(snap+1.0);
%			d = cell->xyz | link->xyz;
%			if (d>snap) { // break a very long link
%				link = 0;
%				continue;
%			}
%			if (ptype==2 && psort==0) { // in loops, refine only bad links
%				if (link==0) break;
%				if (d > bond*1.5) {
%					part2cells(cell, link, bond, kick);
%				}
%				if (d < bond*0.5) {
%					part2cells(cell, link, bond, kick);
%				}
%				continue;
%			}
%			if (ctype==2 && csort) // in SSEs move along close approach not centres
%			{ Vec	AonB = cell->xyz.vec_on_line(link->endN,link->endC),
%				BonA = link->xyz.vec_on_line(cell->endN,cell->endC),
%				midA = BonA & cell->xyz,
%				midB = AonB & link->xyz,
%				newA = midA, newB = midB;
%				separate(newA,newB,bond,kick);
%				cell->move(newA-midA);
%				link->move(newB-midB);
%				continue;
%			}
%			if (link==0) break;
%			if (moltype==3) bond = 0.5*basic; // half atomic bump length for cells
%			part2cells(cell, link, bond, kick);
%		}
%		// type 2 = SSE tube + sort 2 = beta (links 2 and 3 are between strands)
%		if (moltype==0 && ptype==2 && psort==2) fixSheet(cell);
%	}
%	DO(i,cell->kids) { //float far = -cell->far/(float)data[3];
%		linker(cell->child[i]);
%	}
%}
%
%void fixSheet ( Cell *b )
%{
%Cell	*a, *c;
%Vec	s[3][3], oldcent, newcent;
%Vec	A,B,C, x,y,z, mid, up, to, shift, told;
%int	level = b->level, beta = 2;
%Data	*param = Data::model+b->model;
%float	size = param->bonds[level]+param->sizes[level];
%float	bond = 3.8, span = 6.8, link = 4.7, drop = 1.5;
%float	give = size*0.1, kick = 0.2, fix = 0.5, d;
%int	i, j, flip0, flip2, inbeta;
%float	wa, wb, wc, twist = 0.3;
%	kick = fix = 0.1;
%	span *= size/bond;
%	link *= size/bond;
%	drop *= size/bond;
%	bond = size;
%	// check all exist, are on the same level and most in beta
%	if (!b->link) return;
%	a = b->link[2].to;
%	c = b->link[3].to;
%	if (!a || !b || !c) return;
%	wa = fix*(float)b->link[2].next;
%	wc = fix*(float)b->link[3].next;
%	wb = fix*(wa+wc);
%	if (wb > 1.0) wb = 1.0;
%	if (a->parent->sort != beta || b->parent->sort != beta || c->parent->sort != beta) return;
%	if (a->sis->level != level) return; if (a->bro->level != level) return;
%	if (b->sis->level != level) return; if (b->bro->level != level) return;
%	if (c->sis->level != level) return; if (c->bro->level != level) return;
%	inbeta = 0;
%	if (a->sis->parent->sort == beta) inbeta++; if (a->bro->parent->sort == beta) inbeta++;
%	if (b->sis->parent->sort == beta) inbeta++; if (b->bro->parent->sort == beta) inbeta++;
%	if (c->sis->parent->sort == beta) inbeta++; if (c->bro->parent->sort == beta) inbeta++;
%	if (inbeta < 5) return;
%	// check atoms are far enough apart
%	if (abs(a->atom - c->atom) < 6) return; 
%	if (abs(a->atom - b->atom) < 4) return; 
%	if (abs(b->atom - c->atom) < 4) return; 
%	// check all lengths are reasonable before refining geometry
%	d = vdif(a->xyz,a->sis->xyz)-bond;
%	if (d*d>give) { part2cells(a,a->sis,bond,kick); return; }
%	d = vdif(a->xyz,a->bro->xyz)-bond;
%	if (d*d>give) { part2cells(a,a->bro,bond,kick); return; }
%	d = vdif(b->xyz,b->sis->xyz)-bond;
%	if (d*d>give) { part2cells(b,b->sis,bond,kick); return; }
%	d = vdif(b->xyz,b->bro->xyz)-bond;
%	if (d*d>give) { part2cells(b,b->bro,bond,kick); return; }
%	d = vdif(c->xyz,c->sis->xyz)-bond;
%	if (d*d>give) { part2cells(c,c->sis,bond,kick); return; }
%	d = vdif(c->xyz,c->bro->xyz)-bond;
%	if (d*d>give) { part2cells(c,c->bro,bond,kick); return; }
%	d = vdif(a->sis->xyz,a->bro->xyz)-span;
%	if (d*d>give) { part2cells(a->sis,a->bro,span,kick); return; }
%	d = vdif(b->sis->xyz,b->bro->xyz)-span;
%	if (d*d>give) { part2cells(b->sis,b->bro,span,kick); return; }
%	d = vdif(c->sis->xyz,c->bro->xyz)-span;
%	if (d*d>give) { part2cells(c->sis,c->bro,span,kick); return; }
%	d = vdif(a->xyz,b->xyz)-link;
%	if (d*d>give) { part2cells(a,b,link,kick); return; }
%	d = vdif(c->xyz,b->xyz)-link;
%	if (d*d>give) { part2cells(c,b,link,kick); return; }
%	// fill working array (s) with parallel strands (s[strand][residue])
%	s[0][0]=a->sis->xyz; s[0][1]=a->xyz; s[0][2]=a->bro->xyz;
%	s[1][0]=b->sis->xyz; s[1][1]=b->xyz; s[1][2]=b->bro->xyz;
%	s[2][0]=c->sis->xyz; s[2][1]=c->xyz; s[2][2]=c->bro->xyz;
%	x = s[0][0]-s[0][2]; y = s[1][0]-s[1][2]; z = s[2][0]-s[2][2];
%	flip0 = flip2 = 0;
%	if (x*y < 0.0) { flip0 = 1;
%		s[0][0]=a->bro->xyz; s[0][1]=a->xyz; s[0][2]=a->sis->xyz;
%	}
%	if (z*y < 0.0) { flip2 = 1;
%		s[2][0]=c->bro->xyz; s[2][1]=c->xyz; s[2][2]=c->sis->xyz;
%	}
%	oldcent.zero();
%	for (i=0; i<3; i++) for (j=0; j<3; j++) oldcent += s[i][j];
%	oldcent /= 9.0;
%	// set average strand in A-B-C (outer strands have half weight each)
%	A = s[0][0] & s[2][0]; A = s[1][0]&A;
%	B = s[0][1] & s[2][1]; B = s[1][1]&B;
%	C = s[0][2] & s[2][2]; C = s[1][2]&C;
%	separate(A,B,bond,kick);
%	separate(C,B,bond,kick);
%	separate(A,C,span,kick);
%	up = B-(A&C);		// pleat direction 
%	x = A-C;		// chain direction along average strand
%	to = s[2][1]-s[0][1];	// mid strand to strand direction
%	y = to^x;
%	if (y*up < 0.0) y = -y; // keep Y pointing up
%	z = x^y;
%	if (z*to < 0.0) z = -z; // keep Z pointing as given
%	// the xyz frame has x=strand(2-->0), y=up(pleat), z=across(0-->2)
%	x.setVec(span*0.5);
%	y.setVec(drop);
%	z.setVec(link);
%	// set A-B-C as B+/-x with pleat y
%	A = B+x; A -= y;
%	C = B-x; C -= y;
%	// set new positions as average +/- z
%	s[0][0]=A-z; s[0][1]=B-z; s[0][2]=C-z;
%	s[1][0]=A;   s[1][1]=B;   s[1][2]=C;
%	s[2][0]=A+z; s[2][1]=B+z; s[2][2]=C+z;
%	// rotate corners to twist
%	s[0][0].set_rot(s[2][1],s[0][1], twist);
%	s[0][2].set_rot(s[2][1],s[0][1], twist);
%	s[2][0].set_rot(s[2][1],s[0][1],-twist);
%	s[2][2].set_rot(s[2][1],s[0][1],-twist);
%	newcent.zero();
%	for (i=0; i<3; i++) for (j=0; j<3; j++) newcent += s[i][j];
%	newcent /= 9.0;
%	told = oldcent-newcent;	// restore old centre
%	for (i=0; i<3; i++) for (j=0; j<3; j++) s[i][j] += told;
%	// update sheet coordinates (swap if flipped)
%	if (flip0) { x=s[0][0]; s[0][0]=s[0][2]; s[0][2]=x; }
%	if (flip2) { x=s[2][0]; s[2][0]=s[2][2]; s[2][2]=x; }
%	a->sis->xyz += (s[0][0] - a->sis->xyz)*wa*fix;
%	a->xyz      += (s[0][1] - a->xyz     )*wa;
%	a->bro->xyz += (s[0][2] - a->bro->xyz)*wa*fix;
%	b->sis->xyz += (s[1][0] - b->sis->xyz)*wb*fix;
%	b->xyz      += (s[1][1] - b->xyz     )*wb;
%	b->bro->xyz += (s[1][2] - b->bro->xyz)*wb*fix;
%	c->sis->xyz += (s[2][0] - c->sis->xyz)*wc*fix;
%	c->xyz      += (s[2][1] - c->xyz     )*wc;
%	c->bro->xyz += (s[2][2] - c->bro->xyz)*wc*fix;
%}
%
%/*
%#include "main/common.h"
%
%char	num[11];
%
%void linker ( int *datain, Cells *worldin, Types*** typesin )
%{
%int	i,j, m, n=0;
%char	bondid[N+1];
%float	*bondlen[2];
%char	**bondstr[2];
%float	distBP = 3.0;
%//	bondxxx[0|1]: 0=prot, 1=RNA
%	data = datain; world = worldin; types = typesin;
%	if (data[0] > 0 && data[5] == 999 ) return; // freeze during presort
%	for (i=0; i<2; i++) {
%		bondlen[i] = (float*)alloca(sizeof(float)*22);
%		bondstr[i] = (char**)alloca(sizeof(char*)*N);
%		for (j=0; j<N; j++) bondstr[i][j] = (char*)alloca(sizeof(char*)*111);
%	}
%	depth = data[2];
%	strcpy(num,"0123456789");
%	//bondCA = 0.1*(float)(sizes[depth]+bonds[depth]);
%	// Protein
%	strcpy(bondstr[0][depth-1],"10-11-12-13-14-15-20-21-22-23-");  // SSE links
%	// bondlen[0] match          0  1  2  3  4  5  6  7  8  9    to bondstr key
%	strcpy(bondstr[0][depth],"100-101-200-201-202-203-");	// residue links
%	// bondlen[0] match        10  11  12  13  14  15            to bondstr key
%	// RNA
%	strcpy(bondstr[1][depth],"100-101-200-201-202-203-");	// base pair links
%	// bondlen[0] match        16  17  18  19  20  21            to bondstr key
%	// Protein
%	// SSE link lengths
%	bondlen[0][0] = 1.7; bondlen[0][1] = 2.0;		// alpha SSE level
%	bondlen[0][2] = 2.3; bondlen[0][3] = 2.6;		// alpha SSE level
%	bondlen[0][4] = 2.9; bondlen[0][5] = 3.2;		// alpha SSE level
%	bondlen[0][6] = bondlen[0][7] = -bondCA*4.7/3.8;	// beta  SSE level (-ve = only bumps)
%	bondlen[0][8] = bondlen[0][9] = -bondCA*4.7/3.8;	// beta  SSE level (-ve = only bumps)
%	// residue link length
%	bondlen[0][10] = bondCA*5.1/3.8; 			// alpha CA (i->i+3)
%	bondlen[0][11] = bondCA*6.2/3.8;			// alpha CA (i->i+4)
%	bondlen[0][12] = bondlen[0][13] = bondCA*6.9/3.8;	// intra beta CA (i->i+/-2)
%	bondlen[0][14] = bondlen[0][15] = bondCA*4.7/3.8;	// inter beta CA level
%	// RNA
%	// base-pair link length
%	distBP *= bondPP;
%	bondlen[1][10] = distBP;			// RNA ladder
%	bondlen[1][11] = distBP;			// 
%	linkCell(world,0,bondid,bondlen,bondstr);
%	bondCell(world,0,bondid,bondlen,bondstr);
%}
%
%linkCell ( Cells *cell, int level, char *bondid, float **bondlenin, char ***bondstrin )
%{
%float	d;
%Vec	shift;
%int	i, j, n, m, id;
%Cells	*a, *b;
%float	size, bond, basic, kick, len;
%float	*bondlen;
%char	**bondstr;
%int	nbonds;
%	if (cell==0) return;
%	if (cell->empty) return;
%//	if (level && cell->parent->solid > 0) return; 
%	id = cell->id;
%	n = cell->kids;
%	m = cell->model*M*N+N;
%	moltype = data[m];
%	bondlen = bondlenin[moltype];
%	bondstr = bondstrin[moltype];
%	align = data+m+N*1; class = data+m+N*2;
%        sizes = data+m+N*3; bumps = data+m+N*4; links = data+m+N*5; chain = data+m+N*6;
%	kicks = data+m+N*7; keeps = data+m+N*8; repel = data+m+N*9; bonds = data+m+N*10;
%	split = data+m+N*11; local = data+m+N*12; // flags local or generic bond length
%	kick = 0.2; // WAS 0.1;
%	size = 0.1*(float)(sizes[level]+bonds[level]);
%	bondid[level] = num[cell->sort];
%	basic = 0.1*(float)bumps[level]; // default link replaced later from types (if set)
%	if (cell->link) { float d, error = 0.0;
%		bondid[level+2] = '\0';
%		for (i=0; i<links[level]; i++) { char *bids; int bid;
%			if (cell->link[i]==0) continue;
%			if (cell->link[i]->empty) continue;
%			d = vdif(cell->xyz,cell->link[i]->xyz);
%			if (d>30.0) { // break a very long link
%				cell->link[i] = 0;
%				continue;
%			}
%			bond = basic;
%			bondid[level+1] = num[i];
%			bondid[level+2] = '-';
%			bondid[level+3] = '\0';
%			bids = strstr(bondstr[level],bondid+depth-1);
%		       	if (bids) {
%				bid = (int)(bids-bondstr[level]);
%				if (level==depth) bid = bid/4 + 10; else bid = bid/3;
%				bond = bondlen[bid];
%				if (bond < 0.0) {	// just check bumps
%					if (vdif(cell->xyz,cell->link[i]->xyz) > bond) continue;
%					bond = -bond;
%				}
%			} 
%			if (cell->type==2 && cell->sort==0) { // don't refine loops much
%				if (cell->link[i]==0) break;
%				d = vdif(cell->xyz,cell->link[i]->xyz);
%				if (d > bond*1.5) {
%					part2cells(cell, cell->link[i], bond, kick, 1);
%				}
%				if (d < bond*0.5) {
%					part2cells(cell, cell->link[i], bond, kick, 1);
%				}
%				continue;
%			}
%			if (cell->link[i]==0) break;
%			//  stops reactions? bond += (float)(depth-level);
%			part2cells(cell, cell->link[i], bond, kick, 0); // don't use knockon = 1);
%			if (cell->type == 2)	// keep space between SSE ends (-kick = repel only)
%			{ float dnn = vdif(cell->endN,cell->link[i]->endN),
%				dnc = vdif(cell->endN,cell->link[i]->endC),
%				dcn = vdif(cell->endC,cell->link[i]->endN),
%				dcc = vdif(cell->endC,cell->link[i]->endC);
%				if (dnn+dcc < dnc+dcn) {	// parallel 
%					separate(&(cell->endN),&(cell->link[i]->endN),bond,-kick*0.5);
%					separate(&(cell->endC),&(cell->link[i]->endC),bond,-kick*0.5);
%				} else {			// antiparr
%					separate(&(cell->endN),&(cell->link[i]->endC),bond,-kick*0.5);
%					separate(&(cell->endC),&(cell->link[i]->endN),bond,-kick*0.5);
%				}
%			}
%		}
%		bondid[level+1] = '\0';
%		// type 2 = SSE tube + sort 2 = beta and links 2 and 3 are between strands
%		if (level && cell->parent->type==2 && cell->parent->sort==2) {
%			fixSheet(cell->link[2], cell, cell->link[3]);
%		}
%	}
%	for (i=0; i<n; i++) { //float far = -cell->far/(float)data[3];
%		linkCell(cell->child[i],level+1,bondid,bondlenin,bondstrin);
%	}
%}
%
%bondCell ( Cells *cell, int level, char *bondid, float **bondlenin, char ***bondstrin )
%{
%float	d;
%Vec	shift;
%int	i, j, n, m, id;
%Cells	*a, *b;
%float	size, bond, basic, kick, len;
%float	*bondlen;
%char	**bondstr;
%int	nbonds;
%	if (cell==0) return;
%	if (cell->empty) return;
%	id = cell->id;
%	n = cell->kids;
%//	if (level && cell->parent->solid > 0) return; 
%	m = cell->model*M*N+N;
%	moltype = data[m];
%	align = data+m+N*1; class = data+m+N*2;
%        sizes = data+m+N*3; bumps = data+m+N*4; links = data+m+N*5; chain = data+m+N*6;
%	kicks = data+m+N*7; keeps = data+m+N*8; repel = data+m+N*9; bonds = data+m+N*10;
%	split = data+m+N*11; local = data+m+N*12; // flags local or generic bond length
%	len = size = 0.1*(float)(sizes[level]+bonds[level]);
%	kick = 0.3;
%	DO(j,1) {
%		if (moltype < 2) {
%                	if (cell->bond && cell->bond[0].type ) { // use HINGE data for protein or nucleic
%                        	for (i=0; i<5; i++) { Bonds *hinge = cell->bond+i;
%                                	if (hinge->to == 0) break;
%                                	fixHinge(cell,hinge);
%                        	}
%				break;
%			}
%                }
%		if (moltype==0) { // PROTein chain
%			if (local[level]) { // using preset lengths (set by -ve size in param.model
%				len = cell->prox.x;	// use specific sis length (x)
%			}
%			if (chain[level]==0 || (cell->sis->level!=cell->level)) break;
%			if (len > 999.0) len = size;  // use generic bond length
%			part2cells(cell,cell->sis,len,kick,0);
%			break;
%		}
%		if (moltype==1) { // nucleic
%			if (chain[level]==0 || (cell->sis->level!=cell->level)) break;
%			if (level==depth) { // atomic
%				part2cells(cell,cell->sis,size,kick,0);
%			}
%			if (level==depth-1 && (int)cell->endC.z!=1234 && (int)cell->endN.z!=1234   // z = 1234
%			         && (int)cell->sis->endC.z!=1234 && (int)cell->sis->endN.z!=1234 ) // for unset
%			{ char	stemcel, stemsis; // SSE, endC is near termini
%			  float gap, disp; Vec move;
%				if (cell->sort==1) stemcel = 1; else stemcel = 0;
%				if (cell->sis->sort==1) stemsis = 1; else stemsis = 0;
%				if ( stemcel &&  stemsis) vsub(cell->endC,cell->sis->endC, &move);
%				if ( stemcel && !stemsis) vsub(cell->endC,cell->sis->endC, &move);
%				if (!stemcel &&  stemsis) vsub(cell->endN,cell->sis->endC, &move);
%				if (!stemcel && !stemsis) vsub(cell->endC,cell->sis->endC, &move);
%				gap = vmod(move); disp = size-gap;
%				vnorm(&move); vmul(&move,disp*kick*0.5);
%				moveCell(cell->sis,move,-1);
%				moveCell(cell,move, 1);
%			}
%			if (level<depth-1) { // other (as polymer)
%				part2cells(cell,cell->sis,size,kick,0);
%			}
%			break;
%		}
%		if (moltype==2) { // CHEMical bonds
%			nbonds = cell->nbonds;
%			for (i=0; i<nbonds; i++) {
%				if (cell->bond[i].to == 0) continue; // cyclic bonds at end
%				part2cells(cell,cell->bond[i].to,size,kick,0);
%			}
%			break;
%		}
%	}
%	for (i=0; i<n; i++) { // float far = -cell->far/(float)data[3];
%		bondCell(cell->child[i],level+1,bondid,bondlenin,bondstrin);
%	}
%}
%
%rdisp ( Vec a, Vec b, Vec *c, float r )
%{ // add a random displacement (r) to b, keeping fixed length a-b
%Vec	e, f;
%float	d = vdif(a,b);
%	vcopy(b,&e);
%	vradd(&e,r);
%	vsub(e,a,&f);
%	vnorm(&f);
%	vmul(&f,d);
%	vadd(a,f,c);
%}
%
%float scoreFit ( Vec a, Vec b, Vec c, Vec d, Vec e, Vec f, float leng) 
%{
%float	ef,be,cf, dif, ta,td, score, wgap=100.0, woff=10.0, wang=1.0;
%	be = vdif(b,e); cf = vdif(c,f); // keep close to old positions
%	ta = PI - angle(a,e,f); // bias to towards a
%	td = PI - angle(e,f,d); // convex connection
%	ef = vdif(e,f);
%	dif = ef-leng; dif *= dif;	// keep to ideal length
%	score = dif*wgap+(be+cf)*woff+(ta+td)*wang;
%//if (ef>0.0) { Pr(ef) Pr(dif) Pr(be) Pr(cf) Pr(ta) Pr(td) Pr(score) Pr(log(score)) NL } 
%	return score;
%}
%
%#define TRY 10
%
%fixHinge (Cells *at, Bonds *bond)
%{
%Vec	bs[TRY], cs[TRY];
%Vec	a,b,c,d,e,f,g,h, u,v,w, x,y,z, beste, bestf;
%int	i,j, m,n,in, link;
%float	leng, ab,bc,cd,ad, ra,rd, pa,pd, qa,qd, p,q;
%float	ef, be, cf, dif, score, best, t,ta,td, max = 1.0;
%Cells	*to;
%	to = bond->to;
%	link = bond->next;
%	leng = 0.1*(float)bond->type; 
%	if (link==1) { vcopy(at->endC,&a); vcopy(at->endN,&b); vcopy(to->endN,&c); vcopy(to->endC,&d); } // NN
%	if (link==2) { vcopy(at->endC,&a); vcopy(at->endN,&b); vcopy(to->endC,&c); vcopy(to->endN,&d); } // NC
%	if (link==3) { vcopy(at->endN,&a); vcopy(at->endC,&b); vcopy(to->endN,&c); vcopy(to->endC,&d); } // CN
%	if (link==4) { vcopy(at->endN,&a); vcopy(at->endC,&b); vcopy(to->endC,&c); vcopy(to->endN,&d); } // CN
%	ab = vdif(a,b); bc = vdif(b,c); cd = vdif(c,d); ad = vdif(a,d);
%	if (ad > ab+leng+cd) { float dist; // make colinear
%		ta = angle(b,a,d);
%		if (ta > NOISE) {
%			if (ta>max) ta = max;
%			vsub(b,a,&x); vnorm(&x);
%			vsub(d,a,&y); vnorm(&y);
%			vprod(x,y,&z); vadd(a,z,&g);
%			spinCell(at,a,g,-ta);
%		}
%		td = angle(c,d,a);
%		if (td>NOISE) {
%			if (td>max) td = max;
%			vsub(c,d,&x); vnorm(&x);
%			vsub(a,d,&y); vnorm(&y);
%			vprod(x,y,&z); vadd(d,z,&g);
%			spinCell(to,d,g,-td);
%		}
%		dist = leng+(ab+cd)*0.5; // half cell separation
%		part2cells(at,to,dist,0.5,0);
%		return;
%	}
%	t =  leng-bc; if (t<0.0) t = -t;
%	t = sqrt(t);
%	if (t<0.01) return;
%	ta = t*sqrt(ab)*1.0;
%	td = t*sqrt(cd)*1.0;
%	m = n = 1;
%	vcopy(b,bs); vcopy(c,cs); // 1st try (bs[0],cs[0]) = current position
%	best = scoreFit(a,b,c,d,b,c,leng);
%	DO(i,1000) { float len2 = leng*2.0;;
%		if (n==TRY && m==TRY) break;
%		rdisp(a,bs[n-1],&e,ta);
%		rdisp(d,cs[m-1],&f,td);
%		score = scoreFit(a,b,c,d,e,f,leng);
%		if (score > best) continue;
%		in = 0;
%		if (n<TRY && vdif(e,d)>cd) { // e is outside D
%			vcopy(e,bs+n);
%			n++; in++;
%		}
%		if (m<TRY && vdif(f,a)>ab) { // f is outside A
%			vcopy(f,cs+m);
%			m++; in++;
%		}
%		if (in==2) best = score;
%	}
%	vcopy(b,&beste); vcopy(c,&bestf);
%	DO(i,n) { DO(j,m) {
%		score = scoreFit(a,b,c,d,bs[i],cs[j],leng);
%		if (score<best) {
%			best = score; vcopy(bs[i],&beste); vcopy(cs[j],&bestf);
%		}
%	}	}
%	if (vdif(b,beste) > 0.001) { // spin cell at
%		ta = angle(b,a,beste);
%		if (ta > NOISE) {
%			if (ta>max) ta = max;
%			vsub(b,a,&x); vnorm(&x);
%			vsub(beste,a,&y); vnorm(&y);
%			vprod(x,y,&z); vadd(a,z,&g);
%			spinCell(at,a,g,-ta);
%		}
%	}
%	if (vdif(c,bestf) > 0.001) { // spin cell to
%		td = angle(c,d,bestf);
%		if (td>NOISE) {
%			if (td>max) td = max;
%			vsub(c,d,&x); vnorm(&x);
%			vsub(bestf,d,&y); vnorm(&y);
%			vprod(x,y,&z); vadd(d,z,&g);
%			spinCell(to,d,g,-td);
%		}
%	}
%} 
%
%fixSheet (Cells *a, Cells *b, Cells *c)
%{
%Vec	s[3][3], oldcent, newcent;
%Vec	A,B,C, x,y,z, mid, up, to, shift, told;
%float	size = 0.1*(float)(bonds[b->level]+sizes[b->level]);
%float	bond = 3.8, span = 6.8, link = 4.7, drop = 1.5;
%float	give = size*0.1, kick = 0.2, fix = 0.5, d;
%int	i, j, flip0, flip2, inbeta;
%float	twist = 0.3;
%int	level = b->level, beta = 2;
%	span = size*span/bond;
%	link = size*link/bond;
%	drop = size*drop/bond;
%	bond = size;
%	kick = 0.1; fix = 0.01;
%	// check all exist, are on the same level and most in beta
%	if (!a || !b || !c) return;
%	if (a->parent->sort != beta || b->parent->sort != beta || c->parent->sort != beta) return;
%	if (a->sis->level != level) return; if (a->bro->level != level) return;
%	if (b->sis->level != level) return; if (b->bro->level != level) return;
%	if (c->sis->level != level) return; if (c->bro->level != level) return;
%	inbeta = 0;
%	if (a->sis->parent->sort == beta) inbeta++; if (a->bro->parent->sort == beta) inbeta++;
%	if (b->sis->parent->sort == beta) inbeta++; if (b->bro->parent->sort == beta) inbeta++;
%	if (c->sis->parent->sort == beta) inbeta++; if (c->bro->parent->sort == beta) inbeta++;
%	if (inbeta < 5) return;
%	// check atoms are far enough apart
%	if (abs(a->atom - c->atom) < 6) return; 
%	if (abs(a->atom - b->atom) < 4) return; 
%	if (abs(b->atom - c->atom) < 4) return; 
%	// check all lengths are reasonable before refining geometry
%	d = vdif(a->xyz,a->sis->xyz)-bond;
%	if (d*d>give) { separate(&(a->xyz),&(a->sis->xyz),bond,kick); return; }
%	d = vdif(a->xyz,a->bro->xyz)-bond;
%	if (d*d>give) { separate(&(a->xyz),&(a->bro->xyz),bond,kick); return; }
%	d = vdif(b->xyz,b->sis->xyz)-bond;
%	if (d*d>give) { separate(&(b->xyz),&(b->sis->xyz),bond,kick); return; }
%	d = vdif(b->xyz,b->bro->xyz)-bond;
%	if (d*d>give) { separate(&(b->xyz),&(b->bro->xyz),bond,kick); return; }
%	d = vdif(c->xyz,c->sis->xyz)-bond;
%	if (d*d>give) { separate(&(c->xyz),&(c->sis->xyz),bond,kick); return; }
%	d = vdif(c->xyz,c->bro->xyz)-bond;
%	if (d*d>give) { separate(&(c->xyz),&(c->bro->xyz),bond,kick); return; }
%	d = vdif(a->sis->xyz,a->bro->xyz)-span;
%	if (d*d>give) { separate(&(a->sis->xyz),&(a->bro->xyz),span,kick); return; }
%	d = vdif(b->sis->xyz,b->bro->xyz)-span;
%	if (d*d>give) { separate(&(b->sis->xyz),&(b->bro->xyz),span,kick); return; }
%	d = vdif(c->sis->xyz,c->bro->xyz)-span;
%	if (d*d>give) { separate(&(c->sis->xyz),&(c->bro->xyz),span,kick); return; }
%	d = vdif(a->xyz,b->xyz)-link;
%	if (d*d>give) { separate(&(a->xyz),&(b->xyz),link,kick); return; }
%	d = vdif(c->xyz,b->xyz)-link;
%	if (d*d>give) { separate(&(c->xyz),&(b->xyz),link,kick); return; }
%	// fill working array (s) with parallel strands
%	vcopy(a->sis->xyz,&(s[0][0])); vcopy(a->xyz,&(s[0][1])); vcopy(a->bro->xyz, &(s[0][2]));
%	vcopy(b->sis->xyz,&(s[1][0])); vcopy(b->xyz,&(s[1][1])); vcopy(b->bro->xyz, &(s[1][2]));
%	vcopy(c->sis->xyz,&(s[2][0])); vcopy(c->xyz,&(s[2][1])); vcopy(c->bro->xyz, &(s[2][2]));
%	vsub(s[0][0],s[0][2], &x); vsub(s[1][0],s[1][2], &y); vsub(s[2][0],s[2][2], &z);
%	flip0 = flip2 = 0;
%	if (vdot(x,y) < 0.0) {
%		vcopy(a->bro->xyz,&(s[0][0])); vcopy(a->xyz,&(s[0][1])); vcopy(a->sis->xyz, &(s[0][2]));
%		flip0 = 1;
%	}
%	if (vdot(z,y) < 0.0) {
%		vcopy(c->bro->xyz,&(s[2][0])); vcopy(c->xyz,&(s[2][1])); vcopy(c->sis->xyz, &(s[2][2]));
%		flip2 = 1;
%	}
%	vinit(&oldcent);
%	for (i=0; i<3; i++) for (j=0; j<3; j++) vsum(s[i][j], &oldcent);
%	vdiv(&oldcent, 9.0);
%	// set average strand in A-B-C (outer strands half weight)
%	vave(s[0][0], s[2][0], &A); vave(s[1][0], A, &A);
%	vave(s[0][1], s[2][1], &B); vave(s[1][1], B, &B);
%	vave(s[0][2], s[2][2], &C); vave(s[1][2], C, &C);
%	separate(&A,&B,bond,kick);
%	separate(&C,&B,bond,kick);
%	separate(&A,&C,span,kick);
%	vave(A,C, &mid);
%	vsub(B,mid, &up);		// pleat direction 
%	vsub(A, C, &x);			// chain direction
%	vsub(s[2][1], s[0][1], &to);	// strand to strand direction
%	vprod(to,x, &y);
%	if (vdot(y,up) < 0.0) vmul(&y,-1.0); // keep Y pointing up
%	vprod(x,y, &z);
%	if (vdot(z,to) < 0.0) vmul(&z,-1.0); // keep Z pointing as given
%	// the xyz frame has x=strand, y=up(pleat), Z=2-->0
%	vnorm(&x); vmul(&x,span*0.5);
%	vnorm(&y); vmul(&y,drop);
%	vnorm(&z); vmul(&z,link);	// set new positions as average +/- z
%	vadd(B,x,&A); vsub(A,y,&A);
%	vsub(B,x,&C); vsub(C,y,&C);
%	vsub(A,z,&(s[0][0])); vsub(B,z,&(s[0][1])); vsub(C,z,&(s[0][2]));
%	vcopy(A, &(s[1][0])); vcopy(B, &(s[1][1])); vcopy(C, &(s[1][2]));
%	vadd(A,z,&(s[2][0])); vadd(B,z,&(s[2][1])); vadd(C,z,&(s[2][2]));
%	// rotate corners to twist
%	rotate(s[2][1],s[0][1],&(s[0][0]), twist);
%	rotate(s[2][1],s[0][1],&(s[0][2]), twist);
%	rotate(s[2][1],s[0][1],&(s[2][0]),-twist);
%	rotate(s[2][1],s[0][1],&(s[2][2]),-twist);
%	vinit(&newcent);
%	for (i=0; i<3; i++) for (j=0; j<3; j++) vsum(s[i][j], &newcent);
%	vdiv(&newcent, 9.0);
%	vsub(oldcent,newcent, &told);	// restore old centre
%	for (i=0; i<3; i++) for (j=0; j<3; j++) vsum(told, &(s[i][j]));
%	// update sheet coordinates (swap if flipped)
%	if (flip0) { vcopy(s[0][0],&x); vcopy(s[0][2],&(s[0][0])), vcopy(x,&(s[0][2])); }
%	if (flip2) { vcopy(s[2][0],&x); vcopy(s[2][2],&(s[2][0])), vcopy(x,&(s[2][2])); }
%	vsub(s[0][0], a->sis->xyz, &shift); vmul(&shift,fix); vsum(shift, &(a->sis->xyz));
%	vsub(s[0][1], a->xyz,      &shift); vmul(&shift,fix); vsum(shift, &(a->xyz));
%	vsub(s[0][2], a->bro->xyz, &shift); vmul(&shift,fix); vsum(shift, &(a->bro->xyz));
%	vsub(s[1][0], b->sis->xyz, &shift); vmul(&shift,fix); vsum(shift, &(b->sis->xyz));
%	vsub(s[1][1], b->xyz,      &shift); vmul(&shift,fix); vsum(shift, &(b->xyz));
%	vsub(s[1][2], b->bro->xyz, &shift); vmul(&shift,fix); vsum(shift, &(b->bro->xyz));
%	vsub(s[2][0], c->sis->xyz, &shift); vmul(&shift,fix); vsum(shift, &(c->sis->xyz));
%	vsub(s[2][1], c->xyz,      &shift); vmul(&shift,fix); vsum(shift, &(c->xyz));
%	vsub(s[2][2], c->bro->xyz, &shift); vmul(&shift,fix); vsum(shift, &(c->bro->xyz));
%}
%*/
