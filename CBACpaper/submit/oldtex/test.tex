\section{Test data}

\subsection{Colliding Hilbert chains}

To retain a connection to macromolecules, idealised test data was based on a linear chain of atoms.
This chain was then 'packaged' into a hierarchy of spherical objects, each of which contained
eight children arranged as a cube.   To maintain equal bond-lengths between the atoms, the path of the
chain followed a recursive Hilbert curve, in which each level of the hierachry is identical\footnote{
A Hilbert curve ({\tt https://en.wikipedia.org/wiki/Hilbert\_curve}) is the spatial equivalent of a 
Gray code ({\tt https://en.wikipedia.org/wiki/Gray\_code}) in which successive elements are
only one step away from their neighbours.
}.   This arrangement generates homogeneously packed chains which allows the effects of collisions
to be monitored as displacements get propogated through the structure.

Collisions between these objects were then engineered by applying an external
motion to propel two identical objects into each other.   To avoide a direct "head-on"
collision, the objects were displaced by half their radius from their line of approach.
As the chain forms a cube, this means that the collision surface encompases
half a face of each cube.
During the collision, the number of bumping atoms was monitored between the two objects
and also within each object.

\subsubsection{8+8 crash}

To investigate the the contribution of the soft repulsion component of higher level objects,
the first construct of interest involves the collision of two cubes of eight atoms.  The
repulsion strength at both levels was set to a value of 1 for both hard and soft modes
(although the atomic level only has hard repulsion).   With these values, the containing
spherical shells repelled each other before the atoms could make contact.

The $soft$ parameter value was then decreased, allowing interpenetration of the high
level spheres,  until the internal atoms made contact.   This occured when $soft = hard/5$
but most of the displacement still derived from the high-level soft repulsion component
and the internal arrangement of the atoms was almost unchanged.   With no soft repulsion,
the atomic configuration was markedly displaced and as a comprimise, $soft = hard/10$
was taken as a combination that allowed a contribution from each level.

\subsubsection{64+64 crash}

The next level of model considered was the 64 atom chain and keeping the values established
above for the first level in the hierarchy, an equivalment evaluation was made for the second level.
As with the smaller test object, the first level spheres initally made contact when $soft = hard/5$,
however, because of the added buffering effect of the first-level spheres, the atoms remained
well separated between the two colliding objects even when $soft = hard/10$.  To compensate for
this additive contribution, the $hard$ parameter value on both levels was halved and the 10\%
ratio to the $soft$ parameter retained giving $hard = 0.5, soft = 0.05$. 

\subsubsection{512+512 crash}

The evaluation protocol was extended to the 3-level hierarch of 512 atoms per chain.
Transfering the values from the previous test again led to a lack of direct contact at the
atomic level and these were reduced to $hard = 0.2$, keeping the $soft = hard/10$ ratio.
This produced a result at the mid-point of the collision (when the centroids of the bodies
draw level on their collision course) that was comparable to the smaller systems.

\subsubsection{4096+4096 crash}

For the largest model tested, with the chain packaged into 4 levels of containers,  the
progression of reducing the values of the $hard$ and $soft$ parameters was continued.
However, this led to a marked number (100s) of steric violations at the atomic level both
between and within the colliding bodies.   Keeping the two parameters at their previous
levels ($hard = 0.2$, $soft = 0.02$) the number of clashes between the bodies decreased
(10s) but the number of internal violations remained high.

As this model extends beyond the normal size range of compact biological molecules,
no further experimentaion was made.   Of greater interest is the degree of distortion 
observed in the "crumple-zone" and to investigate this a more realistic protein model
was used.


\subsection{Colliding multiple protein domains}

To progress to a more biologically realistic system and also to introduce a variety
of container shapes, the small chemotaxis-Y protein (PDB code: {\tt 3chy}), was used
and modelled with tubes to contain its secondary structure elements and an ellipsoid
to contain the protein.    The structure was also stabilised with 'hydrogen-bond'-like
links between the $i..i+3$ and $i..i+4$ positions in the \AHs\ and links between
hydrogen-bonded positions in the \BS.

A series of multi-domain models were then constructed with the protein chain as a
node on a Hilbert curve giving models of 1, 8 and 64 domains.   The link between 
domains was also optionally broken, making the domains equivalent to subunits.
Collisions between these constructs was engineered as above with the coordinates
being saved at the start of the run and at the end, after the structures were well
past each other.    To monitor the distortion experienced by the domains/subunits
during the collision each domain in the starting structure was comapred to each
domain in the final structure and the smallest, mean and largest root-mean-square
(RMS) deviations recorded. 

\subsubsection{Single domain collision}

Two single domain structures were collided using the same parameters as determined
for the equivalent sized "cubic" chains of the previous section ($hard = 1.0$, $soft = 0.1$).
Comparing combinations of the two starting and two final strutctures, the mean RMS
deviation was 3.6\AA , which is not far in excess of the 2.4\AA\ mean deviation seen
when the two structures travel the same distance but do not collide.   For a protein
of this size, with any RMS deviation under 5\AA, the stuctures retain a clear correspondance. 

Increasing the $soft$ parameter to 0.2 led to less distortion (mean RMS = 3.1) which
is closer to the un-collided value.   While, retaining the same $hard:soft$ ratio
with $hard = 0.5$ gave a deviation of 4.0\AA, which is still acceptable and only
when the parameter values were dropped as low as $hard = 0.5$, $soft = 0.01$ was the
5\AA\ 'threshhold' exceded.  

\subsubsection{8+8 domain collision}

The same approach to the larger 8-domain constructs over a series of collisions 
ranging from a glancing blow to almost head-on collision.  The tests were repeated
both with the domains in a continuous chain and as separate (unbonded) subunits).
For two of the $hard$/$soft$ parameter combinations examined above, all mean RMS
devaiations (over 64 start/final domain combinations) remained below 5\AA, however,
some of the worst individual deviations were over 5 but almost always when the
domains were connected in a chain.   Despite excluding the 5 amino and 5 carboxy
terminal linking residues from the comparison, examination of these worst cases
revealled that most of the deviation came from displacement of the untethered C-terminal
helix as a result of shear forces during the collision.

\begin{verbatim}
----------------------------------------------------------------------------------
        0,    50,   50,   50,  100,  // hard  :kick size = []/100
        0,    20,   20,   20,  100,  // soft  :kick size = []/100

subunit

miss	min = 1.954 ave = 2.50335 max = 3.183
clip	min = 2.082 ave = 2.72707 max = 3.441
half	min = 2.338 ave = 3.24126 max = 4.075
full	min = 2.959 ave = 3.94160 max = 5.076

domain

miss	min = 2.025 ave = 3.15143 max = 4.868
clip	min = 2.298 ave = 3.44083 max = 5.519
half	min = 2.243 ave = 4.08091 max = 6.746
full	min = 2.947 ave = 4.82165 max = 8.467


----------------------------------------------------------------------------------
        0,   100,  100,  100,  100,  // hard  :kick size = []/100
        0,    20,   20,   20,  100,  // soft  :kick size = []/100

subunit

miss	min = 1.848 ave = 2.57102 max = 3.307
clip	min = 2.207 ave = 2.70812 max = 3.240
half	min = 2.261 ave = 3.09951 max = 4.096
full	min = 2.840 ave = 3.66765 max = 4.670

domain

miss	min = 1.850 ave = 3.14264 max = 4.838
clip	min = 2.375 ave = 3.50274 max = 5.259
half	min = 2.335 ave = 3.69418 max = 6.195
full	min = 2.572 ave = 4.16718 max = 6.271

----------------------------------------------------------------------------------
\end{verbatim}
