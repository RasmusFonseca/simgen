\subsection{Collision algorithms}

In the \NAME\ program, the
\TT{bumper} routine has the task of identifying objects that have approached closer
than permitted and repelling them by a fixed-size "kick" specified in the parameter file.
It must overcome two difficulties:  firstly, for large numbers of objects, it is
computationally too expensive to compare all-against-all and, secondly, the objects do not
all have simple shapes so the surface-surface distance between all combinations of
object shapes must be accommodated.

\subsubsection{Avoiding $N \times N$}

The \TT{bumper} routine uses the hierarchic structure of the data to avoid a $N^2$
order calculation between all pairs of objects.  Beginning at the highest level (1)
all objects at that level are compared pairwise but only if two objects at this level
are in collision, are their children then considered. 
As discussed in the Introduction, in computer-graphics terms,
the high-level objects are the bounding-volumes for their children. 

However, before the parents are repelled, a calculation is made of how many collisions
there are between their children.  These two distinct calculations of collisions
within a family and collisions between families are performed by {\tt bumpin()}
and {\tt bumpex()}, respectively.

\paragraph{{\tt \bfseries bumpin()}:\\}

If the number of children in a family is less than 20, {\tt bumpin()}
uses a simple pairwise algorithm (in {\tt getBumps()}) to provide a list of objects that
are potentially in collision.  With more children, then an approximate algorithm
is used that is based on the partially sorted X,Y,Z lists maintained internally in \NAME . (See
source code).  This selection is based on the largest dimension of the object:  the maximum
axis length for an ellipsoid or the larger of the length and diameter of a tube.  The list
is sorted by degree of violation so {\tt bumpin()} will deal with the worst cases first.

{\tt bumpin()} firstly checks the true separation of the two objects ({\tt a,b}) using {\tt touch()}
which returns a negative distance if the objects inter-penetrate. (See further sections
below for details of each type of interaction).   If {\tt a} and {\tt b} are not atoms, then
a count ({\tt m}) is made of how many collisions occur between their children using {\tt bumpex()}.
Two parameters,  $hard$ and $soft$, are specified in the input that set the repulsion step
for each type of object.  If both parameters have been given values,
then the degree of repulsion is calculated based on the number of colliding children, {\tt m},
as:  {\tt boot = f*soft + (1-f)*hard;} where {\tt f = exp(-m*m/100);}.     The resulting value
of {\tt boot} is then used by the utility {\tt part2cells()} to push the objects symmetrically
back towards a distance ({\tt d}) where they are no longer in collision. (See Box 1 for
pseudo code\footnote{
The "pseudo" code in the boxed segments is actually {\tt C++} code but employs a number of macros
as short-hands, most of these are obvious (like {\tt DO} and {\tt FOR}), however the following
less obvious vector operations are used:
"{a$|$b}" = the distance between points {\tt a} and {\tt b},
"{a*b}" = scalar (dot) product,
"{a\^b}" = vector (cross) product and
"{a\&b}" = their mid-point.
The {\tt Cell} data structure represents an object (at any level) and encodes the 
number of children ({\tt kids}) specified in the list: {\tt child[...]}.
A number of utilities operate on each object and most do what they say (eg;
{\tt moveCell()}, {\tt spinCell()}, {\tt part2Cells()}, etc.) and apply the transform
not only to the cell specified in the argument list, but recursively to all that it 
contains.}).

\begin{figure}[h]
\centering
\begin{singlespace}
\begin{tiny}
\begin{Verbatim}[frame=single]
int Cell::bumpin () {
// the children of the current cell <this> are checked for intra-family bumps
// <level> is the position in the hierarchy of the current object
// <depth> is the atomic (lowest) level
       :
       kidlev = level+1;
       :
       FOR(i,in) { Cell *a = list[i].a, *b = list[i].b;
               bump = touch(a,b); // -ve is bad
               if (bump > -NOISE) continue; // ignore touching objects
               if (kidlev<depth) m = bumpex(a,b); else m = 0;  // bumping a+b children parted in bumpex() 
               if (exempt(a,b)) continue;      // exempt parents (exempt atoms are skipped in getBumpin()) 
               // the pair (a,b) are bumping so repel with a kick using the number of bumping children (m)
               //      unless weight=0 then just use unmodified <soft> value
               if (weight) { // Gaussian switch from soft to hard with increasing <m>
                       d = (float)m; f = exp(-d*d*0.01);
                       boot = f*soft + (1.0-f)*hard;
                       boot *= kick;
               } else { boot = soft; }
               d = (a->xyz|b->xyz)-bump*over; // clash = -ve bump
               part2cells(a,b,d,-boot); // -kick = repel only
               :
       }
}
\end{Verbatim}
\end{tiny}
\end{singlespace}
\caption*{
Box 1:
\label{Fig:box1}
\begin{footnotesize}
{\bf Code for the {\ttfamily \bfseries bumpin()} routine} which repels colliding children within the current
object ({\tt Cell} structure).  The routine is called recursively over the tree of objects.   When two
children are in collision, their combined children are evaluated for collisions by {\tt bumpex()}
(see below and Box 2) which returns the number of collisions detected between the families.
\end{footnotesize}
}
\end{figure}

Both {\tt bumpin()} and {\tt getBumps()} employ a filter encoded in {\tt exempt()} that is
{\small \tt TRUE} if the two objects are exempt from collisions, for example, if they are bonded or linked.
In  {\tt bumpin()}, however, {\tt exempt()} is only called after  {\tt bumpex()} as the children
(and their offspring) inside two exempt objects might well be in collision.

\paragraph{{\tt \bfseries bumpex()}:\\}

The {\tt bumpex()} routine evaluates each pair of children between their two colliding parents
and returns the number of collisions.  As it does so, it also takes steps to rectify the
situation by separating the clashing children.  As it is known to which parent each child belongs,
they are also given a nudge back towards their parent before being separated. 
The strength of these kicks depend on both the $hard$ and $soft$ parameter values:
the nudge back home is always $soft$/10 while the separation is $hard$ at the atomic level
and $soft$ for higher levels.  (See Box 2).

\begin{figure}[h]
\centering
\begin{singlespace}
\begin{tiny}
\begin{Verbatim}[frame=single]
int bumpex ( Cell *a, Cell *b ) {
// the children of the cell <a> and <b> are checked for inter-family bumps
float   strength = 0.1;
        :
        kidlev = level+1;
        if (kidlev==depth) kick = hard; else kick = soft;
        axis = b->xyz - a->xyz;
        axis.setVec(soft);        // <soft> length vector from a to b (NB has to be set at atom level)
        :
        FOR(i,a->kids) { Cell* ai = a->child[i];
                FOR(j,b->kids) { Cell* bj = b->child[j];
                        if (exempt(ai,bj)) continue;
                        bump = touch(ai,bj);
                        if (bump > -NOISE) continue; // ignore touching objects
                        d = (ai->xyz|bj->xyz)-bump*over; // d = target gap (NB clash has -ve bump)
                        moveCell(ai,axis,-1);        // nudge ai towards a
                        moveCell(bj,axis, 1);        // nudge bj towards b
                        part2cells(ai,bj,d,-kick*strength); // -kick = repel only
                        :
                        bumpex(ai,bj);
                        n++;
                }
        }
        return n;
}
\end{Verbatim}
\end{tiny}
\end{singlespace}
\caption*{
Box 2:
\label{Fig:box2}
\begin{footnotesize}
{\bf Code for the {\ttfamily \bfseries bumpex()} routine} which repels colliding children between two objects ({\tt Cell}s).
The routine is called from {\tt bumpin()} and acts recursively on pairs of colliding children and their children.
\end{footnotesize}
}
\end{figure}

Once {{\tt bumpin()} has completed at one level, it is called recursively and continues to traverse 
the hierarchic tree of objects in a depth-first order.
The application of the {{\tt bumpin()}/{{\tt bumpex()} pair of routines is not recursive: ie:
{{\tt bumpex()} does not re-call {{\tt bumpin()} on colliding children.  However, {\tt bumpex()}
is itself recursive and is called on the children of any clashing children that it encounters. 

\subsubsection{Calculating contacts}

\NAME\ employes three object types, giving six possible types of encounter
which are dealt with by the {\tt touch()} routine (Box 3).

\begin{figure}[h]
\centering
\begin{singlespace}
\begin{tiny}
\begin{Verbatim}[frame=single]
float touch ( Cell *a, Cell *b )
{ // closest approach between two object surfaces
  // +ve = separation, -ve = penetration depth
       :
       rab = ra + rb;  // average of bump radii
       tab = ta * tb;  // product of object types
       if (tab>1) s = Seg(b->endN,b->endC);
       switch (tab) {
               case 1: // spheres = centre distance
                       return (pa|pb)-rab;
               case 2: // sphere+tube (closest approach to line segment or ends)
                       if (pa.vec_in_seg(s)) return pa.vec_to_line(s)-rab;
                       return fmin(pa|s.A,pa|s.B) - rab;
               case 3: // sphere+ellipsoid (in keeper.cpp)
                       return vec_to_egg(pa,s,db) - ra;
               case 4: // tubes = closest approach of 2 line segments
                       return seg_to_seg(Seg(a->endN,a->endC),Seg(b->endN,b->endC))-rab;
               case 6: // tube+ellipsoid (in bumper.cpp)
                       return tube_to_egg(a,b);
               case 9: // ellipsoid (in bumper.cpp)
                       return egg_to_egg(a,b);
       }
}
\end{Verbatim}
\end{tiny}
\end{singlespace}
\caption*{
Box 3:
\label{Fig:box3}
\begin{footnotesize}
{\bf Code for the {\ttfamily \bfseries touch()} routine} which returns the distance between the surfaces of two objects.
Each combination of the three object types: sphere, tube, ellipsoid (coded 1,2,3), are treated separately.
\end{footnotesize}
}
\end{figure}

\paragraph{Sphere/Sphere:\\}

A pair of spheres are the simplest case, with surface contact made at the average distance of their bumping diameters.
(Or the sum of the corresponding radii: $R_{ab} = R_a + R_b$, for two objects $a$ and $b$).

\paragraph{Sphere/Tube:\\}

For a sphere and a tube, the closest approach is the shortest sphere-centre to tube axis 
line-segment, less their joint radii ($R_{ab}$).   If a perpendicular construction from the 
sphere centre to the axis line lies within the tube end-points then this, less $R_{ab}$,
is the closest approach of their surfaces.  Otherwise, it is simply the shorter tube-end to
sphere distance (less $R_{ab}$).

\paragraph{Sphere/Ellipsoid:\\}

The distance of a point from the surface of a general (scalene) ellipsoid in not trivial,
however, as \NAME\ only deals with spheroids, a simple construction based on the foci of
their ellipse-of-rotation can be used to decide if a point is inside or outside the ellipsoid.
A path from one focus to any point on the ellipse and back to the other focus, has a constant
length.  (A property often exploited to draw an ellipse with a fixed length of string).
The distance of the foci from the centre can be solved from the lengths of the axes, A and B,
as: $c = \surd(a^2-b^2)$, where $a$ and $b$ are the semi-axis lengths.  (See construction and details in Box 4).
So if the summed distance from any point to the foci is longer than the 'string' length, it is outside
and, if less, it is inside.

\begin{figure}[h]
\centering
\epsfxsize=300pt \epsfbox{figs/geom/ellipse.eps}
\caption*{
Box 4:
\label{Fig:box4}
\begin{footnotesize}
{\bf Construct for the {\ttfamily \bfseries inEgg()} routine} which determines if a sphere and an ellipsoid make contact.  
The point {\tt o} is the centre of a sphere lying outside an ellipsoid with major axis
length {\tt a} and minor axis length {\tt b}.  The ellipse (blue) lies in the plane of the ellipsoid axis of rotational
symmetry and the point {\tt o}, with foci are marked as {\tt f1} and {\tt f2}.   If {\tt d} is the distance from a
focus to the minor axis end-point, and {\tt c} is the focus to centre distance, then the length of a focus1-surface-focus2 
path along the major axis is {\tt c+a+(a-c)} = 2{\tt a} and at the minor axis is 2{\tt d}.  Since the paths are equal: {\tt d} = {\tt a};
so {\tt c}$^2$ =  {\tt d}$^2$-{\tt b}$^2$ = {\tt a}$^2$-{\tt b}$^2$ and {\tt c} = {\tt sqrt(a$^2$-b$^2$)}.  Knowing {\tt c},
the path to {\bf o} (dashed) can be found and if this is less than 2{\tt a}, the point lies inside, otherwise out.  However,
we want the surface of the sphere not its centre, but the locus of the centre of a sphere (with radius {\tt r}) in contact with an
ellipsoid is also an ellipsoid, so modifying the semi-axis lengths to {\tt a}+{\tt r} and {\tt b}+{\tt r} gives the required result.
\end{footnotesize}
}
\end{figure}

The sum of the foci distances less the 'string' length is zero on the surface but elsewhere is not
the true distance to the surface.  However, when scaled by 1.4, this value is a good approximation
to the true distance to the surface for both prolate and oblate ellipsoids and is the value returned
by the routine {\tt inEgg()}, which encodes this algorithm.   In its most minimal form, the algorithm
only needs to calculate two distances but as the foci positions are not stored these are also calculated.

In the range $0\ldots R_{ab}$ a more complicated but accurate routine {\tt vec\_to\_egg()} is
used to return the true value of the distance between the surfaces.

\paragraph{Tube/Tube:\\}

If a mutual perpendicular line (the contact normal) is constructed between the axes of a pair
of tubes and if the ends of this lie between the end-points of the tubes, then this is the
closest approach.   Otherwise one of the four end-end distances will be shortest.   The shortest
distance, less $R_{ab}$ is the distance between the surfaces.

\paragraph{Tube/Ellipsoid:\\}

The distance of a tube to an ellipsoid is found using the same algorithm described above
for a point (sphere) and ellipsoid ({\tt inEgg()}) by iteratively bisecting the line between the
tube end-points.

Starting with the three end---mid---end points along the axis ($p1,p2,p3$), then if any corresponding
distance ($d1,d2,d3$) returned by {\tt inEgg()} is negative, there is a clash.   Otherwise, the point
associated with the largest value can be excluded and the calculation repeated with the remaining
two points and their midpoint. (Box 5)   The algorithm converges rapidly and is stopped when the points
get too close.   At the end, the true distance to the ellipsoid surface is returned using the more
complicated {\tt vec\_to\_egg()} routine (since the {\tt inEgg()} value is only exact at the surface).

\begin{figure}[h]
\centering
\begin{singlespace}
\begin{tiny}
\begin{Verbatim}[frame=single]
float tube_to_egg ( Cell *a, Cell *b ) {
// returns an approximation to the closest approach of a tube <a> to an ellipsoid <b> surface
// NB the value returned by inEgg() is not a surface distance but is zero on the surface
// NB assumes radially symmetric ellipsoid
       :
       p1 = a->endN; p2 = a->xyz; p3 = a->endC;
       DO { float d;
               d1 = inEgg(p1,cb,sizeb);
               d2 = inEgg(p2,cb,sizeb);
               d3 = inEgg(p3,cb,sizeb);
               if (d1<0 || d2<0 || d3<0) return -999.9; // flag bump;
               d = p1|p3;
               if (d < 0.01) { // pretty close
                       d = vec_to_egg(p2,cb,sizeb);
                       return d - sizea*0.5;
               }
               if (d1+d2 < d2+d3) {
                       p3 = p2; p2 = p1 & p2;
               } else {
                       p1 = p2; p2 = p3 & p2;
               }
       }
}
\end{Verbatim}
\end{tiny}
\end{singlespace}
\caption*{
Box 5:
\label{Fig:box5}
\begin{footnotesize}
{\bf Code for the {\ttfamily \bfseries tube\_to\_egg} routine} that determines if a tube and an ellipsoid make contact.  
The method uses the fast {\tt inEgg()} routine (Box 4) to check for contact and if none is found, returns the separating
distance using the {\tt vec\_to\_egg()} routine which is based on the algorithm described at:
{\tt http://www.iquilezles.org/www/articles/ellipsedist/ellipsedist.htm}.
\end{footnotesize}
}
\end{figure}

\paragraph{Ellipsoid/Ellipsoid:\\}

Surprisingly, there is no analytic solution for the contact normal between two ellipsoid
surfaces as the expression for this is a quartic equation that requires a numerical solution.
A solution probably could be found for the more symmetric case of two spheroids but
instead, a fast iterative algorithm was used to find the contact normal using a recursive
division approach that is an extension of that used for the simpler tube/ellipsoid problem. 

Rather than iteratively bisect a line, as was done on the tube axis, extending the approach
to a surface leads to the iterative subsection of a triangle --- or rather two, as there
are two ellipsoids to consider.   If the triangles are trisected using a mid-point/vertex
construction, the sub-triangles become progressively elongated.  To avoid this, an internal
triangle was constructed from the mid-points of each edge --- so strictly, each triangle is
quad-sected. (See Box 6).

\begin{figure}[h]
\centering
\epsfxsize=410pt \epsfbox{figs/geom/ellipsoid.eps}
\caption*{
Box 6:
\label{Fig:box6}
\begin{footnotesize}
{\bf Construct for the {\ttfamily \bfseries egg\_to\_egg()} routine} that determines if two ellipsoid make contact.  
The progressively smaller inscribed triangles do not lie on the same plane but on the surface of the ellipsoid
as calculated by the {\tt shell()} routine described in Box 7.   The first division generates four triangles
(bold black lines) and separation of their mid-points, projected on the surface (black dots) is measured between
the two ellipsoids.   The triangles associated with the closest pair are then sub-divided in a similar way
(fine lines) and by the third of fourth division, the midpoints lie close to the contact normal between the
two surfaces (green line).
\end{footnotesize}
}
\end{figure}

A starting set of triangles was obtained from the end-points 
of the axes, giving 8 triangles per ellipsoid and a starting pair was selected which had the
shortest mid/mid point distance.  In all the iterations, the mid point is not simply the mean
of the vertices but is the point where the extension of a line from the centre through this point
cuts the ellipsoid surface.  The utility routine {\tt sholl()} that calculates this is given
in Box 7 and is called by the wrapper routine {\tt shell()} that identifies the ellipsoid type
and reconfigures the argument list appropriately.

\begin{figure}[h]
\centering
\begin{singlespace}
\begin{tiny}
\begin{Verbatim}[frame=single]
Vec sholl ( Vec line, Vec cent, float A, float B, Vec axis ) {
// returns the point on the ellipsoid (<axes> = A>B=C at 0) surface cut by a <line> from the <cent>re
Vec    surf;
float  AA=A*A, BB=B*B, aa,bb,b, d,ff,gg,hh;
/*
in the plane of the major axis (A) and the <line> with components a,b to A,
the point where the line cuts the surface has corresponding components g,h.
Now    gg/AA + hh/BB = 1
and    g/a = h/b
so     gg = AA(1-hh/BB) = aa.hh/bb
       AA - hh.AA/BB = hh.aa/bb
       AA = hh.aa/bb + hh.AA/BB
       hh = AA/(aa/bb+AA/BB)
*/
       line -= cent;                                   // shift line to origin
       b = line.vec_to_line(axis);                     // perpendicular dist from line to axis
       bb = b*b;
       ff = line.sqr();
       aa = ff-bb;
       hh = AA/(aa/bb+AA/BB);
       gg = aa*hh/bb;
       d = sqrt((gg+hh)/ff);
       surf = line*d;                                  // extend <line> to ellipsoid surface
       return cent+surf;                               // added back to centre
}
\end{Verbatim}
\end{tiny}
\end{singlespace}
\caption*{
Box 7:
\label{Fig:box3}
\begin{footnotesize}
{\bf Code for the {\ttfamily \bfseries sholl()} routine} which returns the point at which a line from the
centre cuts the ellipsoid surface.  This routine is specific for a prolate ellipsoid but is called inside 
a wrapper called {\tt shell()} that reconfigures the parameters to deal with any ellipsoid.
\end{footnotesize}
}
\end{figure}

As the selection of the two starting quadrants is based on a rough estimate, all 64 distances are ranked
and the top three combinations taken as separate starting pairs.   The routine then iterates down to the
best pair of (sub-)$_n$triangles in each (max $n=3$) and takes the solution with the shortest separation.
As a final refinement, nine points are selected around each mid-point and the closest pair taken.  
This solution is then checked using {\tt vec\_to\_egg()} to find the distance from the best pair to the
opposing ellipsoid.  These should be identical if the true contact normal has been found.
The error is typically less than 0.1\%.

%--------------------------------------------------
%\begin{figure}[h]
%\centering
%\begin{singlespace}
%\begin{tiny}
%\begin{Verbatim}[frame=single]
%Vec sholl ( Vec line, Vec cent, float A, float B, Vec axis ) {
%// returns the point on the ellipsoid (<axes> = A>B=C at 0) surface cut by a <line> from the <cent>re
%Vec    surf;
%       return cent+surf;                               // added back to centre
%}
%\end{Verbatim}
%\end{tiny}
%\end{singlespace}
%\caption*{
%Box 7:
%\label{Fig:box3}
%\begin{footnotesize}
%{\bf Code for the {\ttfamily \bfseries sholl()} routine} which returns the point at which a line from the
%\end{footnotesize}
%}
%\end{figure}
%--------------------------------------------------
%from cell.cpp
%
%float inEgg ( Vec, Vec, Vec, float );
%
%// short temp names for Data::globals
%char  *names;
%int   *shape, *links, *chain, *split, *local, *align;
%float *sizes, *bumps, *kicks, *keeps, *bonds, *repel, *rejel;
%// common temp variables
%int   total, depth, model, moltype, subtype;
%
%///////// bumper
%
%float touch ( Cell *a, Cell *b )
%{ // closest approach between two cell bump surfaces (of the same shape type)
%  // +ve = separation, -ve = penetration depth
%float	ra, rb, da, db, d, rab;
%int	ta, tb, tab;
%Vec	pa, pb;
%Seg	s;
%	if (a->type > b->type) { Cell *c = a; a = b; b = c; } // swap
%	ta = a->type; tb = b->type;
%	da = Data::model[a->model].bumps[a->level]; ra = da*0.5;
%	db = Data::model[b->model].bumps[b->level]; rb = db*0.5;
%	pa = a->xyz; pb = b->xyz;
%	rab = ra + rb;
%	tab = ta * tb;
%	if (tab>1) s = Seg(b->endN,b->endC);
%	switch (tab) { // smallest type cell first
%		case 0:	// virtual spheres don't bump?
%			return (pa|pb)-rab; // 9999.9;
%		case 1:	// spheres = centre distance
%			return (pa|pb)-rab;
%		case 2:	// sphere+tube (closest approach to line segment or ends)
%			if (pa.vec_in_seg(s)) return pa.vec_to_line(s)-rab;
%			return fmin(pa|s.A,pa|s.B) - rab;
%		case 3:	// sphere+ellipsoid (in keeper.cpp)
%			return vec_to_egg(pa,s,db) - ra;
%		case 4:	// tubes = closest approach of 2 line segments
%			return seg_to_seg(Seg(a->endN,a->endC),Seg(b->endN,b->endC))-rab;
%		case 6:	// tube+ellipsoid (in bumper.cpp)
%			return tube_to_egg(a,b);
%		case 9: // ellipsoid (in bumper.cpp)
%			return egg_to_egg(a,b);
%	}
%}
%
%int bumpex ( Cell *a, Cell *b ) {
%// the children of the cell <a> and <b> are checked for inter-family bumps
%int	i, j, level = a->level, kidlev, n = 0;
%float	d, bumpa,bumpb, bump, over = 1.101;
%float	kicka, kickb, hard, soft, kick;
%float	strength = 0.1;
%Vec	axis;
%	if (a->solid > 0 || b->solid > 0 ) return 0;
%	if (a->empty > 0 || b->empty > 0 ) return 0;
%	if (a->kids == 0 || b->kids == 0 ) return 0;	// no children to bump
%	if (a->level != b->level) return 0; // different level
%	kidlev = level+1;
%	kicka = Data::model[a->model].repel[kidlev],
%	kickb = Data::model[b->model].repel[kidlev],
%	hard = 0.5*(kicka+kickb);
%	kicka = Data::model[a->model].rejel[kidlev],
%	kickb = Data::model[b->model].rejel[kidlev],
%	soft = 0.5*(kicka+kickb);
%	if (kidlev==Data::depth) kick = hard; else kick = soft;
%	axis = b->xyz - a->xyz;
%	axis.setVec(soft);	// soft length vector from a to b (NB has to be set at atom level)
%	FOR(i,a->kids) { Cell* ai = a->child[i];
%		if (ai->empty) continue;
%		FOR(j,b->kids) { Cell* bj = b->child[j];
%			if (bj->empty) continue;
%			if (exempt(ai,bj)) continue;
%			bump = touch(ai,bj);
%			if (bump > -NOISE) continue;
%			d = (ai->xyz|bj->xyz)-bump*over; // d = target gap (NB clash has -ve bump)
%			moveCell(ai,axis,-1);	// nudge ai towards a
%			moveCell(bj,axis, 1);	// nudge bj towards b
%			part2cells(ai,bj,d,-kick*strength); // -kick = repel only
%                        ai->bump = LIVE; bj->bump = LIVE;
%                        ai->hit = bj; bj->hit = ai;
%			n++;
%		}
%	}
%	return n;
%}
%
%int Cell::bumpin () {
%// the children of the current cell <this> are checked for intra-family bumps
%Bumps	*list;
%int	i, j, kidlev, in, m, n = 0;
%float	d, bump, boot, kick, over = 1.01;
%float	hard, soft;
%int	weight = 0;
%	if (this->bump > 0) { // count-down refractory period and if over, clear hit
%	// the plan is to use the period of bumping to activate MD/refinement in potter() 
%		this->bump--;
%		if (this->bump==0) this->hit = 0;
%	}
%	if (solid > 0 ) return 0;
%	if (empty > 0 ) return 0;
%	if (kids == 0 ) return 0;	// no children to bump
%	list = new Bumps[HOLD*kids];
%	kidlev = level+1;
%	hard = Data::model[model].repel[kidlev]; // hard (for hard-shell bump)
%	soft = Data::model[model].rejel[kidlev]; // soft (for jelly bumping)
%	depth = Data::depth;
%	if (kidlev==depth) {
%		kick = hard; // atomic level has hard-shell bump
%	} else {
%		kick = soft; // higher levels have jelly bumping modified by kids if...
%		if (kick > NOISE) weight = 1; // +ve = weight kick by kids (as Gauss(m) soft<-->hard)
%	}
%	if (kick < 0.0) kick = -kick;
%	in = getBumpin(this,list); // list of bumping pairs (by NxN or sort for big N) 
%	for (i=0; i<in; i++) { Cell *a = list[i].a, *b = list[i].b;
%		bump = touch(a,b);
%		if (bump > -NOISE) continue;
%		if (kidlev<depth) m = bumpex(a,b); else m = 0;	// bumping a+b children parted in bumpex() 
%		if (exempt(a,b)) continue;	// exempt parents (exempt atoms skipped in getBumpin()) 
%		// the pair (a,b) are bumping so repel more with more bumping children (m)
%		//	unless weight=0 then just use unmodified <soft> value
%		if (weight) { // Gaussian switch from soft to hard with increasing <m>
%			d = (float)m; d = exp(-d*d);
%			boot = d*soft + (1.0-d)*hard;
%			boot *= kick;
%		} else { boot = kick; }
%		d = (a->xyz|b->xyz)-bump*over; // clash = -ve bump
%		part2cells(a,b,d,-boot); // -kick = repe/l only
%		if (kidlev<depth) {
%               		a->bump = LIVE; b->bump = LIVE;
%                        // a->hit = b; b->hit = a; // replace current interaction? trapped in exempt()?
%               		if (!a->hit) a->hit = b;
%			if (!b->hit) b->hit = a;
%		}
%		n++;
%	}
%	delete [] list;
%}
%
%void Cell::bumps () {
%	this->bumpin();
%	for (int i=0; i<kids; i++) child[i]->bumps();
%}
%
%
%--------------------------------------------------
%from bumper.cpp
%
%
%#include "util.hpp"
%#include "geom.hpp"
%#include "cell.hpp"
%#include "data.hpp"
%
%float inEgg ( Vec, Seg, float );
%
%void bumper ()
%{
%	Cell::world->bumps();
%}
%
%bool exempt ( Cell *a, Cell *b )
%{ // returns 1 if <a> and <b> are exempt from bumping
%//if (a->hit) { Pi(a->level) Pi(a->id) Pi(a->bump) NL }
%	if (a->hit && a->hit==b) return 1;
%	if (b->hit && b->hit==a) return 1;
%	FOR(k,a->nbonds) { if (a->bond[k].to==b) return 1; }
%	FOR(k,b->nbonds) { if (b->bond[k].to==a) return 1; }
%	FOR(k,a->nlinks) { if (a->link[k].to==b) return 1; }
%	FOR(k,b->nlinks) { if (b->link[k].to==a) return 1; }
%	return 0;
%}
%
%#define SWITCH 20	// number of cells below which bump testing is pairwise
%
%int sortBumps ( const void *ac, const void *bc )
%{
%        Bumps   *a = (Bumps*)ac, *b = (Bumps*)bc;
%        if (a->d < b->d) return -1;
%        if (a->d > b->d) return  1;
%        return 0;
%}
%
%int getBumpin ( Cell *cell, Bumps *pairs ) {
%// return the number (and <list>) of possible bumping pairs in cell <c> closer than <bump>
%// <bump>: spheres=rad.s, tubes=lengths, ellipsoids=max-axes
%Cell	*a, *b;
%float	d, dd, bump, bbump;
%Cell	*kid = cell->child[0];
%Data    *p = Data::model+kid->model;
%int	kidlev = kid->level;
%float	size = p->bumps[kidlev];
%int	type = cell->type,
%	kids = cell->kids,
%	hold = HOLD*kids,
%	i, j, k, m, n=0;
%int	*close = new int[hold];
%	if (kids==1) return 0;
%	if (kids < SWITCH) {	// use pairwise
%		for (i=0; i<kids-1; i++) {
%			a = cell->child[i];
%			if (a->empty) continue;
%			for (j=i+1; j<kids; j++) {
%				b = cell->child[j];
%				if (b->empty) continue;
%				if (kidlev==Data::depth && exempt(a,b)) continue; // just filter atom level
%				if (a->len > size) bump = a->len; else bump = size;
%				if (b->len > size) bump += b->len; else bump += size;
%				bump *= 0.5; bbump = bump*bump;
%				dd = a->xyz||b->xyz;
%				if (dd > bbump) continue;
%				pairs[n].c = 0; // not used
%				pairs[n].a = a; pairs[n].b = b;
%				pairs[n].d = sqrt(dd);
%				pairs[n].bump = bump;
%				n++;
%				if (n == hold) break;
%			}
%			if (n == hold) break;
%		}
%		if (n==0) return 0;
%		qsort(pairs,n,sizeof(Bumps),sortBumps);         // sort on separation (closest first)
%		return n;
%	} else	// use sorted lists to find pairs
%	{ int   got, span = 5+(int)sqrt((float)kids);
%		for (i=0; i<kids; i++) {
%			a = cell->child[i];
%			if (a->empty) continue;
%			m = 0;
%			for (j=0; j<3; j++) // gather locally ranked children in each dimension
%			{ int	rat = a->ranks[j], out;
%				out = 0;
%				for (k=rat-span; k<rat+span; k++) { // check the local rank list
%					if (k>=kids) break;	// intra-family so max=kids
%					if (k<0) continue;	// not on list yet
%					if (k==rat) continue;	// skip self
%					if (n == hold) break;
%					b  = cell->rank[k][j];
%					if (b->empty) continue;
%					if (kidlev==Data::depth && exempt(a,b)) continue; // filter atom level
%					close[m] = 10*(b->uid) + j + 1;	// code dim as j/10
%					m++;
%					if (m==hold) break;
%				}
%				if (m==hold) break;
%			}
%			sort(close,m);
%			got = 1; // 123 = XYZ (in sorted order in pairs)
%			FOR(ii,m) { int hit, dim;  // keep only children present in all 3 dimensions
%				hit = (int)(0.1*(float)close[ii]);
%				dim = close[ii]-hit*10;
%				if (dim != got) continue;
%				if (got==3) { float d; // keep
%					got = 0;
%					b = Cell::uid2cell[hit];
%					if (a->len > size) bump = a->len; else bump = size;
%					if (b->len > size) bump += b->len; else bump += size;
%					bump *= 0.5; bbump = bump*bump;
%					dd = a->xyz||b->xyz;
%					if (dd < bbump) {
%						pairs[n].c = 0; // not used
%						pairs[n].a = a; pairs[n].b = b;
%						pairs[n].d = sqrt(dd);
%						pairs[n].bump = bump;
%						n++;
%					}
%					if (n==hold) break;
%				}
%				if (n==hold) break;
%				got++;
%			}
%		}
%		if (n==0) return 0;
%		qsort(pairs,n,sizeof(Bumps),sortBumps);         // sort on separation (closest first)
%		m = 1;
%		FOR(ii,n) { // remove duplicate entries
%			if (ii==0) continue;
%			if ((pairs[ii].a->uid==pairs[ii-1].a->uid) && (pairs[ii].b->uid==pairs[ii-1].b->uid)) continue;
%			if ((pairs[ii].a->uid==pairs[ii-1].b->uid) && (pairs[ii].b->uid==pairs[ii-1].a->uid)) continue;
%			pairs[m] = pairs[ii];
%			m++;
%		}
%		return m;
%	}
%}
%
%float tube_to_egg ( Cell *a, Cell *b ) {
%// returns an approximation to the closest approach of a tube <a> to an ellipsoid <b> surface
%// NB the value returned by inEgg() is not a surface distance but is zero on the surface
%// NB assumes radially symmetric ellipsoid
%Seg	ca = Seg(a->endN,a->endC),	cb = Seg(b->endN,b->endC);
%Data	*pa = Data::model+a->model,	*pb = Data::model+b->model;
%float	sizea = pa->sizes[a->level],	sizeb = pb->sizes[b->level];
%float	d1,d2,d3;
%Vec	p1,p2,p3;
%	if (a->type!=2 && b->type!=3) { Pt(Bad types in tube_to_egg()) Pi(a->type) Pi(b->type) NL exit(1); }
%	p1 = a->endN; p2 = a->xyz; p3 = a->endC;
%	LOOP { float d;
%		d1 = inEgg(p1,cb,sizeb);
%		d2 = inEgg(p2,cb,sizeb);
%		d3 = inEgg(p3,cb,sizeb);
%		if (d1<0 || d2<0 || d3<0) return -999.9; // flag bump;
%		d = p1|p3;
%		if (d < 0.01) { // pretty close
%			d = vec_to_egg(p2,cb,sizeb);
%			return d - sizea*0.5;
%		}
%		if (d1+d2 < d2+d3) {
%			p3 = p2; p2 = p1 & p2;
%		} else {
%			p1 = p2; p2 = p3 & p2;
%		}
%	}
%}
%
%// tested in newsims/bumpell
%#define Nrand  10 // 20 // WAS 500 for testing
%#define Nhold 500
%
%Seg trisect ( int, Vec*, Vec*, Vec*, Vec*, Vec, Vec, Mat, Mat, Vec, Vec, Vec, Vec, float, int );
%Vec shell ( Vec, Vec, Vec, Mat, Vec );
%Vec shall ( Vec, Vec, Vec, Mat, Vec );
%Vec sholl ( Vec, Vec, float, float, Vec );
%float bumping ( int, Vec*, Vec*, Vec, Vec, Vec, Vec, Mat, Mat );
%float in_egg ( Vec, Vec, Vec, Mat );
%
%float egg_to_egg ( Cell *a, Cell *b ) {
%// returns a good approximation to the closest approach of two (general) ellipsoid surfaces
%Data	*pa = Data::model+a->model,	*pb = Data::model+b->model;
%float	sizea = pa->sizes[a->level],	sizeb = pb->sizes[b->level],	// A=B diameter
%	denda = a->endN|a->endC,	dendb = b->endN|b->endC,	// C axis length
%	dmina,dminb, dmaxa,dmaxb, dmin,dmax, da,db, d;
%Seg	ca = Seg(a->endN,a->endC), cb = Seg(b->endN,b->endC), c, p;
%Seg	ends, best;
%float	dends, wsum;
%float	**mat, score, rand = 0.2; // WAS 0.5 for testing
%Mat	axesA, axesB;
%Vec	axisA, axisB, centA, centB, bestA, bestB, surfA, surfB;
%Vec	quad[8], quadA[8], quadB[8], sectA[6], sectB[6];
%Vec	dispA[Nrand], dispB[Nrand];
%Vec	mid, ave, aimA, aimB;
%Vec	moveA, moveB;
%Pairs	dist[64];
%int	rank[64];
%int	qA,qB, qAtop,qBtop, n = 0;
%Vec	allA[Nhold], allB[Nhold];
%Vec	randout;
%int	on = 0;
%	if (a->ends < 0) sizea = denda/Data::Eratio[a->sort]; // use given ratio and C length
%	if (b->ends < 0) sizeb = dendb/Data::Eratio[b->sort]; // use given ratio and C length
%	dmina = min(sizea,denda);	dminb = min(sizeb,dendb);
%	dmaxa = max(sizea,denda);	dmaxb = max(sizeb,dendb);
%	dmin = (dmina+dminb)*0.5;	dmax = (dmaxa+dmaxb)*0.5;
%	d = a->xyz|b->xyz;
%	if (d > dmax) return 999.9;	// beyond maximum contact distance
%	if (d < dmin) return d-dmin;	// centres closer than minimum separation
%	sizea *= 0.5; sizeb *= 0.5; denda *= 0.5; dendb *= 0.5;	// reset to semiaxis lengths
%	mat = new float*[6]; FOR(i,6) mat[i] = new float[6];
%	centA = a->xyz; centB = b->xyz;
%	mid = (centA+centB)*0.5;
%	axisA.x = denda;  axisA.y = axisA.z = sizea;
%	axisB.x = dendb;  axisB.y = axisB.z = sizeb;
%	axesA.A = a->endC - a->xyz;
%	randout = (axesA.A^centB).norm();
%	axesA.B = (axesA.A^randout).getVec(sizea);
%	axesA.C = (axesA.A^axesA.B).getVec(sizea);
%	axesB.A = b->endC - b->xyz;
%	axesB.B = (axesB.A^randout).getVec(sizeb);
%	axesB.C = (axesB.A^axesB.B).getVec(sizeb);
%	quad[0] =  Vec( 1, 1, 1); quad[1] =  Vec( 1, 1,-1); 
%	quad[2] =  Vec( 1,-1, 1); quad[3] =  Vec( 1,-1,-1);
%	quad[4] =  Vec(-1, 1, 1); quad[5] =  Vec(-1, 1,-1);
%	quad[6] =  Vec(-1,-1, 1); quad[7] =  Vec(-1,-1,-1);
%	FOR(i,8) // make real axis end-points for both ellipsoids and put surface centroids in quad[AB]
%	{ Vec	midA, midB;
%		midA = centA + (axesA.A*quad[i].x + axesA.B*quad[i].y + axesA.C*quad[i].z)/3.0;
%		allA[i] = quadA[i] =  shell(midA,centA,axisA,axesA,quad[i]);
%		midB = centB + (axesB.A*quad[i].x + axesB.B*quad[i].y + axesB.C*quad[i].z)/3.0;
%		allB[i] = quadB[i] =  shell(midB,centB,axisB,axesB,quad[i]);
%	}
%	sectA[0] = centA+axesA.A; sectB[0] = centB+axesB.A;
%	sectA[1] = centA+axesA.B; sectB[1] = centB+axesB.B;
%	sectA[2] = centA+axesA.C; sectB[2] = centB+axesB.C;
%	sectA[3] = centA-axesA.A; sectB[3] = centB-axesB.A;
%	sectA[4] = centA-axesA.B; sectB[4] = centB-axesB.B;
%	sectA[5] = centA-axesA.C; sectB[5] = centB-axesB.C;
%	FOR(i,6) { allA[i+8] = sectA[i]; allB[i+8] = sectB[i]; }
%	d = bumping(14, allA,allB, centA,centB, axisA,axisB, axesA,axesB);
%	if (d>NOISE) return -d;
%	n = 0;
%	dends = 9999.9;
%	wsum = 0.0;
%	ave.zero();
%	d = centA | centB;
%	FOR(i,14) FOR(j,14)
%	{ float w, dij; Vec aij; Vec pAi = allA[i], pBj = allB[j];
%		dij = pAi|pBj;  aij = pAi+pBj;
%		if (dij > d) continue;
%		if (dij < dends) { dends = dij; best.A = pAi; best.B = pBj; }
%		w = 1.0/(dij*dij); wsum += w;
%		ave += aij*0.5*w;
%		n++;
%	}
%	ave /= wsum;
%	aimA = shell(ave,centA,axisA,axesA,quad[0]);
%	aimB = shell(ave,centB,axisB,axesB,quad[0]);
%	n = 0;
%	FOR(i,8) FOR(j,8) {
%		dist[n].a = i; dist[n].b = j; 
%		dist[n].s = (quadA[i]|quadB[j])+(aimA|quadA[i])+(aimB|quadB[j]);
%		n++;
%	}
%	sort(dist,rank,-n);
%	dmin = 999.9;
%	allA[6] = aimA; allB[6] = aimB;
%	FOR(i,3) // test just the top 3
%	{ int	r = rank[i], qA = dist[r].a, qB = dist[r].b;
%	  float dab;
%		// pass quadrant end-points and mid-points in sect[AB] to trisect()
%		sectA[0] = centA + axesA.A*quad[qA].x;
%		sectA[2] = centA + axesA.B*quad[qA].y;
%		sectA[4] = centA + axesA.C*quad[qA].z;
%		sectA[1] = shell((sectA[0]+sectA[2])*0.5,centA,axisA,axesA,quad[qA]);
%		sectA[3] = shell((sectA[2]+sectA[4])*0.5,centA,axisA,axesA,quad[qA]);
%		sectA[5] = shell((sectA[4]+sectA[0])*0.5,centA,axisA,axesA,quad[qA]);
%		sectB[0] = centB + axesB.A*quad[qB].x;
%		sectB[2] = centB + axesB.B*quad[qB].y;
%		sectB[4] = centB + axesB.C*quad[qB].z;
%		sectB[1] = shell((sectB[0]+sectB[2])*0.5,centB,axisB,axesB,quad[qB]);
%		sectB[3] = shell((sectB[2]+sectB[4])*0.5,centB,axisB,axesB,quad[qB]);
%		sectB[5] = shell((sectB[4]+sectB[0])*0.5,centB,axisB,axesB,quad[qB]);
%		FOR(j,6) { allA[j] = sectA[j]; allB[j] = sectB[j]; }
%		ends = trisect(7,allA,allB,sectA,sectB,centA,centB,axesA,axesB,axisA,axisB,quad[qA],quad[qB],9999.9,0);
%		if (ends.B.z > 9999.0) return -ends.B.x;	// found bumping
%		dab = ends.len();
%		if (dab > dmin) continue;
%		best = ends; dmin = dab; qAtop = qA; qBtop = qB; n = i;
%	}
%	rand *= sqrt(dmin);
%	moveA = best.A;
%	FOR(i,Nrand) { Vec v = moveA; v.set_disp(rand);
%		surfA = shell(v,centA,axisA,axesA,quad[qAtop]);
%		dispA[i] = surfA;
%		d = surfA|best.B; if (d<dmin) { moveA = surfA; dmin = d; }
%	}
%	moveB = best.B;
%	FOR(i,Nrand) { Vec v = moveB; v.set_disp(rand);
%		surfB = shell(v,centB,axisB,axesB,quad[qBtop]);
%		dispB[i] = surfB;
%		d = surfB|moveA; if (d<dmin) { moveB = surfB; dmin = d; }
%	}
%	FOR(i,Nrand) FOR(j,Nrand) {
%		d = dispA[i]|dispB[j];
%		if (d<dmin) {
%			moveA = dispA[i]; moveB = dispB[j]; dmin = d;
%		}
%	}
%	return dmin;
%}
%
%float check_normal ( Vec A, Vec a0, Vec a1, Vec a2,  Vec B, Vec b0, Vec b1, Vec b2 ) {
%// return the sum of the distances that the normals from <A> and <B> pass their opposing points
%Vec	normA, normB;
%float	toA, toB;
%	normA = A+((a1-a0)^(a2-a0));
%	normB = B+((b1-b0)^(b2-b0));
%	toA = A.vec_to_line(B,normB);
%	toB = B.vec_to_line(A,normA);
%	return toA+toB;
%}
%
%Seg trisect ( int on, Vec *allA, Vec *allB, Vec *sectA, Vec *sectB, Vec centA, Vec centB, Mat axesA, Mat axesB, Vec axisA, Vec axisB, Vec qA, Vec qB, float last , int depth ) {
%int	set[4][3] = { 1,2,3, 3,4,5, 5,0,1, 1,3,5 };
%Vec	midA,midB, bestA,bestB, surfA,surfB;
%int	triA0,triB0, triA1,triB1, triA2,triB2;
%float	wsum, w, d, dif, dab, dmin;
%Vec	A0,A1,A2, B0,B1,B2;
%Vec	aim, aimA, aimB;
%Vec	a[8], b[8];
%Seg	ends;
%int	p,q;
%	aimA.zero();
%	aimB.zero();
%	wsum = 0.0;
%	FOR(i,on) FOR(j,on) {
%		d = allA[i]|allB[j]; w = 1.0/(d*d);
%		aimA += allA[i]*w; aimB += allB[j]*w;
%		wsum += w;
%	} // set target points <aimA>,<aimB> to weighted mean and project onto surface
%	aimA /= wsum; aimB /= wsum;
%	aimA = shell(aimA,centA,axisA,axesA,qA);
%	aimB = shell(aimB,centB,axisB,axesB,qB);
%	allA[on] = aimA; allB[on] = aimB; on++;	// append to current surface points
%	dmin = 9999.9;
%	FOR(i,on) FOR(j,on) { // find min of all current surface pairs
%		d = allA[i]|allB[j];
%		if (d<dmin) { dmin = d; p = i; q = j; }
%	}
%	aimA = allA[p]; aimB = allB[q];
%	FOR(i,4) // make midpoints for the 4 triangles in segments A and B
%	{ int	si0 = set[i][0], si1 = set[i][1], si2 = set[i][2];
%		midA = (sectA[si0]+sectA[si1]+sectA[si2])/3.0;
%		allA[on+i] = a[i] = shell(midA,centA,axisA,axesA,qA);
%		midB = (sectB[si0]+sectB[si1]+sectB[si2])/3.0;
%		allB[on+i] = b[i] = shell(midB,centB,axisB,axesB,qB);
%	}	// and append to collection of points <on> the surfaces 
%	on += 4;
%	d = bumping(on, allA,allB, centA,centB, axisA,axisB, axesA,axesB);
%	if (d>NOISE) { 
%		return Seg(Vec(d,0,9999.9)); // Seg.B.z > 9999 = bump 
%	}
%	dmin = 9999.9;
%	w = (float)depth;
%	if (depth > 1) w = sqrt(w);
%	FOR(i,4) // loop over the 4 midpoints in segment A
%	{ int	si0 = set[i][0], si1 = set[i][1], si2 = set[i][2];
%		midA = a[i];
%		FOR(j,4) // loop over the 4 midpoints in segment B
%		{ int	sj0 = set[j][0], sj1 = set[j][1], sj2 = set[j][2];
%			midB = b[j];
%			d = check_normal(midA,sectA[si0],sectA[si1],sectA[si2],midB,sectB[sj0],sectB[sj1],sectB[sj2]);
%			dab = w*sqrt(d)+(2.0/(1+w))*(midA|midB)+(aimA|midA)+(aimB|midB);
%			if (dab<dmin) {
%				dmin = dab;
%				triA0=si0; triA1=si1; triA2=si2;
%				triB0=sj0; triB1=sj1; triB2=sj2;
%				p = i; q = j;
%			}
%		}
%	}
%	// set best triangle pair for next iteration
%	A0 = sectA[triA0]; A1 = sectA[triA1]; A2 = sectA[triA2];
%	B0 = sectB[triB0]; B1 = sectB[triB1]; B2 = sectB[triB2];
%	surfA = a[p]; surfB = b[q];
%	dab = surfA|surfB;
%	dif = last-dab;
%	//if ((depth > 3 && last-dab < 0.001)||(depth > 6)) {
%	if (depth==3) { // go with whatever (4xfaster than above)
%		// NB for depth > 7, grid points get too close. For depth=9, on=92 (check Nhold)
%		d = bumping(on, allA,allB, centA,centB, axisA,axisB, axesA,axesB);
%		if (d>NOISE) { 
%			return Seg(Vec(d,0,9999.9)); // Seg.B.z > 9999 = bump 
%		}
%		dmin = 9999.9;
%		FOR(i,on) FOR(j,on) { // find min of all current surface pairs
%			d = allA[i]|allB[j];
%			if (d<dmin) { dmin = d; p = i; q = j; }
%		}
%		surfA = allA[p]; surfB = allB[q];
%		return Seg(surfA,surfB);
%	}
%	sectA[0] = A0; allA[on+0] = sectA[1] = shell((A0+A1)*0.5,centA,axisA,axesA,qA);
%	sectA[2] = A1; allA[on+1] = sectA[3] = shell((A1+A2)*0.5,centA,axisA,axesA,qA);
%	sectA[4] = A2; allA[on+2] = sectA[5] = shell((A2+A0)*0.5,centA,axisA,axesA,qA);
%	sectB[0] = B0; allB[on+0] = sectB[1] = shell((B0+B1)*0.5,centB,axisB,axesB,qB);
%	sectB[2] = B1; allB[on+1] = sectB[3] = shell((B1+B2)*0.5,centB,axisB,axesB,qB);
%	sectB[4] = B2; allB[on+2] = sectB[5] = shell((B2+B0)*0.5,centB,axisB,axesB,qB);
%	d = bumping(on+3, allA,allB, centA,centB, axisA,axisB, axesA,axesB);
%	if (d>NOISE) { 
%		return Seg(Vec(d,0,9999.9)); // Seg.B.z > 9999 = bump 
%	}
%	// pass sub-triangles (and append to surface collection)
%	ends = trisect(on+3,allA,allB,sectA,sectB,centA,centB,axesA,axesB,axisA,axisB,qA,qB,dab,depth+1);
%	return ends;
%}
%
%float bumping ( int n, Vec *allA, Vec *allB, Vec centA, Vec centB, Vec axisA, Vec axisB, Mat axesA, Mat axesB )
%{
%int	p, q = 0;
%float	d = 0.0, dmin = 999.9;
%Vec	surfA, surfB, quad = Vec(1,1,1);
%	FOR(i,n) // check if any points lie inside the other ellipsoid
%	{ float din;
%		din = in_egg(allA[i],centB,axisB,axesB);
%		if (din < 1.0-NOISE) { // A is in B (flag with q=1)
%			if (din<dmin) { dmin = din; p = i; q =  1; }
%		}
%		din = in_egg(allB[i],centA,axisA,axesA);
%		if (din < 1.0-NOISE) { // B is in A (flag with q=-1)
%			if (din<dmin) { dmin = din; p = i; q = -1; }
%		}
%	}
%	if (q>0) { // A is deeper inside B
%		surfB = shell(allA[p],centB,axisB,axesB,quad);
%		d = surfB|allA[p];
%		// Pi(n) Pr(dmin) Pt(A inside B by) Pr(d) NL
%	}
%	if (q<0) { // B is deeper inside A
%		surfA = shell(allB[p],centA,axisA,axesA,quad);
%		d = surfA|allB[p];
%		// Pi(n) Pr(dmin) Pt(B inside A by) Pr(d) NL
%	}
%	return d;
%}
%
%float in_egg ( Vec p, Vec c, Vec x, Mat axes ) {
%// Returns the fractional distance <d> of point <p> to the ellipsoid surface.
%// <d> < 1 = inside, <d> > 1 = outside
%// The ellipsoid has axes <M> (at 0 with lengths x)
%float	d;
%Vec	r, q = (p-c).get_frac(axes);		// q = components of <p> in basis set <M>
%	r = Vec(q.x*x.x, q.y*x.y, q.z*x.z);	// r = real-space position of <p> in <M>
%	d = r.x*r.x/(x.x*x.x) + r.y*r.y/(x.y*x.y) + r.z*r.z/(x.z*x.z); // surface definition
%	return d;
%}
%
%Vec shall ( Vec line, Vec cent, Vec axis, Mat axes, Vec quad ) {
%// returns the point on the ellipsoid (<axes>=ABC at 0) surface cut by a <line> from the <cent>re
%/*
%if the axes lie on XYZ then,
%	xx/AA + yy/BB + zz/CC = 1
%	x = pa, y = pb, z = pc
%	pp = 1/(aa/AA+bb/BB+cc/CC)
%	line = abc, surf = xyz
%*/
%Mat	frame;
%Vec	surf;
%float	A,B,C, a,b,c, p;
%	line -= cent;					// quadrant centroid shifted to origin
%	frame = Mat(axes.A.getVec()*quad.x, axes.B.getVec()*quad.y, axes.C.getVec()*quad.z);
%	surf = line*frame;				// rotated to A=X, B=Y, C=Z
%	A=axis.x, B=axis.y, C=axis.z;			// semi-axis lengths
%	a=surf.x, b=surf.y, c=surf.z;			// rotated point
%	p = 1.0/sqrt(a*a/(A*A)+b*b/(B*B)+c*c/(C*C));	// scale factor
%	surf *= p;					// to put point on ellipsoid surface
%	surf *= frame.get_trans();			// rotated back to ABC frame
%	return cent+surf;				// added back to centre
%}
%
%Vec sholl ( Vec line, Vec cent, float A, float B, Vec axis ) {
%// returns the point on the ellipsoid (<axes> = A>B=C at 0) surface cut by a <line> from the <cent>re
%/*
%in the plane of the major axis (A) and the <line> with components a,b to A,
%the point where the line cuts the surface has corresponding components g,h.
%Now	gg/AA + hh/BB = 1
%and	g/a = h/b
%so	gg = AA(1-hh/BB) = aa.hh/bb
%	AA - hh.AA/BB = hh.aa/bb
%	AA = hh.aa/bb + hh.AA/BB
%	hh = AA/(aa/bb+AA/BB)
%*/
%Vec	surf;
%float	AA=A*A, BB=B*B, aa,bb,b, d,ff,gg,hh;
%	line -= cent;					// shift line to origin
%	b = line.vec_to_line(axis);			// perpendicular dist from line to axis
%	bb = b*b;
%	ff = line.sqr();
%	aa = ff-bb;
%	hh = AA/(aa/bb+AA/BB);
%	gg = aa*hh/bb;
%	d = sqrt((gg+hh)/ff);
%	surf = line*d;					// extend <line> to ellipsoid surface
%	return cent+surf;				// added back to centre
%}
%
%Vec shell ( Vec line, Vec cent, Vec axis, Mat axes, Vec quad ) {
%float	ab, bc, ca, close = NOISE;
%	ab = axis.x - axis.y;
%	bc = axis.y - axis.z;
%	ca = axis.z - axis.x;
%	// check for oblate/prolate
%	if (ab*ab < close) return sholl(line,cent,axis.z,axis.x,axes.C);
%	if (bc*bc < close) return sholl(line,cent,axis.x,axis.y,axes.A);
%	if (ca*ca < close) return sholl(line,cent,axis.y,axis.z,axes.B);
%	// otherwise scalene
%	return shall(line,cent,axis,axes,quad);
%}
