\subsection{Confining children in their parent object}

The \TT{keeper} routine keeps the children of the current object inside (or on) its
surface.  There are only the three basic shapes to consider and the treatment
of these uses much the same routines that were described in the previous section.

A parameter $keep$ is set from supplied inputs for each level that sets
the size of the step by which straying children are returned to their parental
object.   The sign of $keep$ is also used as a flag to modify the confinement
behaviour.  If the value of $keep$ 
is positive, then children are confined within spheres and ellipsoids but
are confined to the surface of a tube and these roles are reversed when
$keep$ is negative.   Having the default behaviour to locate children at the
surface of a tube is useful both for protein secondary structures and for
double-stranded nucleic acid segments.  (The tube diameter is automatically
set to the ideal value when these molecules are encountered).  

The use of a tube for a \Bs\ is less obvious as these elements can have a marked 
super-helical twist. However a slightly larger tube gives space for this twist
and indeed confinement on the surface encourages the chain to adopt a super-helix.   

\subsubsection{Spheres}

The {\tt packBall()} routine is used by \TT{keeper} to keep children inside a sphere and
provides a simple template of the other two routines ({\tt packTube()} and {\tt packEgg()}
described below).    The code (Box 8) is self-explanatory but contains two aspects that
require some elaboration.

\begin{figure}[h]
\centering
\begin{singlespace}
\begin{tiny}
\begin{Verbatim}[frame=single]
int packBall ( Cell* cell, Cell* child, float strict )
{
       :
       radius -= model->sizes[child->level]*0.5;       // keep totally inside
       if (push < -NOISE) shell = 1;                   // confine to shell (+/-margin)
                else      shell = 0;                   // confine inside
       :
       shift = cell->xyz - child->xyz;                 // shift from child to zero
       d = shift.len();                                // distance from child to parent
       if (shell) {
               if (d>radius*margin && d<radius*margout) return 0; // in the margin zone
               if (d > radius) push = -push;           // outside the sphere (so pull)
       } else {
               if (d < radius*margout) return 0;       // within the sphere
       }
       shift *= push;
       child->move(shift);
       return 1;
}

\end{Verbatim}
\end{tiny}
\end{singlespace}
\caption*{
Box 8:
\label{Fig:box8}
\begin{footnotesize}
{\bf Code for the {\ttfamily \bfseries packBall()} routine} that confines objects inside a sphere.
The objects are confined by their centres plus their radius or the shorter of their semi-axis lengths.
This only corresponds exactly to their surface for a sphere, so the ends of tubes and oblate ellipsoids
can extend beyond their enclosing surface.
The {\tt shell} flag sets the option for them to be confined at the surface.
\end{footnotesize}
}
\end{figure}

As implemented, the code subtracts the child radius from that of the parent,
so for spheres, the full body of the child will be kept inside the parent.  However for elongated objects,
only the radius normal to the axis of symmetry will be used so the ends of tubes and prolate ellipsoids
can stick-out, whereas oblate ellipsoids will be slightly over-confined.

To save a little computation time, a margin is maintained about the surface, within which no action is
taken.  This is set at $\pm10$\% ({\tt margin}=0.9, {\tt margout}=1.1).   Although it appears that this
will save only a few arithmetic operations, it should be remembered that all geometric operations
applied to objects (such as the {\tt move()} utility above) are
recursive and will be applied to the full underlying sub-tree of objects.
The margin also prevents rapid small in/out fluctuations of children that lie close to the surface
which can be visually disturbing.

\subsubsection{Tubes}

Tubes have a cylindrical body and hemi-spherical end-caps.  By default, children are constrained to 
lie on the surface of the cylinder and its caps.   If the normal from the child centre to the tube axis
intersects between the tube end-points then the child is shifted along the normal towards the surface.  
If outside the axial line-segment,
then it is shifted in the same way as described for a sphere (above), taking the nearest end-point as a
centre.  If the value of the $keep$ parameter is negative, then children that lie inside the tube are left
unmoved.  The same margin zone described above for spheres is also implemented.

An exception is made for the children of protein loop regions (which also have a
tube object associated with them) but they are much less constrained and are only held
inside the tube (not on its surface) with 1/10 the weight of the equivalent \AH\ and \Bs.  
In addition, the end-cap constraints are not enforced.

An exception is also made for double-stranded nucleic acid segments where the tube enclosing the
double helix is a domain level object (atomic-2).   It is therefore quite undesirable to have the
base-pair 'secondary-structure' tubes confined to this surface but rather their children (the pair
of phosphates) should be on the surface.  To implement this, a wrapper routine ({\tt packBase()})
is used to call {\tt packTube()} with a skipped generation as: {\tt packTube(grandparent,child,...)}
instead of the normal {\tt packTube(parent,child,...)}.  {\tt packBase()} also refines the P---P
distance across the basepair and sets the axis end-points of their tube to track the phosphates.
As these constraints have no chiral component, the torsion angle about the axis for baspaired
phosphates is also refined, so maintaining the correct hand of the double helix.

\subsubsection{Ellipsoids}

For ellipsoids, the \TT{keeper} routine follows the template for the sphere but uses the utility {\tt inEgg()},
which was described above, to decide who is out and who is inside.   It will be recalled that
{\tt inEgg()} needs only the distance between the two foci of the ellipse-of-rotation to do this
and so requires little computation time.

Unlike the tube end-caps, where the nearest centre was used to set the shift direction, instead, a
weighted combination of the distances to both foci are used.   The weights are taken as the inverse distance
to each focus: so if the child lies closer to focus-1, it will have a larger weight
on the component of the displacement vector in the direction of focus-1, and {\em vice versa}. 

%#include "util.hpp"
%#include "geom.hpp"
%#include "cell.hpp"
%#include "data.hpp"
%
%int  packBall ( Cell*, Cell*, float );
%int  packTube ( Cell*, Cell*, float );
%void packBase ( Cell*, Cell*, float );
%float inEgg ( Vec, Vec, Vec, float, Vec& );
%float inEgg ( Vec, Vec, Vec, float );
%float inEgg ( Vec, Seg, float );
%int packEgg ( Cell*, Cell*, float );
%
%float margin = 0.9, margout = 1.0/margin;
%
%void center ()
%{
%Cell	*world = Cell::world;
%	DO(i,world->kids) { Cell *child = world->child[i];
%                if (child->empty) continue;
%		child->group();
%	}
%}
%
%void keeper ( Cell *cell )
%{
%int	m = cell->model,
%	moltype = Data::model[m].moltype,
%	subtype = Data::model[m].subtype,
%	dna = moltype*subtype;
%	DO(i,cell->kids) // push the children inside
%	{ Cell *child = cell->child[i];
%	  int	type = cell->type, sort = cell->sort, lost = 0;
%	  float s, strict, balance;
%                if (child->empty) continue;
%/* ?
%		if (chain[level] > 0 && chain[level+1] > 0) {   // in a chain of chains (don't push ends in)
%			if (child == cell->starts || child == cell->finish) continue;
%		}
%*/
%		strict = 1.0;
%		switch (type) {
%			case 0 :	// dummy sphere
%				lost = packBall(cell,child,1.0);
%			break;
%			case 1 :	// simple sphere
%				lost = packBall(cell,child,strict);
%			break;
%			case 2 :	// SSE tube
%				if (sort) s = strict; else s = 0.1*strict;
%				if (moltype==1) {
%					packBase(cell,child,s);
%				} else {
%					lost = packTube(cell,child,s);
%				}
%			break;
%			case 3 :	// Ellipsoid
%				if (cell->sort == E/2) {	// same as a sphere
%					lost = packBall(cell,child,1.0);
%				} else {
%					lost = packEgg(cell,child,1.0);
%				}
%			break;
%		}
%	}
%	DO(i,cell->kids) keeper(cell->child[i]);
%}
%
%void packBase ( Cell *cell, Cell* child, float strict ) {
%// <cell> = 'basepair secondary structure', child = base (P), <pa> = segment (domain)
%Cell	*pa = cell->parent;
%Vec	mid, shift;
%	if (pa->type != 2 ) return;
%	if (child->level != depth) return;
%	// confine atoms to level-2 tube (DNA segment is at domain level)
%	if (cell->sort == 0) strict *= 0.1;
%	packTube(pa,child,strict);
%	if (cell->sort == 0) { // reset loop ends
%		shift = (cell->junior->xyz - cell->senior->xyz)/4.0;
%		cell->endN = cell->xyz - shift;
%		cell->endC = cell->xyz + shift;
%		return;
%	}
%	// reset basepair tube ends
%	part2cells(cell->child[0],cell->child[1],2.7,0.1);
%	cell->xyz = cell->child[0]->xyz & cell->child[1]->xyz;
%	shift = (cell->child[1]->xyz - cell->child[0]->xyz)*0.7/2.0;
%	cell->endN = cell->xyz - shift;
%	cell->endC = cell->xyz + shift;
%	// shift basepair tube centre towards axis
%	mid = (cell->xyz).vec_on_line(pa->endN,pa->endC);
%	shift = mid - cell->xyz;
%	cell->move(shift*0.1);
%}
%
%int packBall ( Cell* cell, Cell* child, float strict )
%{
%int	shell;
%Vec	shift;
%Data	*model = Data::model+cell->model;
%float	radius = model->sizes[cell->level]*0.5,
%	push   = model->keeps[cell->level]*strict,
%	d;
%	if (cell->level==Data::depth) return 0;
%	if ((int)cell->endN.z == 1234) return 0;	// ends not set yet
%	if ((int)cell->endC.z == 1234) return 0;	// ends not set yet
%	radius -= model->sizes[child->level]*0.5;	// keep totally inside
%	if (push < -NOISE) shell = 1;			// confine to shell (+/-margin)
%		 else	   shell = 0;			// confine inside
%	shift = cell->xyz - child->xyz;			// shift from child to zero
%	d = shift.len();
%	if (shell) {
%		if (d>radius*margin && d<radius*margout) return 0;
%		if (d > radius) push = -push;		// outside the sphere
%	} else {
%		if (d < radius*margout) return 0;	// within the sphere
%	}
%	shift *= push;
%	if (shift.sqr()<NOISE) return 0;
%	child->move(shift);
%	return 1;
%}
%
%int packTube ( Cell *cell, Cell *child, float strict )
%{
%int	in, shell, closed = 1;
%Vec	shift, over, temp;
%float	dlast, dnext, dif, d, pale;
%Data	*model = Data::model+cell->model;
%float	radius = model->sizes[cell->level]*0.5,
%	push   = model->keeps[cell->level]*strict;
%int	moltype = model->moltype,
%	subtype = model->subtype,
%	protein = 0;
%	if (moltype==0 && subtype==1) protein = 1;
%	if (cell->level==Data::depth) return 0;
%	if ((int)cell->endN.z == 1234) return 0; // ends not set yet
%	if ((int)cell->endC.z == 1234) return 0; // ends not set yet
%	if (protein && cell->level==depth-1) { // apply SSE factor
%		if (cell->sort==0) radius *= loopTHIC;
%		if (cell->sort==1) radius *= alphTHIC;
%		if (cell->sort==2) radius *= betaTHIC;
%	}
%//	radius -= model->sizes[child->level]*0.5; // keep totally inside
%	pale = radius;
%	if (push < -NOISE) shell = 0;		// confine inside
%		 else	   shell = 1;		// confine to shell (+/-margin)
%	if (shell) push = -push;
%	if (protein && cell->sort==0) {
%		shell = closed = 0;	// protein loops always have open ends and no shell
%		push = -push;		// mimic -ve keep value from param file
%	}
%	if (vdif(cell->endN,cell->endC) < NOISE) return 0;
%	if (closed) {
%		dlast = cell->endN|child->xyz;
%		dnext = cell->endC|child->xyz;
%		if (!shell) { // make quick pre-check at ends
%			if (dlast < pale) return 0;
%			if (dnext < pale) return 0;
%		}
%	}
%	in   = child->xyz.vec_in_seg (cell->endN, cell->endC); // in the line segment
%	d    = child->xyz.vec_to_line(cell->endN, cell->endC); // dist to extended line
%	over = child->xyz.vec_on_line(cell->endN, cell->endC); // image on extended line
%	if (closed) { // confine within segment
%		if (in) { // over line segment
%			if (d < pale) { // inside the tube
%				if (!shell) return 0;	// inside is OK
%				if (d < pale*margin) {	// too deep inside
%					dif = fabs(pale-d);
%					push *= dif;
%				} else { return 0; }	// in the shell margin
%			} else {	// outside the tube
%				push = -push;	// pull in
%			}
%			shift = over - child->xyz;
%		} else { // over an end-cap (so find which one)
%			if (dlast < dnext) {		// over endN
%				shift = cell->endN - child->xyz;
%				d = cell->endN | child->xyz;
%			} else{				// over endC
%				shift = cell->endC - child->xyz;
%				d = cell->endC | child->xyz;
%			}
%			if (d < pale) { // inside the cap
%				if (!shell) return 0;	// inside is OK
%				if ( d < pale*margin) { // too deep inside
%					dif = fabs(pale-d);
%					push *= dif;
%				} else { return 0; }	// in shell margin
%			} else {	// outside the tube
%				push = -push;	// pull in
%			}
%		}
%	} else { // confine within extended tube
%		if (d < pale) { // inside the tube
%			if (!shell) return 0;	// inside is OK
%			if (d < pale*margin) {	// too deep inside
%				dif = fabs(pale-d);
%				push *= dif;
%			} else { return 0; }
%		}
%		shift = over - child->xyz;
%	}
%	shift.setVec(push);
%	if (shift.sqr()<NOISE) return 0;
%	child->move(shift);
%	return 1;
%}
%
%float inEgg ( Vec cell, Vec endN, Vec endC, float len, Vec &move )
%// returns a value that is +ve outside and -ve inside the ellipsoid (not surface distance)
%// cell = point, endN,endC = poles, dB = diameter across minor axes, move = returned shift
%{
%float	dB, rB, dA, rA;
%Vec	p,q,r, cent, axis;
%float	a, b, c, d, e, f, g;
%float	v,w,x,y,z;
%int	apply = 1;
%	if (len<0) { len = -len; apply = 0; }
%	cent = endN & endC;	// set centre
%	axis = endC - endN;	// axis between poles
%	dA = endN|endC;		// dist between poles
%	dB = len;		// diameter across minor axis
%	rA = dA*0.5;		// major (rA) and...
%	rB = dB*0.5;		// minor (rB) semi-axis lengths
%	x = y = 1.0; z = dA/len;
%	// bubble sort axes to x>y>z
%	if (x<y) { w=x; x=y; y=w; }
%	if (y<z) { w=y; y=z; z=w; }
%	if (x<y) { w=x; x=y; y=w; }
%	v = y/x; w = z/x;
%	d = rA;	// major semi-axis length
%	a = 1.0-v; b = 1.0-w; // fractional axis lengths relative to X
%	if (a*a+b*b < 0.04) { // sphere-ish (both minor axes over 0.8-ish)
%		g = 3.0*d/(1.0+v+w);	// mean sphere radius
%		c = cell|cent;
%		if (apply) move = (cent-cell).norm();	// move towards centre
%		return c-g;
%	}
%	if (v > 0.5+0.5*w) {	// oblate
%		g = d;
%		d = d/w*(1.0+v)*0.5; // force radial symmetry about Z (min)
%		p = cell-cent; q = axis; r = p^q;
%		axis = r^q;	// axis now lies on cell-cent-pole plane
%	} else {		//prolate
%		g = d*0.5*(v+w);	// force radial symmetry about X (max)
%	}
%/*
%                             * cell (external point)
%                     ..-+-../
%                .    d /|  /    .           d = focus pole distance
%              .       /b| /       .         b = minor axis length/2
%             :       /  |/c        :        c = foucs cent distance
%        endN |------x---+---x------| endC   x = focii on major axis
%                       cent                 a = major axis length/2
%	length of focus1-surface-focus2 path:
%	at major axis = c+a+(a-c) = 2a
%	at minor axis = 2d, dd = bb+cc
%	since the paths are equal: d = a;
%	so	cc = dd-bb = aa-bb
%	and 	c = sqrt(aa-bb)
%*/
%	c = sqrt(d*d-g*g);	// distance from centre to focii
%	axis.setVec(c);		// set axis to length c
%	p = cent+axis;		// up axis to focus1
%	e = cell|p;		// dist to focus1
%	q = cent-axis;		// down axis to focus2
%	f = cell|q;		// dist of focus2
%	if (apply) { // move is p*f+q*e (swap e/f weights to bias towards nearest focus)
%		p -= cell;	// vect to focus1
%		q -= cell;	// vect to focus2
%		p.setVec(f);
%		q.setVec(e);
%		move = (p+q).norm();
%	}
%	return 1.4*(0.5*(e+f)-d);
%}
%
%float inEgg ( Vec cell, Vec endN, Vec endC, float len )
%{ // just return the distance (-ve len = no movement)
%Vec	x;
%	return inEgg(cell,endN,endC,-len,x);
%}
%
%float inEgg ( Vec cell, Seg ends, float len )
%{ // just return the distance (-ve len = no movement)
%Vec	x;
%	return inEgg(cell,ends.A,ends.B,-len,x);
%}
%
%int packEgg ( Cell *cell, Cell *child, float strict )
%{
%Vec	shift;
%int	shell,
%	level = cell->level,
%	type = cell->type,
%	sort = cell->sort;
%Data	*model = Data::model+cell->model;
%float	radius = model->sizes[cell->level]*0.5,
%	push   = model->keeps[cell->level]*strict,
%	d, dout;
%	if (cell->level==Data::depth) return 0;
%	if ((int)cell->endN.z == 1234) return 0;	// ends not set yet
%	if ((int)cell->endC.z == 1234) return 0;	// ends not set yet
%	if (push < -NOISE) shell = 1;			// confine to shell (+/-margin)
%		else	   shell = 0;
%	d = cell->endN | cell->endC;
%	if (d < NOISE) return 0;			// no axis length
%	if (cell->ends < 0) { float z = Data::Eratio[sort];
%		radius = 0.5*d/z;
%	}
%	if (shell==0) {
%		d = cell->xyz|child->xyz;
%		if (d < radius*margin) return 0;	// inside the inscribed sphere
%	}
%	dout = inEgg(child->xyz,cell->endN,cell->endC,radius*2.0,shift);
%	if (shell) {
%		if (dout>-0.1 && dout<0.1) return 0;	// close enough to surface
%		if (dout>0.0) push = -push;		// switch to pulling in
%	} else {
%		if (dout < 0.0) return 0;		// inside
%	}
%	if (dout > 0.0) push *= sqrt(dout);		// go easy on far-away children
%	shift *= push;					// push back along resultant
%	if (shift.sqr()<NOISE) return 0;
%	child->move(shift);
%	return 1;
%}
%
%/*
%sortTube(Cells *cell) 
%{
%int	i, n = cell->kids;
%int	run=0, in=0;
%Vec	x,y,z,ave;
%	vinit(&ave);
%	for (i=0; i<n-1; i++) { Cells *b1,*b2, *p1,*p2;
%		b1 = cell->child[i];
%		b2 = cell->child[i+1];
%		if (b1->link[0]==0) continue;
%		if (b2->link[0]==0) continue;
%		p1 = b1->link[0];
%		p2 = b2->link[0];
%		if (b1->id == b2->id-1 && p1->id == p2->id+1) {
%			//Pi(b1->id) Pi(p1->id) Pi(b2->id) Pi(p2->id) NL
%			vave(b1->xyz,p1->xyz,&x);
%			vave(b2->xyz,p2->xyz,&y);
%			vsub(x,y,&z);
%			vsum(z,&ave);
%			run++; in++;
%		}
%		if (b1->id == b2->id+1 && p1->id == p2->id-1) {
%			//Pi(b1->id) Pi(p1->id) Pi(b2->id) Pi(p2->id) NL
%			vave(b1->xyz,p1->xyz,&x);
%			vave(b2->xyz,p2->xyz,&y);
%			vsub(x,y,&z);
%			vsum(z,&ave);
%			run--; in++;
%		}
%	}
%	vdiv(&ave,(float)in); vnorm(&ave);
%	// recheck all against consensus direction (ave)
%	for (i=0; i<n-1; i++) { Cells *b1,*b2, *p1,*p2; float d;
%		b1 = cell->child[i];
%		b2 = cell->child[i+1];
%		if (b1->link[0]==0) continue;
%		if (b2->link[0]==0) continue;
%		p1 = b1->link[0];
%		p2 = b2->link[0];
%		if (b1->id == b2->id-1 && p1->id == p2->id+1) {
%			vave(b1->xyz,p1->xyz,&x);
%			vave(b2->xyz,p2->xyz,&y);
%			vsub(x,y,&z); vnorm(&z);
%			d = vdot(z,ave);
%			if (d < -0.2) { // wrong way so swap
%				vcopy(b1->xyz,&x); vcopy(b2->xyz,&(b1->xyz)); vcopy(x,&(b2->xyz));
%				vcopy(p1->xyz,&x); vcopy(p2->xyz,&(p1->xyz)); vcopy(x,&(p2->xyz));
%			}
%		}
%		if (b1->id == b2->id+1 && p1->id == p2->id-1) {
%			vave(b1->xyz,p1->xyz,&x);
%			vave(b2->xyz,p2->xyz,&y);
%			vsub(x,y,&z); vnorm(&z);
%			d = vdot(z,ave);
%			if (d < -0.2) { // wrong way so swap
%				vcopy(b1->xyz,&x); vcopy(b2->xyz,&(b1->xyz)); vcopy(x,&(b2->xyz));
%				vcopy(p1->xyz,&x); vcopy(p2->xyz,&(p1->xyz)); vcopy(x,&(p2->xyz));
%			}
%		}
%	}
%}
%
%groupCell (Cells *cell, int level)
%{
%Vec	centre, shift;
%int	i,j,k, m,n, id, type, sort;
%float	far = -cell->far/(float)data[3];
%float	strict, bias = (float)(cell->level-1);
%	if ((int)cell->endN.z==1234 || (int)cell->endC.z==1234) return;
%	if (cell->empty) return;
%	n = cell->kids;
%	if (n==0) return;
%	id = cell->id;
%	type = cell->type;
%	sort = cell->sort;
%	if (type<0) type = -type;
%	bias = 1.0 - exp(-bias*bias*0.1); // small for top levels to increase selection
%	if (level > 0) shifter(cell);
%	// readjust centres (except for world) going down
%	//
%	for (i=0; i<n; i++) {
%		groupCell(cell->child[i],level+1);
%	}
%	//
%	// gather lost chlidren back to family coming up
%	if (data[0] <= 0) return;
%	if (cell->solid > 0) bias = 2.0; else bias = 0.5; // 4x more like to check cthru cell
%	if (drand48()*bias > exp(far)) return; // distant cells (small exp(far)) checked less
%	// set the model parameters
%	model = cell->model;
%	if (level==0) model = cell->child[0]->model; // world model type is ambiguous
%	m = model*M*N+N;
%	moltype = data[m];
%	align = data+m+N*1; class = data+m+N*2;
%        sizes = data+m+N*3; bumps = data+m+N*4; links = data+m+N*5; chain = data+m+N*6;
%	kicks = data+m+N*7; keeps = data+m+N*8; repel = data+m+N*9; bonds = data+m+N*10;
%	for (i=0; i<N; i++) keep[i] = 0.01*(float)keeps[i];
%	if (rna && type==2 && sort==1) sortTube(cell);
%	for (i=0; i<n; i++)
%	{ Cells *child = cell->child[i];
%	  int	lost = 0; float balance;
%	  	if (data[0] < 0) break;  // don't push before setup is done
%		if (chain[level] > 0 && chain[level+1] > 0) {   // in a chain of chains (don't push ends in)
%			if (child == cell->starts || child == cell->finish) continue;
%		}
%		strict = 1.0;
%		//if (child->type==2 && child->sort==0) continue; // don't push loops
%		switch (type) {
%			case 0 :	// dummy sphere
%				lost = packBall(cell,child,1.0);
%			break;
%			case 1 :	// simple sphere
%				lost = packBall(cell,child,strict);
%			break;
%			case 2 :	// SSE (tube)
%				if (sort && cell->endN.z+cell->endC.z < 1000.0) {
%					vave(cell->endN,cell->endC, &(cell->xyz));
%				}
%//				{ Vec	shift, mid; // bring endpoints in line
%//				  float len = 0.1*(float)sizes[level];
%//					if (sort==1) len *= 3.0/2.0;
%//					vave(cell->endN,cell->endC, &mid);
%//					vsub(cell->xyz,mid, &shift);
%//					vsum(shift, &(cell->endN));
%//					vsum(shift, &(cell->endC));
%//					separate(&(cell->endN),&(cell->endC),len,0.1);
%//				}
%				if (moltype==0) { // protein
%					if (sort==0) strict = 0.01; // loose loop
%					if (sort==1) strict = 1.00; // tight for alpha
%					if (sort==2) strict = 0.10; // looser for beta (to allow bend)
%				}
%				if (moltype==1) { // nucleic
%					strict = 1.0;
%				}
%				lost = packTube(cell,child,strict);
%				if (sort==0) lost = 0; // allow lost children in loops
%			break;
%			case 3 :	// Ellipsoid
%				if (cell->sort < E/2-1 || cell->sort > E/2+1) {
%					lost = packEgg(cell,child,1.0);
%				} else {	// effectively a sphere
%					lost = packBall(cell,child,1.0);
%				}
%			break;
%		}
%	}
%}
%*/
