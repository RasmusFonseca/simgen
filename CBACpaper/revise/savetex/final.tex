\subsection{Summary}

The behaviour of the current method has been investigated using two distinct
approaches: in the examples involving collisions, a constant driving displacement
was applied to a single high-level object (to force it to make contact with 
another object), by contrast,  in the examples involving constraint satisfaction the 
displacements were instead applied to the lowest level objects ("atoms").  In the first
case, the response of the lower level objects to maintain structural integrity was monitored
whereas in the second, the resulting rearrangement of the higher levels objects was of interest.

\subsubsection{Collision algorithms}

The algorithm described in this work for the interaction of a hierarchy of objects
seeks to circumvent a fundamental problem in coarse-grained modelling which
is the loss of fine detail when components become 'bundled' together.
The "currants-in-jelly" model developed here provides a flexible approach in which
the contribution of the soft high-level objects (jelly-like) can be controlled
to protect the underlying atomic structure (currants) while still allowing
them to interact.

Idealised macromolecular chains were used to establish the parameters
to achieve this degree of interaction over a hierarchy spanning four levels.
In a more realistic example using a small globular protein, the extent of the 
distortion experienced by the protein domain structure during collision was then 
examined and the parameters refined to allow an acceptable degree of deformation.

\subsubsection{Constraint satisfaction}

The model of steric repulsion established for idealised systems was then
combined with sets of predicted distance constraints, derived from correlated
mutation analysis,  in two differing applications.    Firstly, an integral
trans-membrane protein was modelled in which the packing of the seven
helices was refined but without topological rearrangement.  Secondly,
an RNA structure was 'folded' under the predicted constraints, starting only
from its 2-dimensional secondary structure prediction.

From the large sequence alignment available for the membrane protein, high
quality distance predictions could be obtained, which combined with good
starting configurations (provided by a simplified lattice model) led to 
the production of high-scoring models with a low RMSD to the known structure.
By contrast, the RNA structure predictions had high RMSD values and although
the stem-loops were correctly located in the higher-scoring models, topological
differences remained which may be a result of constraints imposed by the
flat starting conformation.

\subsection{Limitations and potential}

Whilst the methods and parameters established here have been shown to be
effective, it is unlikely that they are optimal and many aspects of the
model remain to be explored.    The approach taken above was to pick
reasonable values for the extent and strength of an interaction and test
a few variations and combinations in the surrounding parameter-space.
A more systematic approach is needed, and for this the speed of the current
method is a great advantage as many simulations can be run, allowing the
parameter space to be more fully explored and optimal combinations found
for different types of macromolecule.

An earlier, simpler, version of the current method was used to model the
dynamic interaction of actin and myosin.   However, because the atomic (residue)
surfaces did not interact directly in that model, artificial constraints had
to be added to drive the molecular recognition events.  In the current
method, the residue-level surfaces can now come in contact, allowing a more
realistic representation of molecular recognition based on steric compatibility.
However, without a proper atomic interaction potential, the recognition of 
a binding event would still rely on external knowledge of the residues involved.

It was assumed that the parameters used at the start of the simulation remained
the same throughout and equal for all components independently of their
position or interactions.   However, all these aspects could be varied and
if there is a pair of objects that interact preferentially, their parameters
can be set to allow their surfaces to meet whereas others might present a
hard-shell repulsion.  For example, in the actin/myosin example mentioned
above, the actin monomers close to the myosin head could allow atomic
interactions whereas all other interactions retain a high-level repulsion.

The effectiveness of the approach relies on having a rich multi-layer
hierarchy of substructures.   Fortunately, as seen from the varied examples
provided, this is the prevalent situation for the majority of biological
macromolecules.   In the simulation of more homogeneous materials, such
as water, it is unlikely that the approach would be useful.  Similarly, for
a long chain, such as DNA, although segments can be grouped like a string
of sausages, the interaction of two such chains would require evaluation
of the all pairs of sausages.

In large molecules, especially those composed of subunits, the components may not
remain the same throughout a simulation and during an interaction, one subunit may
be transferred to another assembly or a large multi-domain chain may be cleaved.
In its current formulation, the program does not accommodate this, however, such
events could easily be incorporated simply by updating the list of children held
by each object or adding new children to the list. 

\subsection{Relationship to other methods}

Given a model in which all the constraints have been chosen well to avoid bond
or bump violations at all levels, then in the absence of any user applied displacements,
the behaviour of the current method is to do nothing.
It is left entirely up to the user how things should move, which is done by implementing
custom code in what is called the "driver" routine.    This code can be either very
simple, such as the few lines of code needed to implement the collision displacement or
quite complicated such as the code to apply the distance constraints described in the
Results section.

Extrapolating this progression, the constraints could be applied in the form of a potential and
indeed a potential could even be applied to many pairs of atoms \cite{PerioleXet09}.   The next
obvious step would be to implement full molecular dynamics, Lagrangian mechanics or Monte Carlo.
However, the problem with implementing anything complicated in the driver routine is that at some
point, the current method will start to act on the (common) coordinates and if the
driver code requires consistency in terms of distances and derived potentials, forces and
velocities, then a multitude of problems will arise.

The simplest approach to this problem is to attach a warning notice stating that \NAME\ cannot
be used in combination with any method that requires global internal consistency.   The only
route that might avoid this incompatibility would be through a more stochastic approach
in which the displacements made at all levels are treated as a (semi) random Monte Carlo
move of the system.
A second path to resolve the problem might be through a variation of Gaussian elastic
networks, However, elastic
networks require a fixed topology that cannot be expected to remain intact across the
large (driven) displacements envisaged for the current method.

These possible developments will be reconsidered at a later time as it is currently
unclear, not only how a consistent potential could be applied but also how potentials
on different levels should interact.

\subsection{Conclusion}

The method developed here provides a fast and flexible way to capture the
structure of most macromolecules in a hierarchy of increasingly larger coarse-grained
levels without losing the detailed low-level representation.   Although much testing
remains to be done, the system has the potential to be applied to very large dynamic
systems including both protein and nucleic acids.
