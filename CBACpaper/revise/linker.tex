\section{Bonds and links}

The maintenance of bond and link lengths is very similar and the two routines,
\TT{bonder} and \TT{linker}, that implement this task will be considered together.
Both recursively traverse the object tree looking for things to fix.

\subsection{\TT{bonder}}

\subsubsection{Bond lengths}

The \TT{bonder} simply checks if an object has any assigned bonds and if so, uses
the utility {\tt part2cells()} to push them towards their assigned bond length.

\subsubsection{Nucleic acid exceptions}

Exceptions need to be made when bonding tubes in nucleic acids, which occur both at
the secondary structure level as basepairs and the domain level
as segments of double helix. 

For basepairs, if these are part of a double helix, their 'bond-length' is the distance
between their mid-points (object centre) which is refined to an ideal base-stacking separation.  
Outside a base-pair, say in a loop region, the 'secondary structure', like a loop in a protein,
can contain multiple nucleotides and no bond length is refined.

At the domain level, double-stranded DNA segments will always be bonded end-to-end
at a specific distance that allows the helix to run continuously from one segment to
the next.   On the other hand, when the segment is an RNA stem-loop, the chain can enter
and exit the same end of the tube or, with an insertion, even through the side.


\subsection{\TT{linker}}

\subsubsection{Breaking links}

The \TT{linker} follows the same basic outline as the \TT{bonder} but with the main
difference that links can be made and broken during the simulation.   The dynamic creation
of links is not a built-in feature of \NAME\ and must be provided through the user-supplied
\TT{driver} routine.   However, if a link becomes over stretched, it is automatically destroyed in
the \TT{linker}.    The default length of a link is the bump diameter and the default extension
is 50\%, beyond which the link breaks.

\subsubsection{Preset link lengths}

Local links are automatically created for standard secondary structures,  not only between the H-bonded
connections in the \AH, i---i+3 and i---i+4, but also between the i-1---i+1 separation along a \Bs.
However, the non-local links between strands in a \BS\ must be user defined.
